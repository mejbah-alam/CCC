% Setting up the document class
\documentclass[a4paper,14pt]{article}
\usepackage[utf8]{inputenc}
\usepackage[T1]{fontenc}
\usepackage{lmodern}
\usepackage{geometry}
\geometry{margin=1in}
\usepackage{listings}
\usepackage{xcolor}
\usepackage{amsmath}
\usepackage{parskip}
\usepackage{fancyhdr}
\usepackage{lastpage}
\usepackage{enumitem}
\usepackage{hyperref}
\usepackage{titlesec}
\usepackage{tocloft}
\usepackage{ulem}

% Configuring the listings package for C code
\lstset{
    language=C,
    basicstyle=\ttfamily\small,
    keywordstyle=\color{blue}\bfseries,
    stringstyle=\color{red},
    commentstyle=\color{green!60!black}\itshape,
    numbers=none,
    showspaces=false,
    showstringspaces=false,
    frame=single,
    breaklines=true,
    breakatwhitespace=true,
    tabsize=4,
    captionpos=b
}

% Customizing the table of contents
\renewcommand{\cftsecleader}{\cftdotfill{\cftdotsep}}
\renewcommand{\cftsecfont}{\bfseries}
\renewcommand{\cftsubsecfont}{\itshape}
\renewcommand{\cftsubsecleader}{\cftdotfill{\cftdotsep}}
\renewcommand{\cftsubsecnumwidth}{4em}

% Setting up fonts
\usepackage{utopia}

% Title and author
\title{C Programming Exercises}
\author{Compiled by Mejbah}
\date{May 7, 2025}

% Configuring hyperref to hide link borders
\hypersetup{
    hidelinks
}

\begin{document}

% Generating the title page
\maketitle

% Adding a table of contents
\tableofcontents

% Extra newpage causing additional page
\newpage

% Configuring page headers and footers
\pagestyle{fancy}
\fancyhf{}
\fancyhead[L]{ Programming Exercises}
\fancyhead[R]{\thepage\ of \pageref{LastPage}}
\fancyfoot[C]{}

\setcounter{page}{1}

%Introduction
\section{Introduction}
This document provides C programs for 78 programming exercises, covering arithmetic operations, mathematical functions, and conditional logic. Each program includes the source code and a brief description, presented on a separate page for clarity. Users can compile and run these programs using a C compiler (e.g., GCC) to observe the output. The programs are presented in the order specified in the exercise set.
\newpage


\section{Basic Programming Exercises}


\subsection{ASCII Value to ASCII Character}
\vspace{0.5cm}
\begin{lstlisting}[caption={ASCII Value to ASCII Character}]
#include<stdio.h>
int main()
{
    int n;
    printf("Enter Any ASCII Value: ");
    scanf("%d",&n);
    printf("The ASCII Character is: %c",n);
    return 0;
}
\end{lstlisting}
\newpage

\subsection{ASCII Character to ASCII Value}
\vspace{0.5cm}
\begin{lstlisting}[caption={ASCII Character to ASCII Value}]
#include<stdio.h>
int main()
{
    char ch;
    printf("Enter Any ASCII Character: ");
    scanf("%c",&ch);
    printf("The ASCII Value is: %d",ch);
    return 0;
}
\end{lstlisting}
\newpage

\subsection{Lower Case to Upper Case}
\vspace{0.5cm}
\begin{lstlisting}[caption={Lower Case to Upper Case}]
#include<stdio.h>
int main()
{
    char lower;
    printf("Enter Any Lowercase Letter: ");
    scanf("%c",&lower);
    printf("The Uppercase Letter: %c",lower-32);
    return 0;
}
\end{lstlisting}
\newpage

\subsection{Upper Case to Lower Case}
\vspace{0.5cm}
\begin{lstlisting}[caption={Upper Case to Lower Case}]
#include<stdio.h>
int main()
{
    char Upper;
    printf("Enter Any Uppercase Letter: ");
    scanf("%c",&Upper);
    printf("The Lowercase Letter: %c",Upper+32);
    return 0;
}
\end{lstlisting}
\newpage

\subsection{Lower to Upper Case Using Library Function}
\vspace{0.5cm}
\begin{lstlisting}[caption={Lower to Upper Case Using Library Function}]
#include<stdio.h>
int main()
{
    char lower,upper;
    printf("Enter Any Lowercase Letter: ");
    scanf("%c",&lower);
    upper=toupper(lower);
    printf("Uppercase Letter = %c",upper);
    return 0;
}
\end{lstlisting}
\newpage

\subsection{Upper Case to Lower Case Using Library Function}
\vspace{0.5cm}
\begin{lstlisting}[caption={Upper Case to Lower Case Using Library Function}]
#include<stdio.h>
int main()
{
    char lower,upper;
    printf("Enter Any Uppercase Letter: ");
    scanf("%c",&upper);
    lower=tolower(upper);
    printf("Lowercase Letter = %c",lower);
    return 0;
}
\end{lstlisting}
\newpage

\subsection{Decimal to Octal}
\vspace{0.5cm}
\begin{lstlisting}[caption={Decimal to Octal}]
#include<stdio.h>
int main()
{
    int number;
    printf("Enter Any Decimal Number: ");
    scanf("%d",&number);
    printf("Octal number = %o",number);
    return 0;
}
\end{lstlisting}
\newpage

\subsection{Octal to Decimal}
\vspace{0.5cm}
\begin{lstlisting}[caption={Octal to Decimal}]
#include<stdio.h>
int main()
{
    int number;
    printf("Enter Any Octal Number: ");
    scanf("%d",&number);
    printf("Decimal number = %d",number);
    return 0;
}
\end{lstlisting}
\newpage

\subsection{Decimal to HexaDecimal}
\vspace{0.5cm}
\begin{lstlisting}[caption={Decimal to HexaDecimal}]
#include<stdio.h>
int main()
{
    int number;
    printf("Enter Any Decimal Number: ");
    scanf("%d",&number);
    printf("HexaDecimal number = %x",number);
    return 0;
}
\end{lstlisting}
\newpage

\subsection{HexaDecimal to Decimal}
\vspace{0.5cm}
\begin{lstlisting}[caption={HexaDecimal to Decimal}]
#include<stdio.h>
int main()
{
    int number;
    printf("Enter Any HexaDecimal Number: ");
    scanf("%x",&number);
    printf("Decimal number = %d",number);
    return 0;
}
\end{lstlisting}
\newpage

\subsection{Octal to HexaDecimal}
\vspace{0.5cm}
\begin{lstlisting}[caption={Octal to HexaDecimal}]
#include<stdio.h>
int main()
{
    int number;
    printf("Enter any Octal Number: ");
    scanf("%o",&number);
    printf("HexaDecimal number = %x",number);
    return 0;
}
\end{lstlisting}
\newpage

\subsection{HexaDecimal to Octal}
\vspace{0.5cm}
\begin{lstlisting}[caption={HexaDecimal to Octal}]
#include<stdio.h>
int main()
{
    int number;
    printf("Enter any HexaDecimal Number: ");
    scanf("%x",&number);
    printf("Octal number = %o",number);
    return 0;
}
\end{lstlisting}
\newpage


\section{Arithmetic Operator Programming Exercises}


\subsection{Sum of Two Numbers}
\vspace{0.5cm}
\begin{lstlisting}[caption={Sum of Two Numbers}]
#include<stdio.h>
int main()
{
    int num1,num2,sum;
    printf("Enter Two Numbers: ");
    scanf("%d %d",&num1,&num2);
    sum=num1+num2;
    printf("The Sum is: %d\n",sum);
    return 0;
}
\end{lstlisting}
\newpage

\subsection{Sum And Average of Two Numbers}
\vspace{0.5cm}
\begin{lstlisting}[caption={Sum And Average of Two Numbers}]
#include<stdio.h>
int main()
{
    int num1,num2,sum;
    float avg;
    printf("Enter Two Numbers: ");
    scanf("%d %d",&num1,&num2);
    sum=num1+num2;
    printf("The Sum is: %d\n",sum);
    avg=(float)sum/2;		//Typecasting
    printf("The Average is: %f\n",avg);
    return 0;
}
\end{lstlisting}
\newpage

\subsection{All Arithmetic Operators}
\vspace{0.5cm}
\begin{lstlisting}[caption={All Arithmetic Operators}]
#include<stdio.h>
int main()
{
    int num1,num2;
    printf("Enter Two Numbers: ");
    scanf("%d %d",&num1,&num2);
    int sum=num1+num2;
    printf("The Sum is: %d\n",sum);
    int sub=num1-num2;
    printf("The Sub is: %d\n",sub);
    int mul=num1*num2;
    printf("The Mul is: %d\n",mul);
    double div=(float)num1/num2;	// Typecasting
    printf("The Div is: %lf\n",div);
    int rem=num1%num2;
    printf("The Rem is: %d\n",rem);
    return 0;
}
\end{lstlisting}
\newpage

\subsection{Sum of Three Numbers}
\vspace{0.5cm}
\begin{lstlisting}[caption={Sum of Three Numbers}]
#include<stdio.h>
int main()
{
    int num1,num2,num3,sum;
    printf("Enter Three Numbers: ");
    scanf("%d %d %d",&num1,&num2,&num3);
    sum=num1+num2+num3;
    printf("The Sum is: %d\n",sum);
    return 0;
}
\end{lstlisting}
\newpage

\subsection{Sum And Average of Three Numbers}
\vspace{0.5cm}
\begin{lstlisting}[caption={Sum And Average of Three Numbers}]
#include<stdio.h>
int main()
{
    int num1,num2,num3,sum;
    float average;
    printf("Enter Three Numbers: ");
    scanf("%d %d %d",&num1,&num2,&num3);
    sum=num1+num2+num3;
    average=(float)sum/3;
    printf("The Sum is: %d\n",sum);
    printf("The Average is: %f\n",average);
    return 0;
}
\end{lstlisting}
\newpage

\subsection{Area of Triangle}
\vspace{0.5cm}
\begin{lstlisting}[caption={Area of Triangle}]
#include<stdio.h>
int main()
{
    float base,height,area;
    printf("Enter Base: ");
    scanf("%f",&base);
    printf("Enter Height: ");
    scanf("%f",&height);
    area=(float)1/2*base*height;
    printf("The Area is: %f",area);
    return 0;
}
\end{lstlisting}
\newpage

\subsection{Area of Triangle}
\vspace{0.5cm}
\begin{lstlisting}[caption={Area of Triangle}]
#include<stdio.h>
int main()
{
    float base,height,area;
    printf("Enter Base: ");
    scanf("%f",&base);
    printf("Enter Height: ");
    scanf("%f",&height);
    area=0.5*base*height;
    printf("The Area is: %f",area);
    return 0;
}
\end{lstlisting}
\newpage

\subsection{Area of Triangle Using Sides}
\vspace{0.5cm}
\begin{lstlisting}[caption={Area of Triangle Using Sides}]
#include<stdio.h>
int main()
{
    float a,b,c,s,area;
    printf("The Sides of Triangle are: ");
    scanf("%f %f %f",&a,&b,&c);
    s=(a+b+c)/2;
    area=sqrt(s*(s-a)*(s-b)*(s-c));
    printf("The Area is: %f",area);
    return 0;
}
\end{lstlisting}
\newpage

\subsection{Area of Rectangle}
\vspace{0.5cm}
\begin{lstlisting}[caption={Area of Rectangle}]
#include<stdio.h>
int main()
{
    float length,width,area;
    printf("Enter Length: ");
    scanf("%f",&length);
    printf("Enter Width: ");
    scanf("%f",&width);
    area=length*width;
    printf("The Area is: %f",area);   
    return 0;
}
\end{lstlisting}
\newpage

\subsection{Area of Circle}
\vspace{0.5cm}
\begin{lstlisting}[caption={Area of Circle}]
#include<stdio.h>
#define PI 3.14159
int main()
{
    float radius,area;
    printf("Enter Radius: ");
    scanf("%f",&radius);
    area=PI*radius*radius;
    printf("The Area is: %f",area);
    return 0;
}
\end{lstlisting}
\newpage

\subsection{Area of Circle Using Math.h}
\vspace{0.5cm}
\begin{lstlisting}[caption={Area of Circle Using Math.h}]
#include<stdio.h>
#include<math.h>
int main()
{
    float radius,area;
    printf("Enter Radius: ");
    scanf("%f",&radius);
    area=M_PI*radius*radius;
    printf("The Area is: %f",area);
    return 0;
}
\end{lstlisting}
\newpage

\subsection{Celsius to Fahrenheit}
\vspace{0.5cm}
\begin{lstlisting}[caption={Celsius to Fahrenheit}]
#include<stdio.h>
int main()
{
    float C,F;
    printf("Enter Celsius: ");
    scanf("%f",&C);
    F=((9*C/5)+32);
    printf("Temperature in Fahrenheit is: %f",F);
    return 0;
}
\end{lstlisting}
\newpage

\subsection{Fahrenheit to Celsius}
\vspace{0.5cm}
\begin{lstlisting}[caption={Fahrenheit to Celsius}]
#include<stdio.h>
int main()
{
    float C,F;
    printf("Enter Fahrenheit: ");
    scanf("%f",&F);
    C=((F-32)*5)/9;
    printf("Temperature in Celsius is: %f",C);
    return 0;
}
\end{lstlisting}
\newpage

\subsection{Number Swap With Temporary Variable}
\vspace{0.5cm}
\begin{lstlisting}[caption={Number Swap With Temporary Variable}]
#include<stdio.h>
int main()
{
    int num1,num2,temp;
    printf("Enter Number 01: ");
    scanf("%d",&num1);
    printf("Enter Number 02: ");
    scanf("%d",&num2);
    temp=num1;
    num1=num2;
    num2=temp;
    printf("Number 01: %d\n",num1);
    printf("Number 02: %d",num2);
    return 0;
}
\end{lstlisting}
\newpage

\subsection{Number Swap Without Temporary Variable}
\vspace{0.5cm}
\begin{lstlisting}[caption={Number Swap Without Temporary Variable}]
#include<stdio.h>
int main()
{
    int num1,num2;
    printf("Enter 1st Number: ");
    scanf("%d",&num1);
    printf("Enter 2nd Number: ");
    scanf("%d",&num2);
    num1=num1-num2;
    num2=num1+num2;
    num1=num2-num1;
    printf("Number 01: %d\n",num1);
    printf("Number 02: %d",num2);
    return 0;
}
\end{lstlisting}
\newpage

\subsection{Quadratic Formula}
\vspace{0.5cm}
\begin{lstlisting}[caption={Quadratic Formula}]
#include<stdio.h>
#include<math.h>
int main()
{
    double a,b,c,D,X1,X2;
    printf("Enter the Value of a: ");
    scanf("%lf",&a);
    printf("Enter the Value of b: ");
    scanf("%lf",&b);
    printf("Enter the Value of c: ");
    scanf("%lf",&c);
    D=sqrt((b*b)-(4*a*c));
    X1=(-b+D)/(2*a);
    X2=(-b-D)/(2*a);
    printf("The Value of X1: %lf",X1);
    printf("The Value of X2: %lf",X2);
    return 0;
}
\end{lstlisting}
\newpage


\section{Mathematical Operator Programming Exercises}


\subsection{Value Of Absolute}
\vspace{0.5cm}
\begin{lstlisting}[caption={Value Of Absolute}]
#include<stdio.h>
#include<math.h>
int main()
{
    int x,result;
    printf("Enter the Number: ");
    scanf("%d",&x);
    result=abs(x);
    printf("The Value of Absolute: %d",result);
    return 0;
}
\end{lstlisting}
\newpage

\subsection{Value Of Square Root}
\vspace{0.5cm}
\begin{lstlisting}[caption={Value Of Square Root}]
#include<stdio.h>
#include<math.h>
int main()
{
    double x,result;
    printf("Enter the Number: ");
    scanf("%lf",&x);
    result=sqrt(x);
    printf("The Value of Square Root: %lf",result);
    return 0;
}
\end{lstlisting}
\newpage

\subsection{Value Of Power}
\vspace{0.5cm}
\begin{lstlisting}[caption={Value Of Power}]
#include<stdio.h>
#include<math.h>
int main()
{
    int x,y;
    double result;
    printf("Enter the Base: ");
    scanf("%d",&x);
    printf("Enter the Power: ");
    scanf("%d",&y);
    result=pow(x,y);
    printf("The Result is: %lf",result);
    return 0;
}
\end{lstlisting}
\newpage

\subsection{Value Of Log}
\vspace{0.5cm}
\begin{lstlisting}[caption={Value Of Log}]
#include<stdio.h>
#include<math.h>
int main()
{
    double x,result;
    printf("Enter the Value: ");
    scanf("%lf",&x);
    result=log(x);
    printf("The Answer is: %lf",result);
    return 0;
}
\end{lstlisting}
\newpage

\subsection{Value Of Log 10}
\vspace{0.5cm}
\begin{lstlisting}[caption={Value Of Log 10}]
#include<stdio.h>
#include<math.h>
int main()
{
    double x,result;
    printf("Enter the Value: ");
    scanf("%lf",&x);
    result=log10(x);
    printf("The Answer is: %lf",result);
    return 0;
}
\end{lstlisting}
\newpage

\subsection{Value Of Exponential Function}
\vspace{0.5cm}
\begin{lstlisting}[caption={Value Of Exponential Function}]
#include<stdio.h>
#include<math.h>
int main()
{
    double x,result;
    printf("Enter the Value: ");
    scanf("%lf",&x);
    result=exp(x);
    printf("The Answer is: %lf",result);
    return 0;
}
\end{lstlisting}
\newpage

\subsection{Value Of Sine}
\vspace{0.5cm}
\begin{lstlisting}[caption={Value Of Sine}]
#include<stdio.h>
#include<math.h>
int main()
{
    double x,result;
    printf("Enter the Value: ");
    scanf("%lf",&x);
    result=sin(x);
    printf("The Answer is: %lf",result);
    return 0;
}
\end{lstlisting}
\newpage

\subsection{Value Of Cosine}
\vspace{0.5cm}
\begin{lstlisting}[caption={Value Of Cosine}]
#include<stdio.h>
#include<math.h>
int main()
{
    double x,result;
    printf("Enter the Value: ");
    scanf("%lf",&x);
    result=cos(x);
    printf("The Answer is: %lf",result);
    return 0;
}
\end{lstlisting}
\newpage

\subsection{Value Of Tangent}
\vspace{0.5cm}
\begin{lstlisting}[caption={Value Of Tangent}]
#include<stdio.h>
#include<math.h>
int main()
{
    double x,result;
    printf("Enter the Value: ");
    scanf("%lf",&x);
    result=tan(x);
    printf("The Answer is: %lf",result);
    return 0;
}
\end{lstlisting}
\newpage

\subsection{Value Of Round}
\vspace{0.5cm}
\begin{lstlisting}[caption={Value Of Round}]
#include<stdio.h>
#include<math.h>
int main()
{
    double x,result;
    printf("Enter the Value: ");
    scanf("%lf",&x);
    result=round(x);
    printf("The Answer is: %lf",result);
    return 0;
}
\end{lstlisting}
\newpage

\subsection{Value Of Trunc}
\vspace{0.5cm}
\begin{lstlisting}[caption={Value Of Trunc}]
#include<stdio.h>
#include<math.h>
int main()
{
    double x,result;
    printf("Enter the Value: ");
    scanf("%lf",&x);
    result=trunc(x);
    printf("The Answer is: %lf",result);
    return 0;
}
\end{lstlisting}
\newpage

\subsection{Value Of Ceiling}
\vspace{0.5cm}
\begin{lstlisting}[caption={Value Of Ceiling}]
#include<stdio.h>
#include<math.h>
int main()
{
    double x,result;
    printf("Enter the Value: ");
    scanf("%lf",&x);
    result=ceil(x);
    printf("The Answer is: %lf",result);
    return 0;
}
\end{lstlisting}
\newpage

\subsection{Value Of Floor}
\vspace{0.5cm}
\begin{lstlisting}[caption={Value Of Floor}]
#include<stdio.h>
#include<math.h>
int main()
{
    double x,result;
    printf("Enter the Value: ");
    scanf("%lf",&x);
    result=floor(x);
    printf("The Answer is: %lf",result);
    return 0;
}
\end{lstlisting}
\newpage


\section{Assignment Operator Programming Exercises}


\subsection{All Assignment Operators}
\vspace{0.5cm}
\begin{lstlisting}[caption={All Assignment Operators}]
#include<stdio.h>
int main()
{
    int a,b,c,d;
    printf("Enter the Value of a: ");
    scanf("%d",&a);
    printf("Enter the Value of b: ");
    scanf("%d",&b);
    printf("Enter the Value of c: ");
    scanf("%d",&c);
    printf("Enter the Value of d: ");
    scanf("%d",&d);
    a+=2;
    printf("The Value of += is: %d\n",a);
    b-=2;
    printf("The Value of -= is: %d\n",b);
    c*=2;
    printf("The Value of *= is: %d\n",c);
    d/=2;
    printf("The Value of /= is: %d\n",d);
    return 0;
}
\end{lstlisting}
\newpage

\subsection{Unary Plus}
\vspace{0.5cm}
\begin{lstlisting}[caption={Unary Plus}]
#include<stdio.h>
int main()
{
    int x,result;
    printf("Enter the Integer: ");
    scanf("%d",&x);
    result=+x;
    printf("The Unary Value is: %d",result);
    return 0;
}
\end{lstlisting}
\newpage

\subsection{Unary Minus}
\vspace{0.5cm}
\begin{lstlisting}[caption={Unary Minus}]
#include<stdio.h>
int main()
{
    int x,result;
    printf("Enter the Integer: ");
    scanf("%d",&x);
    result=-x;
    printf("The Unary Value is: %d",result);
    return 0;
}
\end{lstlisting}
\newpage

\subsection{Prefix Increment}
\vspace{0.5cm}
\begin{lstlisting}[caption={Prefix Increment}]
#include<stdio.h>
int main()
{
    int X,Y;
    printf("Enter the Integer: ");
    scanf("%d",&X);
    Y=++X;
    printf("The Value of X: %d\n",X);
    printf("The Value of Y: %d",Y)
    return 0;
}
\end{lstlisting}
\newpage

\subsection{Postfix Increment}
\vspace{0.5cm}
\begin{lstlisting}[caption={Postfix Increment}]
#include<stdio.h>
int main()
{
    int X,Y;
    printf("Enter the Integer: ");
    scanf("%d",&X);
    Y=X++;
    printf("The Value of X: %d\n",X);
    printf("The Value of Y: %d",Y);
    return 0;
}
\end{lstlisting}
\newpage

\subsection{Prefix Decrement}
\vspace{0.5cm}
\begin{lstlisting}[caption={Prefix Decrement}]
#include<stdio.h>
int main()
{
    int X,Y;
    printf("Enter the Integer: ");
    scanf("%d",&X);
    Y=--X;
    printf("The Value of X: %d\n",X);
    printf("The Value of Y: %d",Y);
    return 0;
}
\end{lstlisting}
\newpage

\subsection{Postfix Decrement}
\vspace{0.5cm}
\begin{lstlisting}[caption={Postfix Decrement}]
#include<stdio.h>
int main()
{
    int X,Y;
    printf("Enter the Integer: ");
    scanf("%d",&X);
    Y=X--;
    printf("The Value of X: %d\n",X);
    printf("The Value of Y: %d",Y);
    return 0;
}
\end{lstlisting}
\newpage

\subsection{Increment And Decrement}
\vspace{0.5cm}
\begin{lstlisting}[caption={Increment And Decrement}]
#include<stdio.h>
int main()
{
    int X;
    printf("Enter the Integer: ");
    scanf("%d",&X);
    printf("The Value of X: %d\n",X++);  // X=X
    printf("The Value of X: %d\n",X);    // X=X+1
    printf("The Value of X: %d\n",++X);  // X=X+2
    printf("The Value of X: %d\n",X);    // X=X+2
    printf("The Value of X: %d\n",X--);  // X=X+2
    printf("The Value of X: %d\n",X);    // X=X-1
    printf("The Value of X: %d\n",--X);  // X=X-2
    printf("The Value of X: %d\n",X);    // X=X-2
    return 0;
}
\end{lstlisting}
\newpage

\subsection{Increment And Decrement}
\vspace{0.5cm}
\begin{lstlisting}[caption={Increment And Decrement}]
#include<stdio.h>
int main()
{
    int X;
    printf("Enter the Integer: ");
    scanf("%d",&X);
    printf("The Value of X: %d\n",X++);  // X=X
    printf("The Value of X: %d\n",++X);  // X=X+2
    printf("The Value of X: %d\n",X--);  // X=X+2
    printf("The Value of X: %d\n",--X);  // X=X
    return 0;
}
\end{lstlisting}
\newpage


\section{If-Else Programming Exercises}


\subsection{Even And Odd Using If}
\vspace{0.5cm}
\begin{lstlisting}[caption={Even And Odd Using If}]
#include<stdio.h>
int main()
{
    int num;
    printf("Enter an Integer: ");
    scanf("%d",&num);
    if(num%2==0)
        printf("The Number is Even\n");
    if(num%2!=0)
        printf("The Number is Odd\n");
    return 0;
}
\end{lstlisting}
\newpage

\subsection{Even or Odd Using If-else}
\vspace{0.5cm}
\begin{lstlisting}[caption={Even or Odd Using If-else}]
#include<stdio.h>
int main()
{
    int num;
    printf("Enter an Integer: ");
    scanf("%d",&num);
    if(num%2==0)
        printf("The Number is Even\n");
    else
        printf("The Number is Odd\n");
    return 0;
}
\end{lstlisting}
\newpage

\subsection{Positive or Negative Using If-else if-else}
\vspace{0.5cm}
\begin{lstlisting}[caption={Positive or Negative Using If-else if-else}]
#include<stdio.h>
int main()
{
    int num;
    printf("Enter an Integer: ");
    scanf("%d",&num);
    if(num>0)
        printf("The Number is Positive\n");
    else if(num<0)
        printf("The Number is Negative\n");
    else
        printf("The Number is Zero\n");
    return 0;
}
\end{lstlisting}
\newpage

\subsection{Largest of Two Numbers Using If-else if-else}
\vspace{0.5cm}
\begin{lstlisting}[caption={Largest of Two Numbers Using If-else if-else}]
#include<stdio.h>
int main()
{
    int num1,num2;
    printf("Enter 1st Number: ");
    scanf("%d",&num1);
    printf("Enter 2nd Number: ");
    scanf("%d",&num2);
    if(num1>num2)
        printf("1st Number is Largest\n");
    else if(num2>num1)
        printf("2nd Number is Largest\n");
    else
        printf("The Numbers are Equal\n");
    return 0;
}
\end{lstlisting}
\newpage

\subsection{Smallest of Two Numbers Using If-else if-else}
\vspace{0.5cm}
\begin{lstlisting}[caption={Smallest of Two Numbers Using If-else if-else}]
#include<stdio.h>
int main()
{
    int num1,num2;
    printf("Enter 1st Number: ");
    scanf("%d",&num1);
    printf("Enter 2nd Number: ");
    scanf("%d",&num2);
    if(num1<num2)
        printf("1st Number is Smallest\n");
    else if(num2<num1)
        printf("2nd Number is Smallest\n");
    else
        printf("The Numbers are Equal\n");
    return 0;
}
\end{lstlisting}
\newpage

\subsection{Grade Marks Using If-else if-else}
\vspace{0.5cm}
\begin{lstlisting}[caption={Grade Marks Using If-else if-else}]
#include<stdio.h>
int main()
{
    float marks;
    printf("Enter Your Marks: ");
    scanf("%f",&marks);
    if(marks>=93)
        printf("Your Grade is A");
    else if(marks>=89)
        printf("Your Grade is A-");
    else if(marks>=86)
        printf("Your Grade is B+");
    else if(marks>=82)
        printf("Your Grade is B");
    else if(marks>=79)
        printf("Your Grade is B-");
    else if(marks>=75)
        printf("Your Grade is C+");
    else if(marks>=72)
        printf("Your Grade is C");
    else if(marks>=69)
        printf("Your Grade is C-");
    else if(marks>=65)
        printf("Your Grade is D+");
    else if(marks>=60)
        printf("Your Grade is D");
    else
        printf("Your Grade is F");
    return 0;
}
\end{lstlisting}
\newpage

\subsection{Vowel Consonant Using If-else if-else}
\vspace{0.5cm}
\begin{lstlisting}[caption={Vowel Consonant Using If-else if-else}]
#include<stdio.h>
int main()
{
    char ch;
    printf("Enter a Letter: ");
    scanf("%c",&ch);
    if(ch=='a' || ch=='A')
        printf("%c is Vowel",ch);
    else if(ch=='e' || ch=='E')
        printf("%c is Vowel",ch);
    else if(ch=='i' || ch=='I')
        printf("%c is Vowel",ch);
    else if(ch=='o' || ch=='O')
        printf("%c is Vowel",ch);
    else if(ch=='u' || ch=='U')
        printf("%c is Vowel",ch);
    else
        printf("%c is Consonant",ch);
    return 0;
}
\end{lstlisting}
\newpage

\subsection{Vowel Consonant Using Library Function}
\vspace{0.5cm}
\begin{lstlisting}[caption={Vowel Consonant Using Library Function}]
#include<stdio.h>
int main()
{
    char ch,temp;
    printf("Enter a Letter: ");
    scanf("%c",&ch);
    temp=ch;
    ch=tolower(ch);
    if(ch=='a' || ch=='e' || ch=='i' || ch=='o' || ch=='u')
        printf("%c is Vowel",temp);
    else
        printf("%c is Consonant",temp);
    return 0;
}
\end{lstlisting}
\newpage

\subsection{Vowel Consonant Using Nested If}
\vspace{0.5cm}
\begin{lstlisting}[caption={Vowel Consonant Using Nested If}]
#include<stdio.h>
int main()
{
    char ch;
    printf("Enter Any Character: ");
    scanf("%c",&ch);
    if((ch>='a' && ch<='z') || (ch>='A' && ch<='Z'))
    {
        if(ch=='a' || ch=='e' || ch=='i' || ch=='o' || ch=='u' 
        || ch=='A' || ch=='E' || ch=='I' || ch=='O' || ch=='U')
            printf("'%c' is Vowel.",ch);
        else
            printf("'%c' is Consonant.",ch);
    }
    else
    {
        printf("'%c' is Not an Alphabet.",ch);
    }
    return 0;
}
\end{lstlisting}
\newpage

\subsection{Largest of Three Numbers Using If-else if-else}
\vspace{0.5cm}
\begin{lstlisting}[caption={Largest of Three Numbers Using If-else if-else}]
#include<stdio.h>
int main()
{
    int num1,num2,num3;
    printf("Enter Three Numbers: ");
    scanf("%d %d %d",&num1,&num2,&num3);
    if((num1>=num2) && (num1>=num3))
        printf("Largest Number is: %d\n",num1);
    else if((num2>=num3) && (num2>=num1))
        printf("Largest Number is: %d\n",num2);
    else if((num3>=num1) && (num3>=num2))
        printf("Largest Number is: %d\n",num3);
    else
        printf("The Numbers are Equal\n");
    return 0;
}
\end{lstlisting}
\newpage

\subsection{Smallest of Three Numbers Using If-else if-else}
\vspace{0.5cm}
\begin{lstlisting}[caption={Smallest of Three Numbers Using If-else if-else}]
#include<stdio.h>
int main()
{
    int num1,num2,num3;
    printf("Enter Three Numbers: ");
    scanf("%d %d %d",&num1,&num2,&num3);
    if((num1<=num2) && (num1<=num3))
        printf("Smallest Number is: %d\n",num1);
    else if((num2<=num3) && (num2<=num1))
        printf("Smallest Number is: %d\n",num2);
    else if((num3<=num1) && (num3<=num2))
        printf("Smallest Number is: %d\n",num3);
    else
        printf("The Numbers are Equal\n");
    return 0;
}
\end{lstlisting}
\newpage

\subsection{Largest of Three Numbers Using Nested If}
\vspace{0.5cm}
\begin{lstlisting}[caption={Largest of Three Numbers Using Nested If}]
#include<stdio.h>
int main()
{
    int num1,num2,num3;
    printf("Enter three numbers: ");
    scanf("%d %d %d",&num1,&num2,&num3);
    if(num1>=num2)
    {
        if(num1>=num3)
            printf("%d is the Largest Number",num1);
        else
            printf("%d is the Largest Number",num3);
    }
    else if(num2>=num3)
    {
        if(num2>=num1)
            printf("%d is the Largest Number",num2);
        else
            printf("%d is the Largest Number",num1);
    }
    else if(num3>=num1)
    {
        if(num3>=num2)
            printf("%d is the Largest Number",num3);
        else
            printf("%d is the Largest Number",num2);
    }
    else
        printf("The Numbers are Equal\n");
    return 0;
}
\end{lstlisting}
\newpage

\subsection{Smallest of Three Numbers Using Nested If}
\vspace{0.5cm}
\begin{lstlisting}[caption={Smallest of Three Numbers Using Nested If}]
#include<stdio.h>
int main()
{
    int num1,num2,num3;
    printf("Enter three numbers: ");
    scanf("%d %d %d",&num1,&num2,&num3);
    if(num1<=num2)
    {
        if(num1<=num3)
            printf("%d is the Smallest Number",num1);
        else
            printf("%d is the Smallest Number",num3);
    }
    else if(num2<=num3)
    {
        if(num2<=num1)
            printf("%d is the Smallest Number",num2);
        else
            printf("%d is the Smallest Number",num1);
    }
    else if(num3<=num1)
    {
        if(num3<=num2)
            printf("%d is the Smallest Number",num3);
        else
            printf("%d is the Smallest Number",num2);
    }
    else
        printf("The Numbers are Equal\n");
    return 0;
}
\end{lstlisting}
\newpage

\subsection{Largest of Three Numbers Using Logical Operator}
\vspace{0.5cm}
\begin{lstlisting}[caption={Largest of Three Numbers Using Logical Operator}]
#include<stdio.h>
int main()
{
    int num1,num2,num3;
    printf("Enter 1st number: ");
    scanf("%d",&num1);
    printf("Enter 2nd number: ");
    scanf("%d",&num2);
    printf("Enter 3rd number: ");
    scanf("%d",&num3);
    if(num1>num2 && num1>num3)
        printf("\n1st Number is Greatest: %d",num1);
    else if(num2>num1 && num2>num3)
        printf("\n2nd Number is Greatest: %d",num2);
    else if(num3>num1 && num3>num2)
        printf("\n3rd Number is Greatest: %d",num3);
    else if(num1==num2 && num1>num3)
        printf("\n1st & 2nd Number is Greatest: %d",num1);
    else if(num1==num2 && num3>num1)
        printf("\n3rd Number is Greatest: %d",num3);
    else if(num2==num3 && num2>num1)
        printf("\n2nd & 3rd Number is Greatest: %d",num2);
    else if(num2==num3 && num1>num2)
        printf("\n1st Number is Greatest: %d",num1);
    else if(num3==num1 && num3>num2)
        printf("\n1st & 3rd Number is Greatest: %d",num3);
    else if(num3==num1 && num2>num3)
        printf("\n2nd Number is Greatest: %d",num2);
    else
        printf("\nNumbers are Equal");
    return 0;
}
\end{lstlisting}
\newpage

\subsection{Smallest of Three Numbers Using Logical Operator}
\vspace{0.5cm}
\begin{lstlisting}[caption={Smallest of Three Numbers Using Logical Operator}]
#include<stdio.h>
int main()
{
    int num1,num2,num3;
    printf("Enter 1st number: ");
    scanf("%d",&num1);
    printf("Enter 2nd number: ");
    scanf("%d",&num2);
    printf("Enter 3rd number: ");
    scanf("%d",&num3);
    if(num1<num2 && num1<num3)
        printf("\n1st Number is Smallest: %d",num1);
    else if(num2<num1 && num2<num3)
        printf("\n2nd Number is Smallest: %d",num2);
    else if(num3<num1 && num3<num2)
        printf("\n3rd Number is Smallest: %d",num3);
    else if(num1==num2 && num1<num3)
        printf("\n1st & 2nd Number is Smallest: %d",num1);
    else if(num1==num2 && num3<num1)
        printf("\n3rd Number is Smallest: %d",num3);
    else if(num2==num3 && num2<num1)
        printf("\n2nd & 3rd Number is Smallest: %d",num2);
    else if(num2==num3 && num1<num2)
        printf("\n1st Number is Smallest: %d",num1);
    else if(num3==num1 && num3<num2)
        printf("\n1st & 3rd Number is Smallest: %d",num3);
    else if(num3==num1 && num2<num3)
        printf("\n2nd Number is Smallest: %d",num2);
    else
        printf("\nNumbers are Equal");
    return 0;
}
\end{lstlisting}
\newpage

\subsection{Find the Coordinate Quadrant Identification}
\vspace{0.5cm}
\begin{lstlisting}[caption={Find the Coordinate Quadrant Identification}]
#include<stdio.h>
int main()
{
    int co1,co2;
    printf("Input the Values for X & Y Coordinate: ");
    scanf("%d %d",&co1,&co2);
    if(co1>0 && co2>0)
        printf("The Coordinate Point (%d,%d) Lies in the First Quadrant\n",co1,co2);
    else if(co1<0 && co2>0)
        printf("The Coordinate Point (%d,%d) Lies in the Second Quadrant\n",co1,co2);
    else if(co1<0 && co2<0)
        printf("The Coordinate Point (%d,%d) Lies in the Third Quadrant\n",co1,co2);
    else if(co1>0 && co2<0)
        printf("The Coordinate Point (%d,%d) Lies in the Fourth Quadrant\n",co1,co2);
    else
        printf("The Coordinate Point (%d,%d) Lies in the Origin\n",co1,co2);
    return 0;
}
\end{lstlisting}
\newpage

\subsection{Creating a Calculator Using If-else}
\vspace{0.5cm}
\begin{lstlisting}[caption={Creating a Calculator Using If-else}]
#include<stdio.h>
int main()
{
    char oper;
    float num1,num2;
    printf("Enter an Operator: +, -, *, /: ");
    scanf("%c",&oper);
    printf("Enter 1st Number: ");
    scanf("%lf",&num1);
    printf("Enter 2nd Number: ");
    scanf("%lf",&num2);
    if(op == '+')
        printf("Addition is: %lf + %lf = %lf\n",num1,num2,num1+num2);
    else if(op == '-')
        printf("Subtraction is: %lf - %lf = %lf\n",num1,num2,num1-num2);
    else if(op == '*')
        printf("Multiplication is: %lf * %lf = %lf\n",num1,num2,num1*num2);
    else if((op == '/') && (b!=0))
        printf("Division is: %lf / %lf = %lf\n",num1,num2,num1/num2);
    else
        printf("Invalid Input!!!\n");
    return 0;
}
\end{lstlisting}
\newpage

\subsection{Check Leap Year}
\vspace{0.5cm}
\begin{lstlisting}[caption={Check Leap Year}]
#include<stdio.h>
int main()
{
    int year;
    printf("Enter a Year: ");
    scanf("%d",&year);
    if(year%400==0)
        printf("%d is a Leap Year",year);
    else if((year%4==0) && (year%100!=0))
        printf("%d is a Leap Year",year);
    else
        printf("%d is not a Leap Year",year);
    return 0;
}
\end{lstlisting}
\newpage

\subsection{Check Leap Year Using Logical Operator}
\vspace{0.5cm}
\begin{lstlisting}[caption={Check Leap Year Using Logical Operator}]
#include<stdio.h>
int main()
{
    int year;
    printf("Enter a Year: ");
    scanf("%d",&year);
    if(((year%4==0) && (year%100!=0)) || (year%400==0))
        printf("%d is a Leap Year",year);
    else
        printf("%d is not a Leap Year",year);
    return 0;
}
\end{lstlisting}
\newpage

\subsection{Check Uppercase or Lowercase}
\vspace{0.5cm}
\begin{lstlisting}[caption={Check Uppercase or Lowercase}]
#include<stdio.h>
int main()
{
    char ch;
    printf("Enter a Character: ");
    scanf("%c",&ch);
    if((ch>='A') && (ch<='Z'))
        printf("%c is a Uppercase Letter",ch);
    else if((ch>='a') && (ch<='z'))
        printf("%c is a Lowercase Letter",ch);
    else
        printf("%c is not a Letter",ch);
    return 0;
}
\end{lstlisting}
\newpage

\subsection{Check Uppercase or Lowercase Using Function}
\vspace{0.5cm}
\begin{lstlisting}[caption={Check Uppercase or Lowercase Using Function}]
#include<stdio.h>
#include<ctype.h>
int main()
{
    char ch;
    printf("Enter Any Character: ");
    scanf("%c",&ch);
    if(isupper(ch))
        printf("'%c' is Uppercase Alphabet.",ch);
    else if(islower(ch))
        printf("'%c' is Lowercase Alphabet.",ch);
    else
        printf("'%c' is Not an Alphabet.",ch);
    return 0;
}
\end{lstlisting}
\newpage

\subsection{Check Triangle is Valid or Not if Angles Are Given}
\vspace{0.5cm}
\begin{lstlisting}[caption={Check Triangle is Valid or Not if Angles Are Given}]
#include<stdio.h>
int main()
{
    int angle1,angle2,angle3,sum;
    printf("Enter Three Angles of Triangle: ");
    scanf("%d %d %d",&angle1,&angle2,&angle3);
    sum=angle1+angle2+angle3;
    if((sum==180) && (angle1>0) && (angle2>0) && (angle3>0))
    {
        printf("Triangle is Valid.");
    }
    else
    {
        printf("Triangle is Not Valid.");
    }
    return 0;
}
\end{lstlisting}
\newpage

\subsection{Check Triangle is Valid or Not if Sides Are Given}
\vspace{0.5cm}
\begin{lstlisting}[caption={Check Triangle is Valid or Not if Sides Are Given}]
#include<stdio.h>
int main()
{
    int side1,side2,side3;
    printf("Enter Three Sides of Triangle: ");
    scanf("%d %d %d",&side1,&side2,&side3);
    if((side1+side2)>side3)
    {
        if((side2+side3)>side1)
        {
            if((side1+side3)>side2)
            {
                printf("Triangle is Valid.");
            }
            else
            {
                printf("Triangle is Not Valid.");
            }
        }
        else
        {
            printf("Triangle is Not Valid.");
        }
    }
    else
    {
        printf("Triangle is Not Valid.");
    }
    return 0;
}
\end{lstlisting}
\newpage

\subsection{Check Triangle is Valid or Not if Sides Are Given Using Logical Operator}
\vspace{0.5cm}
\begin{lstlisting}[caption={Check Triangle is Valid or Not if Sides Are Given Using Logical Operator}]
#include<stdio.h>
int main()
{
    int side1,side2,side3;
    printf("Enter Three Sides of Triangle: ");
    scanf("%d %d %d",&side1,&side2,&side3);
    if(((side1+side2)>side3) && ((side2+side3)>side1) && ((side3+side1)>side2))
    {
        printf("Triangle is valid.");
    }
    else
    {
        printf("Triangle is not valid.");
    }
    return 0;
}
\end{lstlisting}
\newpage

\subsection{Check a Triangle is Equilateral, Scalene or Isosceles}
\vspace{0.5cm}
\begin{lstlisting}[caption={Check a Triangle is Equilateral, Scalene or Isosceles}]
#include<stdio.h>
int main()
{
    int side1,side2,side3;
    printf("Enter Three Sides of Triangle: ");
    scanf("%d %d %d",&side1,&side2,&side3);
    if((side1==side2) && (side2==side3)) 
    {
        printf("Equilateral triangle.");
    }
    else if((side1==side2) || (side1==side3) || (side2==side3)) 
    {
        printf("Isosceles triangle.");
    }
    else 
    {
        printf("Scalene triangle.");
    }
    return 0;
}
\end{lstlisting}
\newpage

\subsection{Check a Triangle is Acute, Obtuse or Right-angled}
\vspace{0.5cm}
\begin{lstlisting}[caption={Check a Triangle is Acute, Obtuse or Right-angled}]
#include<stdio.h>
int main()
{
    int side1,side2,side3;
    printf("Enter Three Sides: ");
    scanf("%d %d %d",&side1,&side2,&side3);
    if(((side1+side2)>side3) && ((side2+side3)>side1) && ((side3+side1)>side2))
    {
        int a=(side1*side1),b=(side2*side2),c=(side3*side3);
        if(((a+b) == c) || ((b+c) == a) || ((a+c) == b))
            printf("Right-angled\n");
        else if(((a+b) > c) && ((b+c) > a) && ((a+c) > b))
            printf("Acute\n");
        else
            printf("Obtuse\n");
    }
    else
        printf("Not a triangle\n");
    return 0;
}
\end{lstlisting}
\newpage

\subsection{Find The Type of Quadrilateral Based on Sides & Angles}
\vspace{0.5cm}
\begin{lstlisting}[caption={Find The Type of Quadrilateral Based on Sides & Angles}]
#include<stdio.h>
int main()
{
    int side1,side2,side3,side4,angle;
    printf("Enter Four Sides: ");
    scanf("%d %d %d %d",&side1,&side2,&side3,&side4);
    printf("Enter Angle: ");
    scanf("%d",&angle);
    if((side1 == side2) && (side2 == side3) && (side3 == side4))
    {
        if(angle == 90)
            printf("Square\n");
        else
            printf("Rhombus\n");
    }
    else if((side1 == side3) && (side2 == side4))
    {
        if(angle == 90)
            printf("Rectangle\n");
        else
            printf("Parallelogram\n");
    }
    else
        printf("Quadrilateral\n");
    return 0;
}
\end{lstlisting}
\newpage

\subsection{Check a Date is Valid}
\vspace{0.5cm}
\begin{lstlisting}[caption={Check a Date is Valid}]
#include<stdio.h>
int main()
{
    int day,month,year;
    printf("Enter Date: ");
    scanf("%d",&day);
    printf("Enter Month: ");
    scanf("%d",&month);
    printf("Enter Year: ");
    scanf("%d",&year);
    if((month >= 1) && (month <= 12) && (year > 0))
    {
        if(month == 2)
        {
            if(((year%4 == 0) && (year%100 != 0)) || (year%400 == 0))
            {
                if((day >= 1) && (day <= 29))
                    printf("Valid Date\n");
                else
                    printf("Invalid Date\n");
            }
            else
            {
                if((day >= 1) && (day <= 28))
                    printf("Valid Date\n");
                else
                    printf("Invalid Date\n");
            }
        }
        else if((month == 4) || (month == 6) || (month == 9) || (month == 11))
        {
            if((day >= 1) && (day <= 30))
                printf("Valid Date\n");
            else
                printf("Invalid Date\n");
        }
        else
        {
            if((day >= 1) && (day <= 31))
                printf("Valid Date\n");
            else
                printf("Invalid Date\n");
        }
    }
    else
        printf("Invalid Date\n");
    return 0;
}
\end{lstlisting}
\newpage

\subsection{Check Alphabet, Digit or Special Character}
\vspace{0.5cm}
\begin{lstlisting}[caption={Check Alphabet, Digit or Special Character}]
#include<stdio.h>
int main()
{
    char ch;
    printf("Enter Any Character: ");
    scanf("%c",&ch);
    if((ch>='a' && ch<='z') || (ch>='A' && ch<='Z'))
    {
        printf("'%c' is Alphabet.",ch);
    }
    else if(ch>='0' && ch<='9')
    {
        printf("'%c' is Digit.",ch);
    }
    else
    {
        printf("'%c' is Special Character.",ch);
    }
    return 0;
}
\end{lstlisting}
\newpage

\subsection{Check Alphabet, Digit or Special Character Using ASCII}
\vspace{0.5cm}
\begin{lstlisting}[caption={Check Alphabet, Digit or Special Character Using ASCII}]
#include<stdio.h>
int main()
{
    char ch;
    printf("Enter Any Character: ");
    scanf("%c",&ch);
    if((ch>=97 && ch<=122) || (ch>=65 && ch<=90))
    {
        printf("'%c' is Alphabet.",ch);
    }
    else if(ch>=48 && ch<=57)
    {
        printf("'%c' is Digit.",ch);
    }
    else
    {
        printf("'%c' is Special Character.",ch);
    }
    return 0;
}
\end{lstlisting}
\newpage

\subsection{Print Day Name of Week}
\vspace{0.5cm}
\begin{lstlisting}[caption={Print Day Name of Week}]
#include<stdio.h>
int main()
{
    int week;
    printf("Enter Week Number (1-7): ");
    scanf("%d",&week);
    if(week == 1)
        printf("Monday");
    else if(week == 2)
        printf("Tuesday");
    else if(week == 3)
        printf("Wednesday");
    else if(week == 4)
        printf("Thursday");
    else if(week == 5)
        printf("Friday");
    else if(week == 6)
        printf("Saturday");
    else if(week == 7)
        printf("Sunday");
    else
        printf("Invalid Input!!!");
    return 0;
}
\end{lstlisting}
\newpage

\subsection{Print Days in a Month}
\vspace{0.5cm}
\begin{lstlisting}[caption={Print Days in a Month}]
#include<stdio.h>
int main()
{
    int month;
    printf("Enter Month Number (1-12): ");
    scanf("%d",&month);
    if(month==1 || month==3 || month==5 || month==7 
       || month==8 || month==10 || month==12)
    {
        printf("31 days");
    }
    else if(month==4 || month==6 || month==9 || month==11)
    {
        printf("30 days");
    }
    else if(month==2)
    {
        printf("28 or 29 days");
    }
    else
    {
        printf("Invalid Input!!!");
    }
    return 0;
}
\end{lstlisting}
\newpage

\subsection{Find Roots of Quadratic Equation Using If-else}
\vspace{0.5cm}
\begin{lstlisting}[caption={Find Roots of Quadratic Equation Using If-else}]
#include<stdio.h>
#include<math.h>
int main()
{
    float a,b,c,Dis,X1,X2,Ima;
    printf("Enter Values of a, b, c: ");
    scanf("%f %f %f",&a,&b,&c);
    Dis=(b*b)-(4*a*c);
    if(Dis>0)
    {
        X1=(-b+sqrt(Dis))/(2*a);
        X2=(-b-sqrt(Dis))/(2*a);
        printf("Two Distinct & Real Roots Exists: %.2f and %.2f",X1,X2);
    }
    else if(Dis==0)
    {
        X1=X2=-b/(2*a);
        printf("Two Equal & Real Roots Exists: %.2f and %.2f",X1,X2);
    }
    else if(Dis<0)
    {
        X1=X2=-b/(2*a);
        Ima=sqrt(-Dis)/(2*a);
        printf("Two Distinct Complex Roots Exists: %.2f + %.2fi and %.2f - %.2fi",X1,Ima,X2,Ima);
    }
    return 0;
}
\end{lstlisting}
\newpage

\subsection{Calculate Profit or Loss}
\vspace{0.5cm}
\begin{lstlisting}[caption={Calculate Profit or Loss}]
#include<stdio.h>
int main()
 
{
    int CP,SP,Amount;
    printf("Enter Cost Price: ");
    scanf("%d",&CP);
    printf("Enter Selling Price: ");
    scanf("%d",&SP);
    if(SP>CP)
    {
        Amount=SP-CP;
        printf("\nProfit = %d",Amount);
    }
    else if(CP>SP)
    {
        Amount=CP-SP;
        printf("\nLoss = %d",Amount);
    }
    else
    {
        printf("\nNo Profit & Loss.");
    }
    return 0;
}
\end{lstlisting}
\newpage

\subsection{Count Total Number of Notes in Amount}
\vspace{0.5cm}
\begin{lstlisting}[caption={Count Total Number of Notes in Amount}]
#include<stdio.h>
int main()
{
    int amount;
    int note500,note100,note50,note20,note10,note5,note2,note1;
    note500=note100=note50=note20=note10=note5=note2=note1=0;
    printf("Enter Amount: ");
    scanf("%d",&amount);
 
    printf("\nTotal Number of Notes:\n");
    printf("``````````````````````\n");
    if(amount>=500){
        note500 = amount/500;
        amount -= (note500*500);
        printf("Note of 500: %d\n",note500);
    }
    if(amount>=100){
        note100 = amount/100;
        amount -= (note100*100);
        printf("Note of 100: %d\n",note100);
    }
    if(amount>=50){
        note50 = amount/50;
        amount -= (note50*50);
        printf("Note of 50: %d\n",note50);
    }
    if(amount>=20){
        note20 = amount/20;
        amount -= (note20*20);
        printf("Note of 20: %d\n",note20);
    }
    if(amount>=10){
        note10 = amount/10;
        amount -= (note10*10);
        printf("Note of 10: %d\n",note10);
    }
    if(amount>=5){
        note5 = amount/5;
        amount -= (note5*5);
        printf("Note of 5: %d\n",note5);
    }
    if(amount>=2){
        note2 = amount/2;
        amount -= (note2*2);
        printf("Note of 2: %d\n",note2);
    }
    if(amount>=1){
        note1 = amount;
        printf("Note of 1: %d\n",note1);
    }
    return 0;
}
\end{lstlisting}
\newpage

\subsection{Grading System Using Logical Operator}
\vspace{0.5cm}
\begin{lstlisting}[caption={Grading System Using Logical Operator}]
#include<stdio.h>
int main()
{
    int mark;
    printf("Enter Mark: ");
    scanf("%d",&mark);
    if(mark>100 || mark<0)
        printf("Invalid Mark!!!");
    else if(mark>=80 && mark<=100)
        printf("Your Grade is A+ & Number: %d",mark);
    else if(mark>=70 && mark<=79)
        printf("Your Grade is A & Number: %d",mark);
    else if(mark>=60 && mark<=69)
        printf("Your Grade is A- & Number: %d",mark);
    else if(mark>=50 && mark<=59)
        printf("Your Grade is B & Number: %d",mark);
    else if(mark>=40 && mark<=49)
        printf("Your Grade is C & Number: %d",mark);
    else if(mark>=33 && mark<=39)
        printf("Your Grade is D & Number: %d",mark);
    else
        printf("Your Grade is F & Number: %d",mark);
    return 0;
}
\end{lstlisting}
\newpage


\section{Switch Programming Exercises}


\subsection{Print Digit From 0-9 Using Switch}
\vspace{0.5cm}
\begin{lstlisting}[caption={Print Digit From 0-9 Using Switch}]
#include<stdio.h>
int main()
{
    int digit;
    printf("Enter a Digit: ");
    scanf("%d",&digit);
    switch(digit)
    {
    case 0:
        printf("Number Is Zero");
        break;
    case 1:
        printf("Number Is One");
        break;
    case 2:
        printf("Number Is Two");
        break;
    case 3:
        printf("Number Is Three");
        break;
 
   
    case 4:
        printf("Number Is Four");
        break;
    case 5:
        printf("Number Is Five");
        break;
    case 6:
        printf("Number Is Six");
        break;
    case 7:
        printf("Number Is Seven");
        break;
    case 8:
        printf("Number Is Eight");
        break;
    case 9:
        printf("Number Is Nine");
        break;
    default:
        printf("Number Is Not Valid");
    }
    return 0;
}
\end{lstlisting}
\newpage

\subsection{Vowel Consonant Using Switch}
\vspace{0.5cm}
\begin{lstlisting}[caption={Vowel Consonant Using Switch}]
#include<stdio.h>
int main()
{
    char ch;
    printf("Enter a Character: ");
    scanf("%c",&ch);
    switch(ch)
    {
    case 'a':
        printf("This is a Vowel");
        break;
    case 'e':
        printf("This is a Vowel");
        break;
    case 'i':
        printf("This is a Vowel");
        break;
    case 'o':
        printf("This is a Vowel");
        break;
    case 'u':
        printf("This is a Vowel");
        break;
    case 'A':
        printf("This is a Vowel");
        break;
    case 'E':
        printf("This is a Vowel");
        break;
    case 'I':
        printf("This is a Vowel");
        break;
    case 'O':
        printf("This is a Vowel");
        break;
    case 'U':
        printf("This is a Vowel");
        break;
    default:
        printf("This is a Consonant");
    }
    return 0;
}
\end{lstlisting}
\newpage

\subsection{Vowel Consonant Using Switch}
\vspace{0.5cm}
\begin{lstlisting}[caption={Vowel Consonant Using Switch}]
#include<stdio.h>
int main()
{
    char ch;
    printf("Enter a Character: ");
    scanf("%c",&ch);
    switch(ch)
    {
    case 'a':
    case 'e':
    case 'i':
    case 'o':
    case 'u':
    case 'A':
    case 'E':
    case 'I':
    case 'O':
    case 'U':
        printf("This is a Vowel");
        break;
    default:
        printf("This is a Consonant");
    }
    return 0;
}
\end{lstlisting}
\newpage

\subsection{Menu Based Temperature Conversion Using Switch}
\vspace{0.5cm}
\begin{lstlisting}[caption={Menu Based Temperature Conversion Using Switch}]
#include<stdio.h>
int main()
{
    int choice;
    float Temp,ConvertedTemp;
    printf("Temperature Conversion Menu\n");
    printf("1. Fahrenheit to Celsius\n");
    printf("2. Celsius to Fahrenheit\n");
    printf("Enter Your Choice: ");
    scanf("%d",&choice);
    switch(choice)
    {
    case 1:
        {
            printf("Enter the Fahrenheit Temperature: ");
            scanf("%f",&Temp);
            ConvertedTemp=((Temp-32)*5)/9;
            printf("The Temperature in Celsius: %f\n",ConvertedTemp);
            break;
        }
    case 2:
        {
            printf("Enter the Celsius Temperature: ");
            scanf("%f",&Temp);
            ConvertedTemp=(9*Temp/5)+32;
            printf("The Temperature in Fahrenheit: %f\n",ConvertedTemp);
            break;
        }
    default:
        printf("Invalid Choice");
    }
    return 0;
}
\end{lstlisting}
\newpage

\subsection{Creating a Calculator Using Switch}
\vspace{0.5cm}
\begin{lstlisting}[caption={Creating a Calculator Using Switch}]
#include<stdio.h>
int main()
{
    double num1,num2;
    char oper;
    printf("Enter an Operator: +, -, *, /: ");
    scanf("%c",&oper);
    printf("Enter 1st Number: ");
    scanf("%lf",&num1);
    printf("Enter 2nd Number: ");
    scanf("%lf",&num2);
    switch(oper)
    {
    case '+':
        {
            printf("Addition is: %lf + %lf = %lf\n",num1,num2,num1+num2);
            break;
        }
    case '-':
        {
            printf("Subtraction is: %lf - %lf = %lf\n",num1,num2,num1-num2);
            break;
        }
    case '*':
        {
            printf("Multiplication is: %lf * %lf = %lf\n",num1,num2,num1*num2);
            break;
        }
    case '/':
        {
            printf("Division is: %lf / %lf = %lf\n",num1,num2,num1/num2);
            break;
        }
    default:
        printf("Not a Valid Character\n");
    }
    return 0;
}
\end{lstlisting}
\newpage

\subsection{Creating a Calculator Using Switch}
\vspace{0.5cm}
\begin{lstlisting}[caption={Creating a Calculator Using Switch}]
#include <stdio.h>
int main()
{
    double num1,num2,result;
    char oper;
    printf("Enter an Operator: +, -, *, /: ");
    scanf("%c",&oper);
    printf("Enter 1st Number: ");
    scanf("%lf",&num1);
    printf("Enter 2nd Number: ");
    scanf("%lf",&num2);
    switch(oper)
    {
        case '+':
            result=num1+num2;
            printf("Addition is: %lf + %lf = %lf\n",num1,num2,result);
            break;
        case '-':
            result=num1-num2;
            printf("Subtraction is: %lf + %lf = %lf\n",num1,num2,result);
            break;
        case '*':
            result=num1*num2;
            printf("Multiplication is: %lf + %lf = %lf\n",num1,num2,result);
            break;
        case '/':
            if(num2!=0)
            {
                result=num1/num2;
                printf("Division is: %lf + %lf = %lf\n",num1,num2,result);
            }
            else
            {
                printf("Error!!!\n");
            }
            break;
        default:
            printf("Invalid operator!!!\n");
    }
    return 0;
}
\end{lstlisting}
\newpage

\subsection{Check Even or Odd Using Switch}
\vspace{0.5cm}
\begin{lstlisting}[caption={Check Even or Odd Using Switch}]
#include<stdio.h>
int main()
{
    int num;
    printf("Enter Any Number: ");
    scanf("%d",&num);
    switch(num%2)
    {
        case 0:
            printf("%d is Even",num);
            break;
        case 1:
            printf("%d is Odd",num);
            break;
    }
    return 0;
}
\end{lstlisting}
\newpage

\subsection{Find Roots of Quadratic Equation Using Switch}
\vspace{0.5cm}
\begin{lstlisting}[caption={Find Roots of Quadratic Equation Using Switch}]
#include<stdio.h>
#include<math.h>
int main()
{
    float a,b,c,Dis,X1,X2,Ima;
    printf("Enter Values of a, b, c: ");
    scanf("%f %f %f",&a,&b,&c);
    Dis=(b*b)-(4*a*c);
    switch(Dis>0)
    {
        case 1:
            X1=(-b+sqrt(Dis))/(2*a);
            X2=(-b-sqrt(Dis))/(2*a);
            printf("Two Distinct & Real Roots Exists: %.2f and %.2f",X1,X2);
            break;
        case 0:
            switch(Dis<0)
            {
                case 1:
                    X1=X2=-b/(2*a);
                    Ima=sqrt(-Dis)/(2*a);
                    printf("Two Distinct Complex Roots Exists: %.2f + %.2fi and %.2f - %.2fi",X1,Ima,X2,Ima);
                    break;
                case 0:
                    X1=X2=-b/(2*a);
                    printf("Two Equal & Real Roots Exists: %.2f and %.2f",X1,X2);
                    break;
            }
    }
    return 0;
}
\end{lstlisting}
\newpage

\subsection{Check Positive, Negative or Zero Using Switch}
\vspace{0.5cm}
\begin{lstlisting}[caption={Check Positive, Negative or Zero Using Switch}]
#include<stdio.h>
int main()
{
    int num;
    printf("Enter Any Number: ");
    scanf("%d",&num);
    switch(num>0)
 
    {
        case 1:
            printf("%d is Positive.",num);
        break;
        case 0:
            switch(num<0)
            {
                case 1:
                    printf("%d is Negative.",num);
                    break;
                case 0:
                    printf("%d is Zero.",num);
                    break;
            }
        break;
    }
    return 0;
}
\end{lstlisting}
\newpage

\subsection{Print Number of Days in a Month Using Switch}
\vspace{0.5cm}
\begin{lstlisting}[caption={Print Number of Days in a Month Using Switch}]
#include<stdio.h>
int main()
{
    int month;
    printf("Enter Month: ");
    scanf("%d",&month);
    switch(month)
    {
        case 1:
        case 3:
        case 5:
        case 7:
        case 8:
        case 10:
        case 12:
            printf("31 days");
            break;
        case 4:
        case 6:
        case 9:
        case 11:
            printf("30 days");
            break;
        case 2:
            printf("28/29 days");
            break;
        default:
            printf("Invalid Input!!!");
    }
    return 0;
}
\end{lstlisting}
\newpage

\subsection{Grading System Using Switch}
\vspace{0.5cm}
\begin{lstlisting}[caption={Grading System Using Switch}]
#include<stdio.h>
int main()
{
    int marks;
    printf("Enter Your Marks: ");
    scanf("%d",&marks);
    if(marks>100 || marks<0)
        printf("Invalid Marks!!!");
    else
    {
        switch(marks/10)
        {
        case 10:
        case 9:
            printf("\nYour Grade is: A");
            break;
        case 8:
            printf("\nYour Grade is: B" );
            break;
        case 7:
            printf("\nYour Grade is: C" );
            break;
        case 6:
            printf("\nYour Grade is: D" );
            break;
        case 5:
            printf("\nYour Grade is: E" );
            break;
        case 4:
            printf("\nYour Grade is: E-");
            break;
        default:
            printf("\nYour Grade is: F");
        }
    }
}
\end{lstlisting}
\newpage

\subsection{Largest of Two Numbers Using Switch}
\vspace{0.5cm}
\begin{lstlisting}[caption={Largest of Two Numbers Using Switch}]
#include<stdio.h>
int main()
{
    int num1,num2;
    printf("Enter Two Numbers: ");
    scanf("%d %d",&num1,&num2);
    switch(num1>num2)
    {
        case 0:
            printf("%d is Largest",num2);
            break;
        case 1:
            printf("%d is Largest",num1);
            break;
    }
    return 0;
}
\end{lstlisting}
\newpage

\subsection{Smallest of Two Numbers Using Switch}
\vspace{0.5cm}
\begin{lstlisting}[caption={Smallest of Two Numbers Using Switch}]
#include<stdio.h>
int main()
{
    int num1,num2;
    printf("Enter Two Numbers: ");
    scanf("%d %d",&num1,&num2);
    switch(num1<num2)
    {
        case 0:
            printf("%d is Smallest",num2);
            break;
        case 1:
            printf("%d is Smallest",num1);
            break;
    }
    return 0;
}
\end{lstlisting}
\newpage


\section{Conditional Operator Programming Exercises}


\subsection{Largest of Two Numbers Using Conditional Operator}
\vspace{0.5cm}
\begin{lstlisting}[caption={Largest of Two Numbers Using Conditional Operator}]
#include<stdio.h>
int main()
{
    int num1,num2,max;
    printf("Enter 1st Number: ");
    scanf("%d",&num1);
    printf("Enter 2nd Number: ");
    scanf("%d",&num2);
    max = (num1>num2)?num1:num2;
    printf("Largest Between %d & %d is %d\n",num1,num2,max);
    return 0;
}
\end{lstlisting}
\newpage

\subsection{Smallest of Two Numbers Using Conditional Operator}
\vspace{0.5cm}
\begin{lstlisting}[caption={Smallest of Two Numbers Using Conditional Operator}]
#include<stdio.h>
int main()
{
    int num1,num2,min;
    printf("Enter 1st Number: ");
    scanf("%d",&num1);
    printf("Enter 2nd Number: ");
    scanf("%d",&num2);
    min = (num1<num2)?num1:num2;
    printf("Smallest Between %d & %d is %d\n",num1,num2,min);
    return 0;
}
\end{lstlisting}
\newpage

\subsection{Largest of Three Numbers Using Conditional Operator}
\vspace{0.5cm}
\begin{lstlisting}[caption={Largest of Three Numbers Using Conditional Operator}]
#include<stdio.h>
int main()
{
    int num1,num2,num3,max;
    printf("Enter three numbers: ");
    scanf("%d %d %d",&num1,&num2,&num3);
    max = ((num1>num2) && (num1>num3)) ? num1 : ((num2>num3) ? num2 : num3);
    printf("\nLargest Between %d, %d & %d is %d",num1,num2,num3,max);
    return 0;
}
\end{lstlisting}
\newpage

\subsection{Smallest of Three Numbers Using Conditional Operator}
\vspace{0.5cm}
\begin{lstlisting}[caption={Smallest of Three Numbers Using Conditional Operator}]
#include<stdio.h>
int main()
{
    int num1,num2,num3,min;
    printf("Enter three numbers: ");
    scanf("%d %d %d",&num1,&num2,&num3);
    min = ((num1<num2) && (num1<num3)) ? num1 : ((num2<num3) ? num2 : num3);
    printf("\nSmallest Between %d, %d & %d is %d",num1,num2,num3,min);
    return 0;
}
\end{lstlisting}
\newpage

\subsection{Even or Odd Using Conditional Operator}
\vspace{0.5cm}
\begin{lstlisting}[caption={Even or Odd Using Conditional Operator}]
#include<stdio.h>
int main()
{
    int num;
    printf("Enter Any Number: ");
    scanf("%d",&num);
    (num%2==0) ? printf("%d is Even",num) : printf("%d is Odd",num);
    return 0;
}
\end{lstlisting}
\newpage

\subsection{Leap Year Using Conditional Operator}
\vspace{0.5cm}
\begin{lstlisting}[caption={Leap Year Using Conditional Operator}]
#include<stdio.h>
int main()
{
    int year;
    printf("Enter Any Year: ");
    scanf("%d",&year);
    ((year%4==0) && (year%100!=0)) ? printf("%d is a Leap Year",year) : 
        ((year%400==0) ? printf("%d is a Leap Year",year) : 
            printf("%d is a Not Leap Year",year));
    return 0;
}
\end{lstlisting}
\newpage

\subsection{Leap Year Using Conditional Operator}
\vspace{0.5cm}
\begin{lstlisting}[caption={Leap Year Using Conditional Operator}]
#include<stdio.h>
int main()
{
    int year;
    printf("Enter Any Year: ");
    scanf("%d",&year);
    printf("%s",(((year%4==0) && (year%100!=0)) ? "Leap Year" : 
                 ((year%400==0) ? "Leap Year" : "Not Leap Year")));
    return 0;
}
\end{lstlisting}
\newpage

\subsection{Alphabet Using Conditional Operator}
\vspace{0.5cm}
\begin{lstlisting}[caption={Alphabet Using Conditional Operator}]
#include<stdio.h>
int main()
{
    char ch;
    printf("Enter Any Character: ");
    scanf("%c",&ch);
    ((ch>='a' && ch<='z') || (ch>='A' && ch<='Z')) ? 
        printf("%c is a Alphabet",ch) : printf("%c is Not a Alphabet",ch);
    return 0;
}
\end{lstlisting}
\newpage

\subsection{Positive, Negative or Zero Using Conditional Operator}
\vspace{0.5cm}
\begin{lstlisting}[caption={Positive, Negative or Zero Using Conditional Operator}]
#include<stdio.h>
int main()
{
    int num;
    printf("Enter Any Number: ");
    scanf("%d",&num);
    (num>0) ? printf("The Number is Positive\n") : (num<0) ? 
        printf("The Number is Negative\n") : printf("The Number is Zero\n");
    return 0;
}
\end{lstlisting}
\newpage

\subsection{Vowel Consonant Using Conditional Operator}
\vspace{0.5cm}
\begin{lstlisting}[caption={Vowel Consonant Using Conditional Operator}]
#include<stdio.h>
int main()
{
    char ch;
    printf("Enter Any Character: ");
    scanf(" %c",&ch);
    (ch == 'a' || ch == 'e' || ch == 'i' || ch == 'o' || ch == 'u' ||
     ch == 'A' || ch == 'E' || ch == 'I' || ch == 'O' || ch == 'U') ? 
            printf("It is a Vowel") : printf("It is a consonant");
    return 0;
}
\end{lstlisting}
\newpage

\subsection{Coordinate Quadrant Identification Using Conditional Operator}
\vspace{0.5cm}
\begin{lstlisting}[caption={Coordinate Quadrant Identification Using Conditional Operator}]
#include<stdio.h>
int main()
{
    int x,y;
    printf("Input the Values for X & Y Coordinate: ");
    scanf("%d %d",&x,&y);
    ((x==0) || (y==0)) ? printf("Lies in the Origin") : (x>0) ? 
            ((y>0) ? printf("First quadrant") : printf("Fourth quadrant")) : 
                    ((y>0) ? printf("Second quadrant") : printf("Third quadrant"));
    return 0;
}
\end{lstlisting}
\newpage

\subsection{Triangle is Valid or Not if Sides Are Given Using Conditional Operator}
\vspace{0.5cm}
\begin{lstlisting}[caption={Triangle is Valid or Not if Sides Are Given Using Conditional Operator}]
#include<stdio.h>
int main()
{
    int side1,side2,side3;
    printf("Enter Three Sides: ");
    scanf("%d %d %d",&side1,&side2,&side3);
    (((side1+side2)>side3) && ((side2+side3)>side1) && ((side1+side3)>side2)) ? 
            printf("The Triangle is Valid") : printf("The Triangle is Not Valid");
    return 0;
}
\end{lstlisting}
\newpage


\section{Bitwise Operator Programming Exercises}


\subsectionConvert Decimal to Binary Using Bitwise Operator}
\vspace{0.5cm}
\begin{lstlisting}[caption={Convert Decimal to Binary Using Bitwise Operator}]
#include<stdio.h>
int main()
{
    int number;
    printf("Enter any Octal Number: ");
    scanf("%d",&number);
    printf("Decimal number = %d",number);
    return 0;
}
\end{lstlisting}
\newpage

\subsection{Octal to Decimal}
\vspace{0.5cm}
\begin{lstlisting}[caption={Octal to Decimal}]
#include<stdio.h>
int main()
{
    int number;
    printf("Enter any Octal Number: ");
    scanf("%d",&number);
    printf("Decimal number = %d",number);
    return 0;
}
\end{lstlisting}
\newpage

\subsection{Octal to Decimal}
\vspace{0.5cm}
\begin{lstlisting}[caption={Octal to Decimal}]
#include<stdio.h>
int main()
{
    int number;
    printf("Enter any Octal Number: ");
    scanf("%d",&number);
    printf("Decimal number = %d",number);
    return 0;
}
\end{lstlisting}
\newpage

\subsection{Octal to Decimal}
\vspace{0.5cm}
\begin{lstlisting}[caption={Octal to Decimal}]
#include<stdio.h>
int main()
{
    int number;
    printf("Enter any Octal Number: ");
    scanf("%d",&number);
    printf("Decimal number = %d",number);
    return 0;
}
\end{lstlisting}
\newpage

\end{document}