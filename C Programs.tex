% Setting up the document class and basic structure
\documentclass[a4paper,12pt]{article}
\usepackage[utf8]{inputenc}
\usepackage[T1]{fontenc}
\usepackage{lmodern}
\usepackage{geometry}
\geometry{margin=1in}
\usepackage{listings}
\usepackage{xcolor}
\usepackage{tocloft}
\usepackage{amsmath}
\usepackage{parskip}
\usepackage{fancyhdr}
\usepackage{lastpage}
\usepackage{enumitem}
\usepackage{hyperref}

% Configuring the listings package for C code
\lstset{
    language=C,
    basicstyle=\ttfamily\small,
    keywordstyle=\color{blue}\bfseries,
    stringstyle=\color{red},
    commentstyle=\color{green!60!black}\itshape,
    numbers=none, % Disabling line numbers for all listings
    showspaces=false,
    showstringspaces=false,
    frame=single,
    breaklines=true,
    breakatwhitespace=true,
    tabsize=4,
    captionpos=b
}

% Setting up fonts (Latin Modern as a reliable choice)
\usepackage{lmodern}

% Configuring page headers and footers
\pagestyle{fancy}
\fancyhf{}
\fancyhead[L]{C Programming Exercises}
\fancyhead[R]{\thepage\ of \pageref{LastPage}}
\fancyfoot[C]{}

% Title and author
\title{C Programming Exercises}
\author{Compiled by Mejbah}
\date{May 7, 2025}

% Configuring hyperref to hide link borders
\hypersetup{
    hidelinks
}

\begin{document}

% Generating the title page
\maketitle

% Adding a table of contents
\tableofcontents
\newpage

% Introduction section
\section{Introduction}
This document provides C programs for 18 Basic Programming Exercises, 16 Bitwise Operators Exercises, 5 Conditional Operators Exercises, 21 If-else Exercises, 8 Switch Case Exercises, 50 Loop Exercises, 35 Star Pattern Exercises, 59 Number Pattern Exercises, 24 Function And Recursion Exercises, 18 Matrix Exercises, 38 String Exercises, 18 Pointer Exercises, 22 File Handling Exercises, 9 Macro And Pre-processor Directive Exercises. Each program includes the source code and a brief description, and is presented on a separate page for clarity. Users can compile and run these programs using a C compiler (e.g., GCC) to observe the output. The programs are presented in the order listed in the exercise set.

\newpage

% Basic Programming Exercises
\section{Basic Programming Exercises}

% Program 1: Input/Output of All Basic Data Types
\subsection{Input/Output of All Basic Data Types}
This program demonstrates how to input and output all basic data types in C, including int, float, double, char, and long.

\begin{lstlisting}[caption={Input/Output of All Basic Data Types}]
#include <stdio.h>
int main() {
    int integer;
    float float_num;
    double double_num;
    char character;
    long long_num;
    printf("Enter an integer: ");
    scanf("%d", &integer);
    printf("Enter a float: ");
    scanf("%f", &float_num);
    printf("Enter a double: ");
    scanf("%lf", &double_num);
    printf("Enter a character: ");
    scanf(" %c", &character);
    printf("Enter a long integer: ");
    scanf("%ld", &long_num);
    printf("Integer: %d\n", integer);
    printf("Float: %f\n", float_num);
    printf("Double: %lf\n", double_num);
    printf("Character: %c\n", character);
    printf("Long: %ld\n", long_num);
    return 0;
}
\end{lstlisting}

\newpage

% Program 2: Sum of Two Numbers
\subsection{Sum of Two Numbers}
This program takes two numbers as input and calculates their sum.

\begin{lstlisting}[caption={Sum of Two Numbers}]
#include <stdio.h>
int main() {
    float num1, num2, sum;
    printf("Enter first number: ");
    scanf("%f", &num1);
    printf("Enter second number: ");
    scanf("%f", &num2);
    sum = num1 + num2;
    printf("Sum: %.2f\n", sum);
    return 0;
}
\end{lstlisting}

\newpage

% Program 3: All Arithmetic Operations
\subsection{Perform All Arithmetic Operations}
This program takes two numbers and performs addition, subtraction, multiplication, division, and modulus operations.

\begin{lstlisting}[caption={Perform All Arithmetic Operations}]
#include <stdio.h>
int main() {
    float num1, num2;
    printf("Enter two numbers: ");
    scanf("%f %f", &num1, &num2);
    printf("Addition: %.2f\n", num1 + num2);
    printf("Subtraction: %.2f\n", num1 - num2);
    printf("Multiplication: %.2f\n", num1 * num2);
    if (num2 != 0)
        printf("Division: %.2f\n", num1 / num2);
    else
        printf("Division: Undefined (division by zero)\n");
    if ((int)num2 != 0)
        printf("Modulus: %d\n", (int)num1 % (int)num2);
    else
        printf("Modulus: Undefined (division by zero)\n");
    return 0;
}
\end{lstlisting}

\newpage

% Program 4: Perimeter of Rectangle
\subsection{Perimeter of Rectangle}
This program calculates the perimeter of a rectangle given its length and breadth.

\begin{lstlisting}[caption={Perimeter of Rectangle}]
#include <stdio.h>
int main() {
    float length, breadth, perimeter;
    printf("Enter length and breadth of rectangle: ");
    scanf("%f %f", &length, &breadth);
    perimeter = 2 * (length + breadth);
    printf("Perimeter: %.2f\n", perimeter);
    return 0;
}
\end{lstlisting}

\newpage

% Program 5: Area of Rectangle
\subsection{Area of Rectangle}
This program calculates the area of a rectangle given its length and breadth.

\begin{lstlisting}[caption={Area of Rectangle}]
#include <stdio.h>
int main() {
    float length, breadth, area;
    printf("Enter length and breadth of rectangle: ");
    scanf("%f %f", &length, &breadth);
    area = length * breadth;
    printf("Area: %.2f\n", area);
    return 0;
}
\end{lstlisting}

\newpage

% Program 6: Circle Calculations
\subsection{Diameter, Circumference, and Area of Circle}
This program calculates the diameter, circumference, and area of a circle given its radius.

\begin{lstlisting}[caption={Diameter, Circumference, and Area of Circle}]
#include <stdio.h>
#define PI 3.14159
int main() {
    float radius, diameter, circumference, area;
    printf("Enter radius of circle: ");
    scanf("%f", &radius);
    diameter = 2 * radius;
    circumference = 2 * PI * radius;
    area = PI * radius * radius;
    printf("Diameter: %.2f\n", diameter);
    printf("Circumference: %.2f\n", circumference);
    printf("Area: %.2f\n", area);
    return 0;
}
\end{lstlisting}

\newpage

% Program 7: Length Conversion
\subsection{Convert Centimeter to Meter and Kilometer}
This program converts a length from centimeters to meters and kilometers.

\begin{lstlisting}[caption={Convert Centimeter to Meter and Kilometer}]
#include <stdio.h>
int main() {
    float cm, meter, km;
    printf("Enter length in centimeters: ");
    scanf("%f", &cm);
    meter = cm / 100;
    km = cm / 100000;
    printf("Meters: %.2f\n", meter);
    printf("Kilometers: %.5f\n", km);
    return 0;
}
\end{lstlisting}

\newpage

% Program 8: Celsius to Fahrenheit
\subsection{Convert Celsius to Fahrenheit}
This program converts temperature from Celsius to Fahrenheit.

\begin{lstlisting}[caption={Convert Celsius to Fahrenheit}]
#include <stdio.h>
int main() {
    float celsius, fahrenheit;
    printf("Enter temperature in Celsius: ");
    scanf("%f", &celsius);
    fahrenheit = (celsius * 9 / 5) + 32;
    printf("Fahrenheit: %.2f\n", fahrenheit);
    return 0;
}
\end{lstlisting}

\newpage

% Program 9: Fahrenheit to Celsius
\subsection{Convert Fahrenheit to Celsius}
This program converts temperature from Fahrenheit to Celsius.

\begin{lstlisting}[caption={Convert Fahrenheit to Celsius}]
#include <stdio.h>
int main() {
    float fahrenheit, celsius;
    printf("Enter temperature in Fahrenheit: ");
    scanf("%f", &fahrenheit);
    celsius = (fahrenheit - 32) * 5 / 9;
    printf("Celsius: %.2f\n", celsius);
    return 0;
}
\end{lstlisting}

\newpage

% Program 10: Days to Years, Weeks, Days
\subsection{Convert Days to Years, Weeks, and Days}
This program converts a number of days into years, weeks, and remaining days.

\begin{lstlisting}[caption={Convert Days to Years, Weeks, and Days}]
#include <stdio.h>
int main() {
    int days, years, weeks, remaining_days;
    printf("Enter number of days: ");
    scanf("%d", &days);
    years = days / 365;
    weeks = (days % 365) / 7;
    remaining_days = (days % 365) % 7;
    printf("Years: %d\n", years);
    printf("Weeks: %d\n", weeks);
    printf("Days: %d\n", remaining_days);
    return 0;
}
\end{lstlisting}

\newpage

% Program 11: Power of a Number
\subsection{Power of a Number}
This program calculates x raised to the power y using the pow function.

\begin{lstlisting}[caption={Power of a Number}]
#include <stdio.h>
#include <math.h>
int main() {
    double x, y, result;
    printf("Enter base (x) and exponent (y): ");
    scanf("%lf %lf", &x, &y);
    result = pow(x, y);
    printf("%.2lf ^ %.2lf = %.2lf\n", x, y, result);
    return 0;
}
\end{lstlisting}

\newpage

% Program 12: Square Root
\subsection{Calculate Square Root}
This program calculates the square root of a number.

\begin{lstlisting}[caption={Calculate Square Root}]
#include <stdio.h>
#include <math.h>
int main() {
    double num, result;
    printf("Enter a number: ");
    scanf("%lf", &num);
    if (num >= 0) {
        result = sqrt(num);
        printf("Square root: %.2lf\n", result);
    } else {
        printf("Invalid input (negative number)\n");
    }
    return 0;
}
\end{lstlisting}

\newpage

% Program 13: Third Angle of Triangle
\subsection{Find Third Angle of Triangle}
This program calculates the third angle of a triangle given two angles.

\begin{lstlisting}[caption={Find Third Angle of Triangle}]
#include <stdio.h>
int main() {
    float angle1, angle2, angle3;
    printf("Enter two angles of triangle (in degrees): ");
    scanf("%f %f", &angle1, &angle2);
    angle3 = 180 - (angle1 + angle2);
    if (angle3 > 0)
        printf("Third angle: %.2f degrees\n", angle3);
    else
        printf("Invalid angles\n");
    return 0;
}
\end{lstlisting}

\newpage

% Program 14: Area of Triangle
\subsection{Area of Triangle}
This program calculates the area of a triangle given its base and height.

\begin{lstlisting}[caption={Area of Triangle}]
#include <stdio.h>
int main() {
    float base, height, area;
    printf("Enter base and height of triangle: ");
    scanf("%f %f", &base, &height);
    area = 0.5 * base * height;
    printf("Area: %.2f\n", area);
    return 0;
}
\end{lstlisting}

\newpage

% Program 15: Area of Equilateral Triangle
\subsection{Area of Equilateral Triangle}
This program calculates the area of an equilateral triangle given its side length.

\begin{lstlisting}[caption={Area of Equilateral Triangle}]
#include <stdio.h>
#include <math.h>
int main() {
    float side, area;
    printf("Enter side length of equilateral triangle: ");
    scanf("%f", &side);
    area = (sqrt(3) / 4) * side * side;
    printf("Area: %.2f\n", area);
    return 0;
}
\end{lstlisting}

\newpage

% Program 16: Marks Calculation
\subsection{Calculate Total, Average, and Percentage}
This program calculates the total, average, and percentage of marks for five subjects.

\begin{lstlisting}[caption={Calculate Total, Average, and Percentage}]
#include <stdio.h>
int main() {
    float marks[5], total = 0, average, percentage;
    printf("Enter marks of five subjects (out of 100 each):\n");
    for (int i = 0; i < 5; i++) {
        scanf("%f", &marks[i]);
        total += marks[i];
    }
    average = total / 5;
    percentage = (total / 500) * 100;
    printf("Total: %.2f\n", total);
    printf("Average: %.2f\n", average);
    printf("Percentage: %.2f%%\n", percentage);
    return 0;
}
\end{lstlisting}

\newpage

% Program 17: Simple Interest
\subsection{Calculate Simple Interest}
This program calculates simple interest given principal, time, and rate.

\begin{lstlisting}[caption={Calculate Simple Interest}]
#include <stdio.h>
int main() {
    float principal, time, rate, simple_interest;
    printf("Enter principal, time (years), and rate (%%): ");
    scanf("%f %f %f", &principal, &time, &rate);
    simple_interest = (principal * time * rate) / 100;
    printf("Simple Interest: %.2f\n", simple_interest);
    return 0;
}
\end{lstlisting}

\newpage

% Program 18: Compound Interest
\subsection{Calculate Compound Interest}
This program calculates compound interest given principal, time, and rate.

\begin{lstlisting}[caption={Calculate Compound Interest}]
#include <stdio.h>
#include <math.h>
int main() {
    float principal, time, rate, compound_interest, amount;
    printf("Enter principal, time (years), and rate (%%): ");
    scanf("%f %f %f", &principal, &time, &rate);
    amount = principal * pow((1 + rate / 100), time);
    compound_interest = amount - principal;
    printf("Compound Interest: %.2f\n", compound_interest);
    return 0;
}
\end{lstlisting}

\newpage

% Bitwise Operators Exercises
\section{Bitwise Operators Exercises}

% Program 1: Check LSB
\subsection{Check Least Significant Bit (LSB) of a Number}
This program checks if the Least Significant Bit (LSB) of a number is set (1) or not (0).

\begin{lstlisting}[caption={Check Least Significant Bit (LSB) of a Number}]
#include <stdio.h>
int main() {
    int num;
    printf("Enter a number: ");
    scanf("%d", &num);
    if (num & 1)
        printf("LSB of %d is set (1)\n", num);
    else
        printf("LSB of %d is not set (0)\n", num);
    return 0;
}
\end{lstlisting}

\newpage

% Program 2: Check MSB
\subsection{Check Most Significant Bit (MSB) of a Number}
This program checks if the Most Significant Bit (MSB) of a number is set (1) or not (0).

\begin{lstlisting}[caption={Check Most Significant Bit (MSB) of a Number}]
#include <stdio.h>
int main() {
    int num;
    printf("Enter a number: ");
    scanf("%d", &num);
    int msb = sizeof(num) * 8 - 1;
    if (num & (1 << msb))
        printf("MSB of %d is set (1)\n", num);
    else
        printf("MSB of %d is not set (0)\n", num);
    return 0;
}
\end{lstlisting}

\newpage

% Program 3: Get nth Bit
\subsection{Get nth Bit of a Number}
This program retrieves the nth bit of a given number.

\begin{lstlisting}[caption={Get nth Bit of a Number}]
#include <stdio.h>
int main() {
    int num, n;
    printf("Enter a number: ");
    scanf("%d", &num);
    printf("Enter the bit position (0-based): ");
    scanf("%d", &n);
    if (num & (1 << n))
        printf("The %dth bit of %d is 1\n", n, num);
    else
        printf("The %dth bit of %d is 0\n", n, num);
    return 0;
}
\end{lstlisting}

\newpage

% Program 4: Set nth Bit
\subsection{Set nth Bit of a Number}
This program sets the nth bit of a given number to 1.

\begin{lstlisting}[caption={Set nth Bit of a Number}]
#include <stdio.h>
int main() {
    int num, n;
    printf("Enter a number: ");
    scanf("%d", &num);
    printf("Enter the bit position to set (0-based): ");
    scanf("%d", &n);
    num = num | (1 << n);
    printf("After setting the %dth bit, the number is: %d\n", n, num);
    return 0;
}
\end{lstlisting}

\newpage

% Program 5: Clear nth Bit
\subsection{Clear nth Bit of a Number}
This program clears (sets to 0) the nth bit of a given number.

\begin{lstlisting}[caption={Clear nth Bit of a Number}]
#include <stdio.h>
int main() {
    int num, n;
    printf("Enter a number: ");
    scanf("%d", &num);
    printf("Enter the bit position to clear (0-based): ");
    scanf("%d", &n);
    num = num & ~(1 << n);
    printf("After clearing the %dth bit, the number is: %d\n", n, num);
    return 0;
}
\end{lstlisting}

\newpage

% Program 6: Toggle nth Bit
\subsection{Toggle nth Bit of a Number}
This program toggles (flips) the nth bit of a given number.

\begin{lstlisting}[caption={Toggle nth Bit of a Number}]
#include <stdio.h>
int main() {
    int num, n;
    printf("Enter a number: ");
    scanf("%d", &num);
    printf("Enter the bit position to toggle (0-based): ");
    scanf("%d", &n);
    num = num ^ (1 << n);
    printf("After toggling the %dth bit, the number is: %d\n", n, num);
    return 0;
}
\end{lstlisting}

\newpage

% Program 7: Get Highest Set Bit
\subsection{Get Highest Set Bit of a Number}
This program finds the position of the highest set bit in a given number.

\begin{lstlisting}[caption={Get Highest Set Bit of a Number}]
#include <stdio.h>
int main() {
    int num;
    printf("Enter a number: ");
    scanf("%d", &num);
    if (num == 0) {
        printf("No set bits in 0\n");
        return 0;
    }
    int pos = 0;
    while (num) {
        num >>= 1;
        pos++;
    }
    printf("Highest set bit position (0-based) is: %d\n", pos - 1);
    return 0;
}
\end{lstlisting}

\newpage

% Program 8: Get Lowest Set Bit
\subsection{Get Lowest Set Bit of a Number}
This program finds the position of the lowest set bit in a given number.

\begin{lstlisting}[caption={Get Lowest Set Bit of a Number}]
#include <stdio.h>
int main() {
    int num;
    printf("Enter a number: ");
    scanf("%d", &num);
    if (num == 0) {
        printf("No set bits in 0\n");
        return 0;
    }
    int pos = 0;
    while (!(num & 1)) {
        num >>= 1;
        pos++;
    }
    printf("Lowest set bit position (0-based) is: %d\n", pos);
    return 0;
}
\end{lstlisting}

\newpage

% Program 9: Count Trailing Zeros
\subsection{Count Trailing Zeros in a Binary Number}
This program counts the number of trailing zeros in the binary representation of a number.

\begin{lstlisting}[caption={Count Trailing Zeros in a Binary Number}]
#include <stdio.h>
int main() {
    int num;
    printf("Enter a number: ");
    scanf("%d", &num);
    if (num == 0) {
        printf("Number is 0, all bits are zero\n");
        return 0;
    }
    int count = 0;
    while (!(num & 1)) {
        count++;
        num >>= 1;
    }
    printf("Number of trailing zeros: %d\n", count);
    return 0;
}
\end{lstlisting}

\newpage

% Program 10: Count Leading Zeros
\subsection{Count Leading Zeros in a Binary Number}
This program counts the number of leading zeros in the binary representation of a number.

\begin{lstlisting}[caption={Count Leading Zeros in a Binary Number}]
#include <stdio.h>
int main() {
    int num;
    printf("Enter a number: ");
    scanf("%d", &num);
    if (num == 0) {
        printf("Number is 0, all bits are zero\n");
        return 0;
    }
    int count = 0;
    int msb = sizeof(num) * 8 - 1;
    while (!(num & (1 << msb)) && msb >= 0) {
        count++;
        msb--;
    }
    printf("Number of leading zeros: %d\n", count);
    return 0;
}
\end{lstlisting}

\newpage

% Program 11: Flip Bits
\subsection{Flip Bits of a Binary Number}
This program flips all bits of a given number (i.e., changes 0 to 1 and 1 to 0).

\begin{lstlisting}[caption={Flip Bits of a Binary Number}]
#include <stdio.h>
int main() {
    int num;
    printf("Enter a number: ");
    scanf("%d", &num);
    num = ~num;
    printf("After flipping all bits, the number is: %d\n", num);
    return 0;
}
\end{lstlisting}

\newpage

% Program 12: Count Total Zeros and Ones
\subsection{Count Total Zeros and Ones in a Binary Number}
This program counts the total number of zeros and ones in the binary representation of a number.

\begin{lstlisting}[caption={Count Total Zeros and Ones in a Binary Number}]
#include <stdio.h>
int main() {
    int num;
    printf("Enter a number: ");
    scanf("%d", &num);
    int ones = 0, zeros = 0;
    int bits = sizeof(num) * 8;
    for (int i = 0; i < bits; i++) {
        if (num & (1 << i))
            ones++;
        else
            zeros++;
    }
    printf("Total ones: %d\n", ones);
    printf("Total zeros: %d\n", zeros);
    return 0;
}
\end{lstlisting}

\newpage

% Program 13: Rotate Bits
\subsection{Rotate Bits of a Given Number}
This program rotates the bits of a given number to the left by a specified number of positions.

\begin{lstlisting}[caption={Rotate Bits of a Given Number}]
#include <stdio.h>
int main() {
    int num, rotate;
    printf("Enter a number: ");
    scanf("%d", &num);
    printf("Enter number of positions to rotate left: ");
    scanf("%d", &rotate);
    int bits = sizeof(num) * 8;
    rotate = rotate % bits;
    num = (num << rotate) | (num >> (bits - rotate));
    printf("After rotating left by %d positions, the number is: %d\n", rotate, num);
    return 0;
}
\end{lstlisting}

\newpage

% Program 14: Decimal to Binary
\subsection{Convert Decimal to Binary Using Bitwise Operator}
This program converts a decimal number to its binary representation using bitwise operators.

\begin{lstlisting}[caption={Convert Decimal to Binary Using Bitwise Operator}]
#include <stdio.h>
int main() {
    int num;
    printf("Enter a decimal number: ");
    scanf("%d", &num);
    printf("Binary representation of %d: ", num);
    int bits = sizeof(num) * 8;
    for (int i = bits - 1; i >= 0; i--) {
        if (num & (1 << i))
            printf("1");
        else
            printf("0");
    }
    printf("\n");
    return 0;
}
\end{lstlisting}

\newpage

% Program 15: Swap Two Numbers
\subsection{Swap Two Numbers Using Bitwise Operator}
This program swaps two numbers using the XOR bitwise operator.

\begin{lstlisting}[caption={Swap Two Numbers Using Bitwise Operator}]
#include <stdio.h>
int main() {
    int a, b;
    printf("Enter two numbers (a and b): ");
    scanf("%d %d", &a, &b);
    printf("Before swapping: a = %d, b = %d\n", a, b);
    a = a ^ b;
    b = a ^ b;
    a = a ^ b;
    printf("After swapping: a = %d, b = %d\n", a, b);
    return 0;
}
\end{lstlisting}

\newpage

% Program 16: Check Even or Odd
\subsection{Check Whether a Number is Even or Odd Using Bitwise Operator}
This program checks if a number is even or odd using the LSB.

\begin{lstlisting}[caption={Check Whether a Number is Even or Odd Using Bitwise Operator}]
#include <stdio.h>
int main() {
    int num;
    printf("Enter a number: ");
    scanf("%d", &num);
    if (num & 1)
        printf("%d is odd\n", num);
    else
        printf("%d is even\n", num);
    return 0;
}
\end{lstlisting}

\newpage

% Conditional Operators Exercises
\section{Conditional Operators Exercises}

% Program 1: Maximum Between Two Numbers
\subsection{Find Maximum Between Two Numbers Using Conditional Operator}
This program finds the maximum of two numbers using the conditional operator (? :).

\begin{lstlisting}[caption={Find Maximum Between Two Numbers Using Conditional Operator}]
#include <stdio.h>
int main() {
    int a, b, max;
    printf("Enter two numbers: ");
    scanf("%d %d", &a, &b);
    max = (a > b) ? a : b;
    printf("Maximum number is: %d\n", max);
    return 0;
}
\end{lstlisting}

\newpage

% Program 2: Maximum Between Three Numbers
\subsection{Find Maximum Between Three Numbers Using Conditional Operator}
This program finds the maximum of three numbers using the conditional operator (? :).

\begin{lstlisting}[caption={Find Maximum Between Three Numbers Using Conditional Operator}]
#include <stdio.h>
int main() {
    int a, b, c, max;
    printf("Enter three numbers: ");
    scanf("%d %d %d", &a, &b, &c);
    max = (a > b) ? ((a > c) ? a : c) : ((b > c) ? b : c);
    printf("Maximum number is: %d\n", max);
    return 0;
}
\end{lstlisting}

\newpage

% Program 3: Check Even or Odd
\subsection{Check Whether a Number is Even or Odd Using Conditional Operator}
This program checks if a number is even or odd using the conditional operator (? :).

\begin{lstlisting}[caption={Check Whether a Number is Even or Odd Using Conditional Operator}]
#include <stdio.h>
int main() {
    int num;
    printf("Enter a number: ");
    scanf("%d", &num);
    (num % 2 == 0) ? printf("%d is even\n", num) : printf("%d is odd\n", num);
    return 0;
}
\end{lstlisting}

\newpage

% Program 4: Check Leap Year
\subsection{Check Whether Year is Leap Year or Not Using Conditional Operator}
This program checks if a given year is a leap year using the conditional operator (? :).

\begin{lstlisting}[caption={Check Whether Year is Leap Year or Not Using Conditional Operator}]
#include <stdio.h>
int main() {
    int year;
    printf("Enter a year: ");
    scanf("%d", &year);
    (year % 4 == 0 && year % 100 != 0) || (year % 400 == 0) ?
        printf("%d is a leap year\n", year) :
        printf("%d is not a leap year\n", year);
    return 0;
}
\end{lstlisting}

\newpage

% Program 5: Check Alphabet
\subsection{Check Whether Character is an Alphabet or Not Using Conditional Operator}
This program checks if a character is an alphabet using the conditional operator (? :).

\begin{lstlisting}[caption={Check Whether Character is an Alphabet or Not Using Conditional Operator}]
#include <stdio.h>
int main() {
    char ch;
    printf("Enter a character: ");
    scanf(" %c", &ch);
    (ch >= 'a' && ch <= 'z') || (ch >= 'A' && ch <= 'Z') ?
        printf("%c is an alphabet\n", ch) :
        printf("%c is not an alphabet\n", ch);
    return 0;
}
\end{lstlisting}

\newpage

% If-Else Programming Exercises
\section{If-Else Programming Exercises}

% Program 1: Maximum Between Two Numbers
\subsection{Find Maximum Between Two Numbers}
This program finds the maximum of two numbers using if-else statements.

\begin{lstlisting}[caption={Find Maximum Between Two Numbers}]
#include <stdio.h>
int main() {
    int a, b;
    printf("Enter two numbers: ");
    scanf("%d %d", &a, &b);
    if (a > b)
        printf("Maximum number is: %d\n", a);
    else
        printf("Maximum number is: %d\n", b);
    return 0;
}
\end{lstlisting}

\newpage

% Program 2: Maximum Between Three Numbers
\subsection{Find Maximum Between Three Numbers}
This program finds the maximum of three numbers using if-else statements.

\begin{lstlisting}[caption={Find Maximum Between Three Numbers}]
#include <stdio.h>
int main() {
    int a, b, c;
    printf("Enter three numbers: ");
    scanf("%d %d %d", &a, &b, &c);
    if (a >= b && a >= c)
        printf("Maximum number is: %d\n", a);
    else if (b >= a && b >= c)
        printf("Maximum number is: %d\n", b);
    else
        printf("Maximum number is: %d\n", c);
    return 0;
}
\end{lstlisting}

\newpage

% Program 3: Check Negative, Positive, or Zero
\subsection{Check Whether a Number is Negative, Positive, or Zero}
This program checks if a number is negative, positive, or zero using if-else statements.

\begin{lstlisting}[caption={Check Whether a Number is Negative, Positive, or Zero}]
#include <stdio.h>
int main() {
    int num;
    printf("Enter a number: ");
    scanf("%d", &num);
    if (num > 0)
        printf("%d is positive\n", num);
    else if (num < 0)
        printf("%d is negative\n", num);
    else
        printf("%d is zero\n", num);
    return 0;
}
\end{lstlisting}

\newpage

% Program 4: Check Divisibility by 5 and 11
\subsection{Check Whether a Number is Divisible by 5 and 11}
This program checks if a number is divisible by both 5 and 11 using if-else statements.

\begin{lstlisting}[caption={Check Whether a Number is Divisible by 5 and 11}]
#include <stdio.h>
int main() {
    int num;
    printf("Enter a number: ");
    scanf("%d", &num);
    if (num % 5 == 0 && num % 11 == 0)
        printf("%d is divisible by 5 and 11\n", num);
    else
        printf("%d is not divisible by 5 and 11\n", num);
    return 0;
}
\end{lstlisting}

\newpage

% Program 5: Check Even or Odd
\subsection{Check Whether a Number is Even or Odd}
This program checks if a number is even or odd using if-else statements.

\begin{lstlisting}[caption={Check Whether a Number is Even or Odd}]
#include <stdio.h>
int main() {
    int num;
    printf("Enter a number: ");
    scanf("%d", &num);
    if (num % 2 == 0)
        printf("%d is even\n", num);
    else
        printf("%d is odd\n", num);
    return 0;
}
\end{lstlisting}

\newpage

% Program 6: Check Leap Year
\subsection{Check Whether a Year is Leap Year or Not}
This program checks if a given year is a leap year using if-else statements.

\begin{lstlisting}[caption={Check Whether a Year is Leap Year or Not}]
#include <stdio.h>
int main() {
    int year;
    printf("Enter a year: ");
    scanf("%d", &year);
    if ((year % 4 == 0 && year % 100 != 0) || (year % 400 == 0))
        printf("%d is a leap year\n", year);
    else
        printf("%d is not a leap year\n", year);
    return 0;
}
\end{lstlisting}

\newpage

% Program 7: Check Alphabet
\subsection{Check Whether a Character is an Alphabet or Not}
This program checks if a character is an alphabet using if-else statements.

\begin{lstlisting}[caption={Check Whether a Character is an Alphabet or Not}]
#include <stdio.h>
int main() {
    char ch;
    printf("Enter a character: ");
    scanf(" %c", &ch);
    if ((ch >= 'a' && ch <= 'z') || (ch >= 'A' && ch <= 'Z'))
        printf("%c is an alphabet\n", ch);
    else
        printf("%c is not an alphabet\n", ch);
    return 0;
}
\end{lstlisting}

\newpage

% Program 8: Check Vowel or Consonant
\subsection{Check Whether an Alphabet is Vowel or Consonant}
This program checks if an alphabet is a vowel or consonant using if-else statements.

\begin{lstlisting}[caption={Check Whether an Alphabet is Vowel or Consonant}]
#include <stdio.h>
int main() {
    char ch;
    printf("Enter an alphabet: ");
    scanf(" %c", &ch);
    if ((ch >= 'a' && ch <= 'z') || (ch >= 'A' && ch <= 'Z')) {
        if (ch == 'a' || ch == 'e' || ch == 'i' || ch == 'o' || ch == 'u' ||
            ch == 'A' || ch == 'E' || ch == 'I' || ch == 'O' || ch == 'U')
            printf("%c is a vowel\n", ch);
        else
            printf("%c is a consonant\n", ch);
    } else {
        printf("Please enter a valid alphabet\n");
    }
    return 0;
}
\end{lstlisting}

\newpage

% Program 9: Check Alphabet, Digit, or Special Character
\subsection{Check Whether a Character is Alphabet, Digit, or Special Character}
This program checks if a character is an alphabet, digit, or special character using if-else statements.

\begin{lstlisting}[caption={Check Whether a Character is Alphabet, Digit, or Special Character}]
#include <stdio.h>
int main() {
    char ch;
    printf("Enter a character: ");
    scanf(" %c", &ch);
    if ((ch >= 'a' && ch <= 'z') || (ch >= 'A' && ch <= 'Z'))
        printf("%c is an alphabet\n", ch);
    else if (ch >= '0' && ch <= '9')
        printf("%c is a digit\n", ch);
    else
        printf("%c is a special character\n", ch);
    return 0;
}
\end{lstlisting}

\newpage

% Program 10: Check Uppercase or Lowercase Alphabet
\subsection{Check Whether a Character is Uppercase or Lowercase Alphabet}
This program checks if a character is an uppercase or lowercase alphabet using if-else statements.

\begin{lstlisting}[caption={Check Whether a Character is Uppercase or Lowercase Alphabet}]
#include <stdio.h>
int main() {
    char ch;
    printf("Enter a character: ");
    scanf(" %c", &ch);
    if (ch >= 'A' && ch <= 'Z')
        printf("%c is an uppercase alphabet\n", ch);
    else if (ch >= 'a' && ch <= 'z')
        printf("%c is a lowercase alphabet\n", ch);
    else
        printf("%c is not an alphabet\n", ch);
    return 0;
}
\end{lstlisting}

\newpage

% Program 11: Print Week Day from Week Number
\subsection{Input Week Number and Print Week Day}
This program takes a week number (1-7) and prints the corresponding day of the week using if-else statements.

\begin{lstlisting}[caption={Input Week Number and Print Week Day}]
#include <stdio.h>
int main() {
    int week;
    printf("Enter week number (1-7): ");
    scanf("%d", &week);
    if (week == 1)
        printf("Monday\n");
    else if (week == 2)
        printf("Tuesday\n");
    else if (week == 3)
        printf("Wednesday\n");
    else if (week == 4)
        printf("Thursday\n");
    else if (week == 5)
        printf("Friday\n");
    else if (week == 6)
        printf("Saturday\n");
    else if (week == 7)
        printf("Sunday\n");
    else
        printf("Invalid week number\n");
    return 0;
}
\end{lstlisting}

\newpage

% Program 12: Print Number of Days in Month
\subsection{Input Month Number and Print Number of Days in That Month}
This program takes a month number (1-12) and prints the number of days in that month using if-else statements.

\begin{lstlisting}[caption={Input Month Number and Print Number of Days in That Month}]
#include <stdio.h>
int main() {
    int month;
    printf("Enter month number (1-12): ");
    scanf("%d", &month);
    if (month == 1 || month == 3 || month == 5 || month == 7 || month == 8 || month == 10 || month == 12)
        printf("Number of days: 31\n");
    else if (month == 4 || month == 6 || month == 9 || month == 11)
        printf("Number of days: 30\n");
    else if (month == 2)
        printf("Number of days: 28 or 29 (depending on leap year)\n");
    else
        printf("Invalid month number\n");
    return 0;
}
\end{lstlisting}

\newpage

% Program 13: Count Total Number of Notes
\subsection{Count Total Number of Notes in Given Amount}
This program calculates the number of currency notes (of denominations 500, 100, 50, 20, 10, 5, 2, 1) in a given amount using if-else statements.

\begin{lstlisting}[caption={Count Total Number of Notes in Given Amount}]
#include <stdio.h>
int main() {
    int amount;
    printf("Enter the amount: ");
    scanf("%d", &amount);
    int notes[] = {500, 100, 50, 20, 10, 5, 2, 1};
    int count;
    printf("Number of notes:\n");
    for (int i = 0; i < 8; i++) {
        count = amount / notes[i];
        if (count > 0)
            printf("%d notes of %d\n", count, notes[i]);
        amount = amount % notes[i];
    }
    return 0;
}
\end{lstlisting}

\newpage

% Program 14: Check Triangle Validity by Angles
\subsection{Check Whether Triangle is Valid or Not (Angles)}
This program checks if a triangle is valid based on the sum of its angles (must be 180 degrees) using if-else statements.

\begin{lstlisting}[caption={Check Whether Triangle is Valid or Not (Angles)}]
#include <stdio.h>
int main() {
    float angle1, angle2, angle3;
    printf("Enter three angles of the triangle: ");
    scanf("%f %f %f", &angle1, &angle2, &angle3);
    if (angle1 + angle2 + angle3 == 180 && angle1 > 0 && angle2 > 0 && angle3 > 0)
        printf("Triangle is valid\n");
    else
        printf("Triangle is not valid\n");
    return 0;
}
\end{lstlisting}

\newpage

% Program 15: Check Triangle Validity by Sides
\subsection{Check Whether Triangle is Valid or Not (Sides)}
This program checks if a triangle is valid based on its sides (sum of any two sides must be greater than the third side) using if-else statements.

\begin{lstlisting}[caption={Check Whether Triangle is Valid or Not (Sides)}]
#include <stdio.h>
int main() {
    float side1, side2, side3;
    printf("Enter three sides of the triangle: ");
    scanf("%f %f %f", &side1, &side2, &side3);
    if (side1 + side2 > side3 && side2 + side3 > side1 && side1 + side3 > side2 &&
        side1 > 0 && side2 > 0 && side3 > 0)
        printf("Triangle is valid\n");
    else
        printf("Triangle is not valid\n");
    return 0;
}
\end{lstlisting}

\newpage

% Program 16: Check Triangle Type
\subsection{Check Whether the Triangle is Equilateral, Isosceles, or Scalene}
This program determines if a triangle is equilateral, isosceles, or scalene based on its sides using if-else statements.

\begin{lstlisting}[caption={Check Whether the Triangle is Equilateral, Isosceles, or Scalene}]
#include <stdio.h>
int main() {
    float side1, side2, side3;
    printf("Enter three sides of the triangle: ");
    scanf("%f %f %f", &side1, &side2, &side3);
    if (side1 + side2 > side3 && side2 + side3 > side1 && side1 + side3 > side2 &&
        side1 > 0 && side2 > 0 && side3 > 0) {
        if (side1 == side2 && side2 == side3)
            printf("Triangle is equilateral\n");
        else if (side1 == side2 || side2 == side3 || side1 == side3)
            printf("Triangle is isosceles\n");
        else
            printf("Triangle is scalene\n");
    } else {
        printf("Triangle is not valid\n");
    }
    return 0;
}
\end{lstlisting}

\newpage

% Program 17: Find Roots of Quadratic Equation
\subsection{Find All Roots of a Quadratic Equation}
This program calculates the roots of a quadratic equation \(ax^2 + bx + c = 0\) using if-else statements.

\begin{lstlisting}[caption={Find All Roots of a Quadratic Equation}]
#include <stdio.h>
#include <math.h>
int main() {
    float a, b, c, discriminant, root1, root2;
    printf("Enter coefficients a, b, and c: ");
    scanf("%f %f %f", &a, &b, &c);
    discriminant = b * b - 4 * a * c;
    if (a == 0) {
        printf("Not a quadratic equation (a cannot be 0)\n");
    } else if (discriminant > 0) {
        root1 = (-b + sqrt(discriminant)) / (2 * a);
        root2 = (-b - sqrt(discriminant)) / (2 * a);
        printf("Roots are real and different:\n");
        printf("Root 1 = %.2f\n", root1);
        printf("Root 2 = %.2f\n", root2);
    } else if (discriminant == 0) {
        root1 = -b / (2 * a);
        printf("Roots are real and equal:\n");
        printf("Root 1 = Root 2 = %.2f\n", root1);
    } else {
        float realPart = -b / (2 * a);
        float imagPart = sqrt(-discriminant) / (2 * a);
        printf("Roots are complex:\n");
        printf("Root 1 = %.2f + %.2fi\n", realPart, imagPart);
        printf("Root 2 = %.2f - %.2fi\n", realPart, imagPart);
    }
    return 0;
}
\end{lstlisting}

\newpage

% Program 18: Calculate Profit or Loss
\subsection{Calculate Profit or Loss}
This program calculates profit or loss based on cost price and selling price using if-else statements.

\begin{lstlisting}[caption={Calculate Profit or Loss}]
#include <stdio.h>
int main() {
    float cp, sp, profit, loss;
    printf("Enter cost price: ");
    scanf("%f", &cp);
    printf("Enter selling price: ");
    scanf("%f", &sp);
    if (sp > cp) {
        profit = sp - cp;
        printf("Profit: %.2f\n", profit);
    } else if (cp > sp) {
        loss = cp - sp;
        printf("Loss: %.2f\n", loss);
    } else {
        printf("No profit, no loss\n");
    }
    return 0;
}
\end{lstlisting}

\newpage

% Program 19: Calculate Percentage and Grade
\subsection{Calculate Percentage and Grade from Marks}
This program calculates the percentage and grade based on marks in five subjects using if-else statements.

\begin{lstlisting}[caption={Calculate Percentage and Grade from Marks}]
#include <stdio.h>
int main() {
    float physics, chemistry, biology, mathematics, computer, total, percentage;
    printf("Enter marks of Physics, Chemistry, Biology, Mathematics, and Computer (out of 100 each):\n");
    scanf("%f %f %f %f %f", &physics, &chemistry, &biology, &mathematics, &computer);
    total = physics + chemistry + biology + mathematics + computer;
    percentage = (total / 500) * 100;
    printf("Percentage: %.2f%%\n", percentage);
    if (percentage >= 90)
        printf("Grade: A\n");
    else if (percentage >= 80)
        printf("Grade: B\n");
    else if (percentage >= 70)
        printf("Grade: C\n");
    else if (percentage >= 60)
        printf("Grade: D\n");
    else if (percentage >= 40)
        printf("Grade: E\n");
    else
        printf("Grade: F\n");
    return 0;
}
\end{lstlisting}

\newpage

% Program 20: Calculate Gross Salary
\subsection{Calculate Gross Salary from Basic Salary}
This program calculates the gross salary of an employee based on basic salary, HRA, and DA using if-else statements.

\begin{lstlisting}[caption={Calculate Gross Salary from Basic Salary}]
#include <stdio.h>
int main() {
    float basic, hra, da, gross;
    printf("Enter basic salary: ");
    scanf("%f", &basic);
    if (basic <= 10000) {
        hra = 0.20 * basic;
        da = 0.80 * basic;
    } else if (basic <= 20000) {
        hra = 0.25 * basic;
        da = 0.90 * basic;
    } else {
        hra = 0.30 * basic;
        da = 0.95 * basic;
    }
    gross = basic + hra + da;
    printf("Gross salary: %.2f\n", gross);
    return 0;
}
\end{lstlisting}

\newpage

% Program 21: Calculate Electricity Bill
\subsection{Calculate Total Electricity Bill}
This program calculates the total electricity bill based on unit charges and a surcharge using if-else statements.

\begin{lstlisting}[caption={Calculate Total Electricity Bill}]
#include <stdio.h>
int main() {
    float units, bill = 0, surcharge, total;
    printf("Enter electricity units consumed: ");
    scanf("%f", &units);
    if (units <= 50) {
        bill = units * 0.50;
    } else if (units <= 150) {
        bill = 50 * 0.50 + (units - 50) * 0.75;
    } else if (units <= 250) {
        bill = 50 * 0.50 + 100 * 0.75 + (units - 150) * 1.20;
    } else {
        bill = 50 * 0.50 + 100 * 0.75 + 100 * 1.20 + (units - 250) * 1.50;
    }
    surcharge = 0.20 * bill;
    total = bill + surcharge;
    printf("Total electricity bill: Rs. %.2f\n", total);
    return 0;
}
\end{lstlisting}

\newpage

% Switch Case Programming Exercises
\section{Switch Case Programming Exercises}

% Program 1: Print Day of Week
\subsection{Print Day of Week Name Using Switch Case}
This program takes a week number (1-7) and prints the corresponding day of the week using a switch case statement.

\begin{lstlisting}[caption={Print Day of Week Name Using Switch Case}]
#include <stdio.h>
int main() {
    int week;
    printf("Enter week number (1-7): ");
    scanf("%d", &week);
    switch (week) {
        case 1:
            printf("Monday\n");
            break;
        case 2:
            printf("Tuesday\n");
            break;
        case 3:
            printf("Wednesday\n");
            break;
        case 4:
            printf("Thursday\n");
            break;
        case 5:
            printf("Friday\n");
            break;
        case 6:
            printf("Saturday\n");
            break;
        case 7:
            printf("Sunday\n");
            break;
        default:
            printf("Invalid week number\n");
    }
    return 0;
}
\end{lstlisting}

\newpage

% Program 2: Print Total Number of Days in a Month
\subsection{Print Total Number of Days in a Month Using Switch Case}
This program takes a month number (1-12) and prints the number of days in that month using a switch case statement.

\begin{lstlisting}[caption={Print Total Number of Days in a Month Using Switch Case}]
#include <stdio.h>
int main() {
    int month;
    printf("Enter month number (1-12): ");
    scanf("%d", &month);
    switch (month) {
        case 1:
        case 3:
        case 5:
        case 7:
        case 8:
        case 10:
        case 12:
            printf("Number of days: 31\n");
            break;
        case 4:
        case 6:
        case 9:
        case 11:
            printf("Number of days: 30\n");
            break;
        case 2:
            printf("Number of days: 28 or 29 (depending on leap year)\n");
            break;
        default:
            printf("Invalid month number\n");
    }
    return 0;
}
\end{lstlisting}

\newpage

% Program 3: Check Vowel or Consonant
\subsection{Check Whether an Alphabet is Vowel or Consonant Using Switch Case}
This program checks if an alphabet is a vowel or consonant using a switch case statement.

\begin{lstlisting}[caption={Check Whether an Alphabet is Vowel or Consonant Using Switch Case}]
#include <stdio.h>
int main() {
    char ch;
    printf("Enter an alphabet: ");
    scanf(" %c", &ch);
    switch (ch) {
        case 'a':
        case 'e':
        case 'i':
        case 'o':
        case 'u':
        case 'A':
        case 'E':
        case 'I':
        case 'O':
        case 'U':
            printf("%c is a vowel\n", ch);
            break;
        default:
            if ((ch >= 'a' && ch <= 'z') || (ch >= 'A' && ch <= 'Z'))
                printf("%c is a consonant\n", ch);
            else
                printf("Not a valid alphabet\n");
    }
    return 0;
}
\end{lstlisting}

\newpage

% Program 4: Find Maximum Between Two Numbers
\subsection{Find Maximum Between Two Numbers Using Switch Case}
This program finds the maximum of two numbers using a switch case statement with a user choice.

\begin{lstlisting}[caption={Find Maximum Between Two Numbers Using Switch Case}]
#include <stdio.h>
int main() {
    int a, b, choice;
    printf("Enter two numbers: ");
    scanf("%d %d", &a, &b);
    printf("Enter 1 to find maximum: ");
    scanf("%d", &choice);
    switch (choice) {
        case 1:
            if (a > b)
                printf("Maximum number is: %d\n", a);
            else
                printf("Maximum number is: %d\n", b);
            break;
        default:
            printf("Invalid choice\n");
    }
    return 0;
}
\end{lstlisting}

\newpage

% Program 5: Check Even or Odd
\subsection{Check Whether a Number is Even or Odd Using Switch Case}
This program checks if a number is even or odd using a switch case statement based on the remainder.

\begin{lstlisting}[caption={Check Whether a Number is Even or Odd Using Switch Case}]
#include <stdio.h>
int main() {
    int num, remainder;
    printf("Enter a number: ");
    scanf("%d", &num);
    remainder = num % 2;
    switch (remainder) {
        case 0:
            printf("%d is even\n", num);
            break;
        case 1:
            printf("%d is odd\n", num);
            break;
    }
    return 0;
}
\end{lstlisting}

\newpage

% Program 6: Check Positive, Negative, or Zero
\subsection{Check Whether a Number is Positive, Negative, or Zero Using Switch Case}
This program checks if a number is positive, negative, or zero using a switch case statement.

\begin{lstlisting}[caption={Check Whether a Number is Positive, Negative, or Zero Using Switch Case}]
#include <stdio.h>
int main() {
    int num, sign;
    printf("Enter a number: ");
    scanf("%d", &num);
    if (num > 0)
        sign = 1;
    else if (num < 0)
        sign = -1;
    else
        sign = 0;
    switch (sign) {
        case 1:
            printf("%d is positive\n", num);
            break;
        case -1:
            printf("%d is negative\n", num);
            break;
        case 0:
            printf("%d is zero\n", num);
            break;
    }
    return 0;
}
\end{lstlisting}

\newpage

% Program 7: Find Roots of Quadratic Equation
\subsection{Find Roots of a Quadratic Equation Using Switch Case}
This program calculates the roots of a quadratic equation \(ax^2 + bx + c = 0\) using a switch case statement.

\begin{lstlisting}[caption={Find Roots of a Quadratic Equation Using Switch Case}]
#include <stdio.h>
#include <math.h>
int main() {
    float a, b, c, discriminant, root1, root2;
    printf("Enter coefficients a, b, and c: ");
    scanf("%f %f %f", &a, &b, &c);
    discriminant = b * b - 4 * a * c;
    if (a == 0) {
        printf("Not a quadratic equation (a cannot be 0)\n");
    } else {
        switch (discriminant > 0) {
            case 1:
                root1 = (-b + sqrt(discriminant)) / (2 * a);
                root2 = (-b - sqrt(discriminant)) / (2 * a);
                printf("Roots are real and different:\n");
                printf("Root 1 = %.2f\n", root1);
                printf("Root 2 = %.2f\n", root2);
                break;
            case 0:
                switch (discriminant == 0) {
                    case 1:
                        root1 = -b / (2 * a);
                        printf("Roots are real and equal:\n");
                        printf("Root 1 = Root 2 = %.2f\n", root1);
                        break;
                    case 0:
                        float realPart = -b / (2 * a);
                        float imagPart = sqrt(-discriminant) / (2 * a);
                        printf("Roots are complex:\n");
                        printf("Root 1 = %.2f + %.2fi\n", realPart, imagPart);
                        printf("Root 2 = %.2f - %.2fi\n", realPart, imagPart);
                        break;
                }
                break;
        }
    }
    return 0;
}
\end{lstlisting}

\newpage

% Program 8: Simple Calculator
\subsection{Create Simple Calculator Using Switch Case}
This program performs basic arithmetic operations (add, subtract, multiply, divide) using a switch case statement.

\begin{lstlisting}[caption={Create Simple Calculator Using Switch Case}]
#include <stdio.h>
int main() {
    float a, b, result;
    int choice;
    printf("Enter two numbers: ");
    scanf("%f %f", &a, &b);
    printf("Enter choice (1: Add, 2: Subtract, 3: Multiply, 4: Divide): ");
    scanf("%d", &choice);
    switch (choice) {
        case 1:
            result = a + b;
            printf("Result: %.2f\n", result);
            break;
        case 2:
            result = a - b;
            printf("Result: %.2f\n", result);
            break;
        case 3:
            result = a * b;
            printf("Result: %.2f\n", result);
            break;
        case 4:
            if (b != 0)
                result = a / b;
            else {
                printf("Error: Division by zero\n");
                return 0;
            }
            printf("Result: %.2f\n", result);
            break;
        default:
            printf("Invalid choice\n");
    }
    return 0;
}
\end{lstlisting}

\newpage

% Loop Programming Exercises
\section{Loop Programming Exercises}

% Program 1: Print Natural Numbers 1 to n (While Loop)
\subsection{Print All Natural Numbers from 1 to n Using While Loop}
This program prints all natural numbers from 1 to a user-specified number \(n\) using a while loop.

\begin{lstlisting}[caption={Print All Natural Numbers from 1 to n Using While Loop}]
#include <stdio.h>
int main() {
    int n, i = 1;
    printf("Enter a positive number: ");
    scanf("%d", &n);
    while (i <= n) {
        printf("%d ", i);
        i++;
    }
    printf("\n");
    return 0;
}
\end{lstlisting}

\newpage

% Program 2: Print Natural Numbers Reverse (While Loop)
\subsection{Print All Natural Numbers in Reverse (from n to 1) Using While Loop}
This program prints all natural numbers from a user-specified number \(n\) to 1 in reverse order using a while loop.

\begin{lstlisting}[caption={Print All Natural Numbers in Reverse (from n to 1) Using While Loop}]
#include <stdio.h>
int main() {
    int n;
    printf("Enter a positive number: ");
    scanf("%d", &n);
    while (n >= 1) {
        printf("%d ", n);
        n--;
    }
    printf("\n");
    return 0;
}
\end{lstlisting}

\newpage

% Program 3: Print Alphabets a to z (While Loop)
\subsection{Print All Alphabets from a to z Using While Loop}
This program prints all lowercase alphabets from 'a' to 'z' using a while loop.

\begin{lstlisting}[caption={Print All Alphabets from a to z Using While Loop}]
#include <stdio.h>
int main() {
    char ch = 'a';
    while (ch <= 'z') {
        printf("%c ", ch);
        ch++;
    }
    printf("\n");
    return 0;
}
\end{lstlisting}

\newpage

% Program 4: Print Even Numbers 1 to 100 (While Loop)
\subsection{Print All Even Numbers Between 1 to 100 Using While Loop}
This program prints all even numbers between 1 and 100 using a while loop.

\begin{lstlisting}[caption={Print All Even Numbers Between 1 to 100 Using While Loop}]
#include <stdio.h>
int main() {
    int i = 2;
    while (i <= 100) {
        printf("%d ", i);
        i += 2;
    }
    printf("\n");
    return 0;
}
\end{lstlisting}

\newpage

% Program 5: Print Odd Numbers 1 to 100
\subsection{Print All Odd Numbers Between 1 to 100}
This program prints all odd numbers between 1 and 100 using a while loop.

\begin{lstlisting}[caption={Print All Odd Numbers Between 1 to 100}]
#include <stdio.h>
int main() {
    int i = 1;
    while (i <= 100) {
        printf("%d ", i);
        i += 2;
    }
    printf("\n");
    return 0;
}
\end{lstlisting}

\newpage

% Program 6: Sum of Natural Numbers 1 to n
\subsection{Find Sum of All Natural Numbers Between 1 to n}
This program calculates the sum of all natural numbers from 1 to a user-specified number \(n\) using a while loop.

\begin{lstlisting}[caption={Find Sum of All Natural Numbers Between 1 to n}]
#include <stdio.h>
int main() {
    int n, i = 1, sum = 0;
    printf("Enter a positive number: ");
    scanf("%d", &n);
    while (i <= n) {
        sum += i;
        i++;
    }
    printf("Sum = %d\n", sum);
    return 0;
}
\end{lstlisting}

\newpage

% Program 7: Sum of Even Numbers 1 to n
\subsection{Find Sum of All Even Numbers Between 1 to n}
This program calculates the sum of all even numbers from 1 to a user-specified number \(n\) using a while loop.

\begin{lstlisting}[caption={Find Sum of All Even Numbers Between 1 to n}]
#include <stdio.h>
int main() {
    int n, i = 2, sum = 0;
    printf("Enter a positive number: ");
    scanf("%d", &n);
    while (i <= n) {
        sum += i;
        i += 2;
    }
    printf("Sum of even numbers = %d\n", sum);
    return 0;
}
\end{lstlisting}

\newpage

% Program 8: Sum of Odd Numbers 1 to n
\subsection{Find Sum of All Odd Numbers Between 1 to n}
This program calculates the sum of all odd numbers from 1 to a user-specified number \(n\) using a while loop.

\begin{lstlisting}[caption={Find Sum of All Odd Numbers Between 1 to n}]
#include <stdio.h>
int main() {
    int n, i = 1, sum = 0;
    printf("Enter a positive number: ");
    scanf("%d", &n);
    while (i <= n) {
        sum += i;
        i += 2;
    }
    printf("Sum of odd numbers = %d\n", sum);
    return 0;
}
\end{lstlisting}

\newpage

% Program 9: Multiplication Table
\subsection{Print Multiplication Table of Any Number}
This program prints the multiplication table (up to 10) of a user-specified number using a while loop.

\begin{lstlisting}[caption={Print Multiplication Table of Any Number}]
#include <stdio.h>
int main() {
    int n, i = 1;
    printf("Enter a number: ");
    scanf("%d", &n);
    while (i <= 10) {
        printf("%d x %d = %d\n", n, i, n * i);
        i++;
    }
    return 0;
}
\end{lstlisting}

\newpage

% Program 10: Count Number of Digits
\subsection{Count Number of Digits in a Number}
This program counts the number of digits in a given integer using a while loop.

\begin{lstlisting}[caption={Count Number of Digits in a Number}]
#include <stdio.h>
int main() {
    int num, count = 0;
    printf("Enter a number: ");
    scanf("%d", &num);
    while (num != 0) {
        count++;
        num /= 10;
    }
    printf("Number of digits = %d\n", count);
    return 0;
}
\end{lstlisting}

\newpage

% Program 11: Find First and Last Digit
\subsection{Find First and Last Digit of a Number}
This program finds the first and last digits of a given number using a while loop.

\begin{lstlisting}[caption={Find First and Last Digit of a Number}]
#include <stdio.h>
int main() {
    int num, first, last;
    printf("Enter a number: ");
    scanf("%d", &num);
    last = num % 10;
    while (num >= 10) {
        num /= 10;
    }
    first = num;
    printf("First digit = %d, Last digit = %d\n", first, last);
    return 0;
}
\end{lstlisting}

\newpage

% Program 12: Sum of First and Last Digit
\subsection{Find Sum of First and Last Digit of a Number}
This program calculates the sum of the first and last digits of a given number using a while loop.

\begin{lstlisting}[caption={Find Sum of First and Last Digit of a Number}]
#include <stdio.h>
int main() {
    int num, first, last, sum;
    printf("Enter a number: ");
    scanf("%d", &num);
    last = num % 10;
    while (num >= 10) {
        num /= 10;
    }
    first = num;
    sum = first + last;
    printf("Sum of first and last digit = %d\n", sum);
    return 0;
}
\end{lstlisting}

\newpage

% Program 13: Swap First and Last Digits
\subsection{Swap First and Last Digits of a Number}
This program swaps the first and last digits of a given number using a while loop.

\begin{lstlisting}[caption={Swap First and Last Digits of a Number}]
#include <stdio.h>
#include <math.h>
int main() {
    int num, original, last, first, digits, swapped;
    printf("Enter a number: ");
    scanf("%d", &num);
    original = num;
    last = num % 10;
    while (num >= 10) {
        num /= 10;
    }
    first = num;
    digits = 0;
    num = original;
    while (num != 0) {
        digits++;
        num /= 10;
    }
    num = original;
    swapped = last;
    swapped = swapped * (int)pow(10, digits - 1) + (num % (int)pow(10, digits - 1)) / 10 * 10 + first;
    printf("After swapping: %d\n", swapped);
    return 0;
}
\end{lstlisting}

\newpage

% Program 14: Sum of Digits
\subsection{Calculate Sum of Digits of a Number}
This program calculates the sum of all digits in a given number using a while loop.

\begin{lstlisting}[caption={Calculate Sum of Digits of a Number}]
#include <stdio.h>
int main() {
    int num, sum = 0, digit;
    printf("Enter a number: ");
    scanf("%d", &num);
    while (num != 0) {
        digit = num % 10;
        sum += digit;
        num /= 10;
    }
    printf("Sum of digits = %d\n", sum);
    return 0;
}
\end{lstlisting}

\newpage

% Program 15: Product of Digits
\subsection{Calculate Product of Digits of a Number}
This program calculates the product of all digits in a given number using a while loop.

\begin{lstlisting}[caption={Calculate Product of Digits of a Number}]
#include <stdio.h>
int main() {
    int num, product = 1, digit;
    printf("Enter a number: ");
    scanf("%d", &num);
    while (num != 0) {
        digit = num % 10;
        product *= digit;
        num /= 10;
    }
    printf("Product of digits = %d\n", product);
    return 0;
}
\end{lstlisting}

\newpage

% Program 16: Print Reverse of a Number
\subsection{Enter a Number and Print Its Reverse}
This program reverses a given number and prints it using a while loop.

\begin{lstlisting}[caption={Enter a Number and Print Its Reverse}]
#include <stdio.h>
int main() {
    int num, reversed = 0, digit;
    printf("Enter a number: ");
    scanf("%d", &num);
    while (num != 0) {
        digit = num % 10;
        reversed = reversed * 10 + digit;
        num /= 10;
    }
    printf("Reversed number = %d\n", reversed);
    return 0;
}
\end{lstlisting}

\newpage

% Program 17: Check Palindrome
\subsection{Check Whether a Number is Palindrome or Not}
This program checks if a given number is a palindrome using a while loop.

\begin{lstlisting}[caption={Check Whether a Number is Palindrome or Not}]
#include <stdio.h>
int main() {
    int num, original, reversed = 0, digit;
    printf("Enter a number: ");
    scanf("%d", &num);
    original = num;
    while (num != 0) {
        digit = num % 10;
        reversed = reversed * 10 + digit;
        num /= 10;
    }
    if (original == reversed)
        printf("%d is a palindrome\n", original);
    else
        printf("%d is not a palindrome\n", original);
    return 0;
}
\end{lstlisting}

\newpage

% Program 18: Frequency of Digits
\subsection{Find Frequency of Each Digit in a Given Integer}
This program finds the frequency of each digit (0-9) in a given integer using a while loop.

\begin{lstlisting}[caption={Find Frequency of Each Digit in a Given Integer}]
#include <stdio.h>
int main() {
    int num, digit, freq[10] = {0};
    printf("Enter a number: ");
    scanf("%d", &num);
    while (num != 0) {
        digit = num % 10;
        freq[digit]++;
        num /= 10;
    }
    printf("Frequency of digits:\n");
    for (int i = 0; i < 10; i++) {
        if (freq[i] > 0)
            printf("Digit %d: %d times\n", i, freq[i]);
    }
    return 0;
}
\end{lstlisting}

\newpage

% Program 19: Print Number in Words
\subsection{Enter a Number and Print It in Words}
This program converts a number (up to 3 digits) into words using a while loop for processing.

\begin{lstlisting}[caption={Enter a Number and Print It in Words}]
#include <stdio.h>
int main() {
    int num, digit;
    printf("Enter a number (1-999): ");
    scanf("%d", &num);
    if (num == 0) {
        printf("zero\n");
        return 0;
    }
    char *ones[] = {"", "one", "two", "three", "four", "five", "six", "seven", "eight", "nine"};
    char *teens[] = {"", "eleven", "twelve", "thirteen", "fourteen", "fifteen", "sixteen", "seventeen", "eighteen", "nineteen"};
    char *tens[] = {"", "ten", "twenty", "thirty", "forty", "fifty", "sixty", "seventy", "eighty", "ninety"};
    if (num > 999 || num < 0) {
        printf("Invalid input\n");
        return 0;
    }
    int hundreds = num / 100;
    int remainder = num % 100;
    int tens_digit = remainder / 10;
    int ones_digit = remainder % 10;
    if (hundreds > 0) {
        printf("%s hundred", ones[hundreds]);
        if (remainder > 0)
            printf(" and ");
    }
    if (remainder > 0) {
        if (tens_digit == 1 && ones_digit > 0) {
            printf("%s", teens[ones_digit]);
        } else {
            if (tens_digit > 0)
                printf("%s", tens[tens_digit]);
            if (ones_digit > 0)
                printf(" %s", ones[ones_digit]);
        }
    }
    printf("\n");
    return 0;
}
\end{lstlisting}

\newpage

% Program 20: Print ASCII Characters
\subsection{Print All ASCII Characters with Their Values}
This program prints all ASCII characters (0-127) along with their values using a while loop.

\begin{lstlisting}[caption={Print All ASCII Characters with Their Values}]
#include <stdio.h>
int main() {
    int i = 0;
    while (i <= 127) {
        printf("ASCII value %d = %c\n", i, i);
        i++;
    }
    return 0;
}
\end{lstlisting}

\newpage

% Program 21: Power of a Number (For Loop)
\subsection{Find Power of a Number Using For Loop}
This program calculates the power of a number (\(base^{exponent}\)) using a for loop.

\begin{lstlisting}[caption={Find Power of a Number Using For Loop}]
#include <stdio.h>
int main() {
    int base, exp, result = 1;
    printf("Enter base and exponent: ");
    scanf("%d %d", &base, &exp);
    for (int i = 1; i <= exp; i++) {
        result *= base;
    }
    printf("%d^%d = %d\n", base, exp, result);
    return 0;
}
\end{lstlisting}

\newpage

% Program 22: Find All Factors
\subsection{Find All Factors of a Number}
This program finds and prints all factors of a given number using a for loop.

\begin{lstlisting}[caption={Find All Factors of a Number}]
#include <stdio.h>
int main() {
    int num;
    printf("Enter a number: ");
    scanf("%d", &num);
    printf("Factors of %d are: ", num);
    for (int i = 1; i <= num; i++) {
        if (num % i == 0)
            printf("%d ", i);
    }
    printf("\n");
    return 0;
}
\end{lstlisting}

\newpage

% Program 23: Calculate Factorial
\subsection{Calculate Factorial of a Number}
This program calculates the factorial of a given number using a for loop.

\begin{lstlisting}[caption={Calculate Factorial of a Number}]
#include <stdio.h>
int main() {
    int num, factorial = 1;
    printf("Enter a positive number: ");
    scanf("%d", &num);
    for (int i = 1; i <= num; i++) {
        factorial *= i;
    }
    printf("Factorial of %d = %d\n", num, factorial);
    return 0;
}
\end{lstlisting}

\newpage

% Program 24: Find HCF (GCD)
\subsection{Find HCF (GCD) of Two Numbers}
This program calculates the Highest Common Factor (HCF) or Greatest Common Divisor (GCD) of two numbers using a for loop.

\begin{lstlisting}[caption={Find HCF (GCD) of Two Numbers}]
#include <stdio.h>
int main() {
    int a, b, hcf;
    printf("Enter two numbers: ");
    scanf("%d %d", &a, &b);
    for (int i = 1; i <= a && i <= b; i++) {
        if (a % i == 0 && b % i == 0)
            hcf = i;
    }
    printf("HCF of %d and %d = %d\n", a, b, hcf);
    return 0;
}
\end{lstlisting}

\newpage

% Program 25: Find LCM of Two Numbers
\subsection{Find LCM of Two Numbers}
This program calculates the Least Common Multiple (LCM) of two numbers using a for loop.

\begin{lstlisting}[caption={Find LCM of Two Numbers}]
#include <stdio.h>
int main() {
    int a, b, max, lcm;
    printf("Enter two numbers: ");
    scanf("%d %d", &a, &b);
    max = (a > b) ? a : b;
    for (int i = max; i <= a * b; i += max) {
        if (i % a == 0 && i % b == 0) {
            lcm = i;
            break;
        }
    }
    printf("LCM of %d and %d = %d\n", a, b, lcm);
    return 0;
}
\end{lstlisting}

\newpage

% Program 26: Check Prime Number
\subsection{Check Whether a Number is Prime Number or Not}
This program checks if a given number is a prime number using a for loop.

\begin{lstlisting}[caption={Check Whether a Number is Prime Number or Not}]
#include <stdio.h>
int main() {
    int num, isPrime = 1;
    printf("Enter a number: ");
    scanf("%d", &num);
    for (int i = 2; i <= num / 2; i++) {
        if (num % i == 0) {
            isPrime = 0;
            break;
        }
    }
    if (num <= 1)
        printf("%d is not a prime number\n", num);
    else if (isPrime)
        printf("%d is a prime number\n", num);
    else
        printf("%d is not a prime number\n", num);
    return 0;
}
\end{lstlisting}

\newpage

% Program 27: Print Prime Numbers 1 to n
\subsection{Print All Prime Numbers Between 1 to n}
This program prints all prime numbers between 1 and a user-specified number \(n\) using a for loop.

\begin{lstlisting}[caption={Print All Prime Numbers Between 1 to n}]
#include <stdio.h>
int main() {
    int n;
    printf("Enter a number: ");
    scanf("%d", &n);
    printf("Prime numbers between 1 and %d are: ", n);
    for (int i = 2; i <= n; i++) {
        int isPrime = 1;
        for (int j = 2; j <= i / 2; j++) {
            if (i % j == 0) {
                isPrime = 0;
                break;
            }
        }
        if (isPrime && i > 1)
            printf("%d ", i);
    }
    printf("\n");
    return 0;
}
\end{lstlisting}

\newpage

% Program 28: Sum of Prime Numbers 1 to n
\subsection{Find Sum of All Prime Numbers Between 1 to n}
This program calculates the sum of all prime numbers between 1 and a user-specified number \(n\) using a for loop.

\begin{lstlisting}[caption={Find Sum of All Prime Numbers Between 1 to n}]
#include <stdio.h>
int main() {
    int n, sum = 0;
    printf("Enter a number: ");
    scanf("%d", &n);
    for (int i = 2; i <= n; i++) {
        int isPrime = 1;
        for (int j = 2; j <= i / 2; j++) {
            if (i % j == 0) {
                isPrime = 0;
                break;
            }
        }
        if (isPrime && i > 1)
            sum += i;
    }
    printf("Sum of prime numbers = %d\n", sum);
    return 0;
}
\end{lstlisting}

\newpage

% Program 29: Prime Factors
\subsection{Find All Prime Factors of a Number}
This program finds and prints all prime factors of a given number using a for loop.

\begin{lstlisting}[caption={Find All Prime Factors of a Number}]
#include <stdio.h>
int main() {
    int num;
    printf("Enter a number: ");
    scanf("%d", &num);
    printf("Prime factors of %d are: ", num);
    for (int i = 2; i <= num; i++) {
        while (num % i == 0) {
            printf("%d ", i);
            num /= i;
        }
    }
    printf("\n");
    return 0;
}
\end{lstlisting}

\newpage

% Program 30: Check Armstrong Number
\subsection{Check Whether a Number is Armstrong Number or Not}
This program checks if a given number is an Armstrong number using a for loop for digit counting and summation.

\begin{lstlisting}[caption={Check Whether a Number is Armstrong Number or Not}]
#include <stdio.h>
#include <math.h>
int main() {
    int num, original, remainder, result = 0, digits = 0;
    printf("Enter a number: ");
    scanf("%d", &num);
    original = num;
    while (original != 0) {
        digits++;
        original /= 10;
    }
    original = num;
    while (original != 0) {
        remainder = original % 10;
        int power = 1;
        for (int i = 0; i < digits; i++)
            power *= remainder;
        result += power;
        original /= 10;
    }
    if (result == num)
        printf("%d is an Armstrong number\n", num);
    else
        printf("%d is not an Armstrong number\n", num);
    return 0;
}
\end{lstlisting}

\newpage

% Program 31: Print Armstrong Numbers 1 to n
\subsection{Print All Armstrong Numbers Between 1 to n}
This program prints all Armstrong numbers between 1 and a user-specified number \(n\) using a for loop.

\begin{lstlisting}[caption={Print All Armstrong Numbers Between 1 to n}]
#include <stdio.h>
#include <math.h>
int main() {
    int n;
    printf("Enter a number: ");
    scanf("%d", &n);
    printf("Armstrong numbers between 1 and %d are: ", n);
    for (int num = 1; num <= n; num++) {
        int original = num, remainder, result = 0, digits = 0;
        int temp = num;
        while (temp != 0) {
            digits++;
            temp /= 10;
        }
        temp = num;
        while (temp != 0) {
            remainder = temp % 10;
            int power = 1;
            for (int i = 0; i < digits; i++)
                power *= remainder;
            result += power;
            temp /= 10;
        }
        if (result == num)
            printf("%d ", num);
    }
    printf("\n");
    return 0;
}
\end{lstlisting}

\newpage

% Program 32: Check Perfect Number
\subsection{Check Whether a Number is Perfect Number or Not}
This program checks if a given number is a perfect number using a for loop.

\begin{lstlisting}[caption={Check Whether a Number is Perfect Number or Not}]
#include <stdio.h>
int main() {
    int num, sum = 0;
    printf("Enter a number: ");
    scanf("%d", &num);
    for (int i = 1; i <= num / 2; i++) {
        if (num % i == 0)
            sum += i;
    }
    if (sum == num && num != 0)
        printf("%d is a perfect number\n", num);
    else
        printf("%d is not a perfect number\n", num);
    return 0;
}
\end{lstlisting}

\newpage

% Program 33: Print Perfect Numbers 1 to n
\subsection{Print All Perfect Numbers Between 1 to n}
This program prints all perfect numbers between 1 and a user-specified number \(n\) using a for loop.

\begin{lstlisting}[caption={Print All Perfect Numbers Between 1 to n}]
#include <stdio.h>
int main() {
    int n;
    printf("Enter a number: ");
    scanf("%d", &n);
    printf("Perfect numbers between 1 and %d are: ", n);
    for (int num = 1; num <= n; num++) {
        int sum = 0;
        for (int i = 1; i <= num / 2; i++) {
            if (num % i == 0)
                sum += i;
        }
        if (sum == num && num != 0)
            printf("%d ", num);
    }
    printf("\n");
    return 0;
}
\end{lstlisting}

\newpage

% Program 34: Check Strong Number
\subsection{Check Whether a Number is Strong Number or Not}
This program checks if a given number is a strong number using a for loop for factorial calculation.

\begin{lstlisting}[caption={Check Whether a Number is Strong Number or Not}]
#include <stdio.h>
int main() {
    int num, original, sum = 0, digit, fact;
    printf("Enter a number: ");
    scanf("%d", &num);
    original = num;
    while (num != 0) {
        digit = num % 10;
        fact = 1;
        for (int i = 1; i <= digit; i++)
            fact *= i;
        sum += fact;
        num /= 10;
    }
    if (sum == original)
        printf("%d is a strong number\n", original);
    else
        printf("%d is not a strong number\n", original);
    return 0;
}
\end{lstlisting}

\newpage

% Program 35: Print Strong Numbers 1 to n
\subsection{Print All Strong Numbers Between 1 to n}
This program prints all strong numbers between 1 and a user-specified number \(n\) using a for loop.

\begin{lstlisting}[caption={Print All Strong Numbers Between 1 to n}]
#include <stdio.h>
int main() {
    int n;
    printf("Enter a number: ");
    scanf("%d", &n);
    printf("Strong numbers between 1 and %d are: ", n);
    for (int num = 1; num <= n; num++) {
        int original = num, sum = 0, digit, fact;
        int temp = num;
        while (temp != 0) {
            digit = temp % 10;
            fact = 1;
            for (int i = 1; i <= digit; i++)
                fact *= i;
            sum += fact;
            temp /= 10;
        }
        if (sum == original)
            printf("%d ", num);
    }
    printf("\n");
    return 0;
}
\end{lstlisting}

\newpage

% Program 36: Print Fibonacci Series
\subsection{Print Fibonacci Series Up to n Terms}
This program prints the Fibonacci series up to a user-specified number of terms using a for loop.

\begin{lstlisting}[caption={Print Fibonacci Series Up to n Terms}]
#include <stdio.h>
int main() {
    int n, first = 0, second = 1, next;
    printf("Enter the number of terms: ");
    scanf("%d", &n);
    printf("Fibonacci Series: ");
    for (int i = 0; i < n; i++) {
        if (i <= 1)
            next = i;
        else {
            next = first + second;
            first = second;
            second = next;
        }
        printf("%d ", next);
    }
    printf("\n");
    return 0;
}
\end{lstlisting}

\newpage

% Program 37: One’s Complement of Binary
\subsection{Find One’s Complement of a Binary Number}
This program calculates the one’s complement of a binary number (represented as a decimal) using a for loop.

\begin{lstlisting}[caption={Find One’s Complement of a Binary Number}]
#include <stdio.h>
int main() {
    int num, bin[32], comp[32], i = 0;
    printf("Enter a decimal number: ");
    scanf("%d", &num);
    while (num > 0) {
        bin[i] = num % 2;
        num /= 2;
        i++;
    }
    printf("Binary: ");
    for (int j = i - 1; j >= 0; j--)
        printf("%d", bin[j]);
    printf("\nOne's Complement: ");
    for (int j = i - 1; j >= 0; j--)
        printf("%d", (bin[j] == 0) ? 1 : 0);
    printf("\n");
    return 0;
}
\end{lstlisting}

\newpage

% Program 38: Two’s Complement of Binary
\subsection{Find Two’s Complement of a Binary Number}
This program calculates the two’s complement of a binary number (represented as a decimal) using a for loop.

\begin{lstlisting}[caption={Find Two’s Complement of a Binary Number}]
#include <stdio.h>
int main() {
    int num, bin[32], comp[32], i = 0, carry = 1;
    printf("Enter a decimal number: ");
    scanf("%d", &num);
    while (num > 0) {
        bin[i] = num % 2;
        num /= 2;
        i++;
    }
    printf("Binary: ");
    for (int j = i - 1; j >= 0; j--)
        printf("%d", bin[j]);
    printf("\nOne's Complement: ");
    for (int j = i - 1; j >= 0; j--) {
        comp[j] = (bin[j] == 0) ? 1 : 0;
        printf("%d", comp[j]);
    }
    printf("\nTwo's Complement: ");
    for (int j = i - 1; j >= 0; j--) {
        if (comp[j] == 0 && carry == 1) {
            comp[j] = 1;
            carry = 0;
        } else if (comp[j] == 1 && carry == 1) {
            comp[j] = 0;
        }
        printf("%d", comp[j]);
    }
    printf("\n");
    return 0;
}
\end{lstlisting}

\newpage

% Program 39: Binary to Octal
\subsection{Convert Binary to Octal Number System}
This program converts a binary number to its octal equivalent using a for loop for intermediate decimal conversion.

\begin{lstlisting}[caption={Convert Binary to Octal Number System}]
#include <stdio.h>
#include <math.h>
int main() {
    int bin, oct = 0, dec = 0, i = 0;
    printf("Enter a binary number: ");
    scanf("%d", &bin);
    while (bin != 0) {
        dec += (bin % 10) * pow(2, i);
        i++;
        bin /= 10;
    }
    i = 1;
    while (dec != 0) {
        oct += (dec % 8) * i;
        dec /= 8;
        i *= 10;
    }
    printf("Octal number: %d\n", oct);
    return 0;
}
\end{lstlisting}

\newpage

% Program 40: Binary to Decimal
\subsection{Convert Binary to Decimal Number System}
This program converts a binary number to its decimal equivalent using a for loop.

\begin{lstlisting}[caption={Convert Binary to Decimal Number System}]
#include <stdio.h>
#include <math.h>
int main() {
    int bin, dec = 0, i = 0;
    printf("Enter a binary number: ");
    scanf("%d", &bin);
    while (bin != 0) {
        dec += (bin % 10) * pow(2, i);
        i++;
        bin /= 10;
    }
    printf("Decimal number: %d\n", dec);
    return 0;
}
\end{lstlisting}

\newpage

% Program 41: Binary to Hexadecimal
\subsection{Convert Binary to Hexadecimal Number System}
This program converts a binary number to its hexadecimal equivalent using a for loop for intermediate decimal conversion.

\begin{lstlisting}[caption={Convert Binary to Hexadecimal Number System}]
#include <stdio.h>
#include <math.h>
int main() {
    int bin, dec = 0, i = 0, hex[100], j = 0;
    char hexChar[] = "0123456789ABCDEF";
    printf("Enter a binary number: ");
    scanf("%d", &bin);
    while (bin != 0) {
        dec += (bin % 10) * pow(2, i);
        i++;
        bin /= 10;
    }
    while (dec != 0) {
        hex[j] = dec % 16;
        dec /= 16;
        j++;
    }
    printf("Hexadecimal number: ");
    for (int k = j - 1; k >= 0; k--)
        printf("%c", hexChar[hex[k]]);
    printf("\n");
    return 0;
}
\end{lstlisting}

\newpage

% Program 42: Octal to Binary
\subsection{Convert Octal to Binary Number System}
This program converts an octal number to its binary equivalent using a for loop for intermediate decimal conversion.

\begin{lstlisting}[caption={Convert Octal to Binary Number System}]
#include <stdio.h>
#include <math.h>
int main() {
    int oct, bin = 0, i = 1, remainder;
    printf("Enter an octal number: ");
    scanf("%d", &oct);
    while (oct != 0) {
        remainder = oct % 10;
        bin += remainder * i;
        i *= 8;
        oct /= 10;
    }
    printf("Binary number: ");
    for (int j = 31; j >= 0; j--) {
        int bit = (bin >> j) & 1;
        printf("%d", bit);
    }
    printf("\n");
    return 0;
}
\end{lstlisting}

\newpage

% Program 43: Octal to Decimal
\subsection{Convert Octal to Decimal Number System}
This program converts an octal number to its decimal equivalent using a for loop.

\begin{lstlisting}[caption={Convert Octal to Decimal Number System}]
#include <stdio.h>
#include <math.h>
int main() {
    int oct, dec = 0, i = 0;
    printf("Enter an octal number: ");
    scanf("%d", &oct);
    while (oct != 0) {
        dec += (oct % 10) * pow(8, i);
        i++;
        oct /= 10;
    }
    printf("Decimal number: %d\n", dec);
    return 0;
}
\end{lstlisting}

\newpage

% Program 44: Octal to Hexadecimal
\subsection{Convert Octal to Hexadecimal Number System}
This program converts an octal number to its hexadecimal equivalent using a for loop for intermediate decimal conversion.

\begin{lstlisting}[caption={Convert Octal to Hexadecimal Number System}]
#include <stdio.h>
#include <math.h>
int main() {
    int oct, dec = 0, i = 0, hex[100], j = 0;
    char hexChar[] = "0123456789ABCDEF";
    printf("Enter an octal number: ");
    scanf("%d", &oct);
    while (oct != 0) {
        dec += (oct % 10) * pow(8, i);
        i++;
        oct /= 10;
    }
    while (dec != 0) {
        hex[j] = dec % 16;
        dec /= 16;
        j++;
    }
    printf("Hexadecimal number: ");
    for (int k = j - 1; k >= 0; k--)
        printf("%c", hexChar[hex[k]]);
    printf("\n");
    return 0;
}
\end{lstlisting}

\newpage

% Program 45: Decimal to Binary
\subsection{Convert Decimal to Binary Number System}
This program converts a decimal number to its binary equivalent using a for loop.

\begin{lstlisting}[caption={Convert Decimal to Binary Number System}]
#include <stdio.h>
int main() {
    int dec, bin[32], i = 0;
    printf("Enter a decimal number: ");
    scanf("%d", &dec);
    while (dec > 0) {
        bin[i] = dec % 2;
        dec /= 2;
        i++;
    }
    printf("Binary number: ");
    for (int j = i - 1; j >= 0; j--)
        printf("%d", bin[j]);
    printf("\n");
    return 0;
}
\end{lstlisting}

\newpage

% Program 46: Decimal to Octal
\subsection{Convert Decimal to Octal Number System}
This program converts a decimal number to its octal equivalent using a for loop.

\begin{lstlisting}[caption={Convert Decimal to Octal Number System}]
#include <stdio.h>
int main() {
    int dec, oct[32], i = 0;
    printf("Enter a decimal number: ");
    scanf("%d", &dec);
    while (dec > 0) {
        oct[i] = dec % 8;
        dec /= 8;
        i++;
    }
    printf("Octal number: ");
    for (int j = i - 1; j >= 0; j--)
        printf("%d", oct[j]);
    printf("\n");
    return 0;
}
\end{lstlisting}

\newpage

% Program 47: Decimal to Hexadecimal
\subsection{Convert Decimal to Hexadecimal Number System}
This program converts a decimal number to its hexadecimal equivalent using a for loop.

\begin{lstlisting}[caption={Convert Decimal to Hexadecimal Number System}]
#include <stdio.h>
int main() {
    int dec, hex[32], i = 0;
    char hexChar[] = "0123456789ABCDEF";
    printf("Enter a decimal number: ");
    scanf("%d", &dec);
    while (dec > 0) {
        hex[i] = dec % 16;
        dec /= 16;
        i++;
    }
    printf("Hexadecimal number: ");
    for (int j = i - 1; j >= 0; j--)
        printf("%c", hexChar[hex[j]]);
    printf("\n");
    return 0;
}
\end{lstlisting}

\newpage

% Program 48: Hexadecimal to Binary
\subsection{Convert Hexadecimal to Binary Number System}
This program converts a hexadecimal number to its binary equivalent using a for loop.

\begin{lstlisting}[caption={Convert Hexadecimal to Binary Number System}]
#include <stdio.h>
#include <string.h>
int main() {
    char hex[100];
    int bin[1000], i = 0, j, k = 0;
    printf("Enter a hexadecimal number: ");
    scanf("%s", hex);
    for (i = 0; hex[i] != '\0'; i++) {
        int value;
        if (hex[i] >= '0' && hex[i] <= '9')
            value = hex[i] - '0';
        else
            value = hex[i] - 'A' + 10;
        for (j = 3; j >= 0; j--) {
            bin[k] = (value >> j) & 1;
            k++;
        }
    }
    printf("Binary number: ");
    for (j = 0; j < k; j++)
        printf("%d", bin[j]);
    printf("\n");
    return 0;
}
\end{lstlisting}

\newpage

% Program 49: Hexadecimal to Octal
\subsection{Convert Hexadecimal to Octal Number System}
This program converts a hexadecimal number to its octal equivalent using a for loop for intermediate decimal conversion.

\begin{lstlisting}[caption={Convert Hexadecimal to Octal Number System}]
#include <stdio.h>
#include <string.h>
int main() {
    char hex[100];
    int dec = 0, i = 0, oct[100], j = 0;
    printf("Enter a hexadecimal number: ");
    scanf("%s", hex);
    for (i = 0; hex[i] != '\0'; i++) {
        if (hex[i] >= '0' && hex[i] <= '9')
            dec = dec * 16 + (hex[i] - '0');
        else
            dec = dec * 16 + (hex[i] - 'A' + 10);
    }
    while (dec > 0) {
        oct[j] = dec % 8;
        dec /= 8;
        j++;
    }
    printf("Octal number: ");
    for (i = j - 1; i >= 0; i--)
        printf("%d", oct[i]);
    printf("\n");
    return 0;
}
\end{lstlisting}

\newpage

% Program 50: Hexadecimal to Decimal
\subsection{Convert Hexadecimal to Decimal Number System}
This program converts a hexadecimal number to its decimal equivalent using a for loop.

\begin{lstlisting}[caption={Convert Hexadecimal to Decimal Number System}]
#include <stdio.h>
#include <string.h>
int main() {
    char hex[100];
    int dec = 0, i = 0;
    printf("Enter a hexadecimal number: ");
    scanf("%s", hex);
    for (i = 0; hex[i] != '\0'; i++) {
        if (hex[i] >= '0' && hex[i] <= '9')
            dec = dec * 16 + (hex[i] - '0');
        else
            dec = dec * 16 + (hex[i] - 'A' + 10);
    }
    printf("Decimal number: %d\n", dec);
    return 0;
}
\end{lstlisting}

\newpage

% Program 51: Print Pascal Triangle
\subsection{Print Pascal Triangle Up to n Rows}
This program prints Pascal's triangle up to a user-specified number of rows using a for loop.

\begin{lstlisting}[caption={Print Pascal Triangle Up to n Rows}]
#include <stdio.h>
int main() {
    int n, coef = 1;
    printf("Enter number of rows: ");
    scanf("%d", &n);
    for (int i = 0; i < n; i++) {
        for (int space = 1; space <= n - i; space++)
            printf("  ");
        for (int j = 0; j <= i; j++) {
            if (j == 0 || i == j)
                coef = 1;
            else
                coef = coef * (i - j + 1) / j;
            printf("%4d", coef);
        }
        printf("\n");
    }
    return 0;
}
\end{lstlisting}

\newpage

% Starting programming exercises section
\section{Star Pattern Programming Exercises}

% Subsection for each program
\subsection{Square Star Pattern}
This program prints a 5x5 square star pattern.
\begin{lstlisting}[caption={Square Star Pattern}]
#include <stdio.h>
int main() {
    int n;
    printf("Enter Number of Rows: ");
    scanf("%d",&n);
    for (int i = 0; i < n; i++) {
        for (int j = 0; j < n; j++) {
            printf("*");
        }
        printf("\n");
    }
    return 0;
}
\end{lstlisting}
\clearpage

\subsection{Hollow Square Star Pattern}
This program prints a 5x5 hollow square star pattern.
\begin{lstlisting}[caption={Hollow Square Star Pattern}]
#include <stdio.h>
int main() {
    int n;
    printf("Enter Number of Rows: ");
    scanf("%d",&n);
    for (int i = 0; i < n; i++) {
        for (int j = 0; j < n; j++) {
            if (i == 0 || i == n-1 || j == 0 || j == n-1)
                printf("*");
            else
                printf(" ");
        }
        printf("\n");
    }
    return 0;
}
\end{lstlisting}
\clearpage

\subsection{Hollow Square Star Pattern with Diagonal}
This program prints a 5x5 hollow square star pattern with diagonals.
\begin{lstlisting}[caption={Hollow Square Star Pattern with Diagonal}]
#include <stdio.h>
int main() {
    int n;
    printf("Enter Number of Rows: ");
    scanf("%d",&n);
    for (int i = 0; i < n; i++) {
        for (int j = 0; j < n; j++) {
            if (i == 0 || i == n-1 || j == 0 || j == n-1 || i == j || i + j == n-1)
                printf("*");
            else
                printf(" ");
        }
        printf("\n");
    }
    return 0;
}
\end{lstlisting}
\clearpage

\subsection{Rhombus Star Pattern}
This program prints a rhombus star pattern with 5 rows.
\begin{lstlisting}[caption={Rhombus Star Pattern}]
#include <stdio.h>
int main() {
    int n;
    printf("Enter Number of Rows: ");
    scanf("%d",&n);
    for (int i = 0; i < n; i++) {
        for (int j = 0; j < n - i - 1; j++) printf(" ");
        for (int j = 0; j < n; j++) printf("*");
        printf("\n");
    }
    return 0;
}
\end{lstlisting}
\clearpage

\subsection{Hollow Rhombus Star Pattern}
This program prints a hollow rhombus star pattern with 5 rows.
\begin{lstlisting}[caption={Hollow Rhombus Star Pattern}]
#include <stdio.h>
int main() {
    int n;
    printf("Enter Number of Rows: ");
    scanf("%d",&n);
    for (int i = 0; i < n; i++) {
        for (int j = 0; j < n - i - 1; j++) printf(" ");
        for (int j = 0; j < n; j++) {
            if (i == 0 || i == n-1 || j == 0 || j == n-1)
                printf("*");
            else
                printf(" ");
        }
        printf("\n");
    }
    return 0;
}
\end{lstlisting}
\clearpage

\subsection{Mirrored Rhombus Star Pattern}
This program prints a mirrored rhombus star pattern with 5 rows.
\begin{lstlisting}[caption={Mirrored Rhombus Star Pattern}]
#include <stdio.h>
int main() {
    int n;
    printf("Enter Number of Rows: ");
    scanf("%d",&n);
    for (int i = 0; i < n; i++) {
        for (int j = 0; j < i; j++) printf(" ");
        for (int j = 0; j < n; j++) printf("*");
        printf("\n");
    }
    return 0;
}
\end{lstlisting}
\clearpage

\subsection{Hollow Mirrored Rhombus Star Pattern}
This program prints a hollow mirrored rhombus star pattern with 5 rows.
\begin{lstlisting}[caption={Hollow Mirrored Rhombus Star Pattern}]
#include <stdio.h>
int main() {
    int n;
    printf("Enter Number of Rows: ");
    scanf("%d",&n);
    for (int i = 0; i < n; i++) {
        for (int j = 0; j < i; j++) printf(" ");
        for (int j = 0; j < n; j++) {
            if (i == 0 || i == n-1 || j == 0 || j == n-1)
                printf("*");
            else
                printf(" ");
        }
        printf("\n");
    }
    return 0;
}
\end{lstlisting}
\clearpage

\subsection{Right Triangle Star Pattern}
This program prints a right triangle star pattern with 5 rows.
\begin{lstlisting}[caption={Right Triangle Star Pattern}]
#include <stdio.h>
int main() {
    int n;
    printf("Enter Number of Rows: ");
    scanf("%d",&n);
    for (int i = 0; i < n; i++) {
        for (int j = 0; j <= i; j++) {
            printf("*");
        }
        printf("\n");
    }
    return 0;
}
\end{lstlisting}
\clearpage

\subsection{Hollow Right Triangle Star Pattern}
This program prints a hollow right triangle star pattern with 5 rows.
\begin{lstlisting}[caption={Hollow Right Triangle Star Pattern}]
#include <stdio.h>
int main() {
    int n;
    printf("Enter Number of Rows: ");
    scanf("%d",&n);
    for (int i = 0; i < n; i++) {
        for (int j = 0; j <= i; j++) {
            if (j == 0 || j == i || i == n-1)
                printf("*");
            else
                printf(" ");
        }
        printf("\n");
    }
    return 0;
}
\end{lstlisting}
\clearpage

\subsection{Mirrored Right Triangle Star Pattern}
This program prints a mirrored right triangle star pattern with 5 rows.
\begin{lstlisting}[caption={Mirrored Right Triangle Star Pattern}]
#include <stdio.h>
int main() {
    int n;
    printf("Enter Number of Rows: ");
    scanf("%d",&n);
    for (int i = 0; i < n; i++) {
        for (int j = 0; j < n - i - 1; j++) printf(" ");
        for (int j = 0; j <= i; j++) printf("*");
        printf("\n");
    }
    return 0;
}
\end{lstlisting}
\clearpage

\subsection{Hollow Mirrored Right Triangle Star Pattern}
This program prints a hollow mirrored right triangle star pattern with 5 rows.
\begin{lstlisting}[caption={Hollow Mirrored Right Triangle Star Pattern}]
#include <stdio.h>
int main() {
    int n;
    printf("Enter Number of Rows: ");
    scanf("%d",&n);
    for (int i = 0; i < n; i++) {
        for (int j = 0; j < n - i - 1; j++) printf(" ");
        for (int j = 0; j <= i; j++) {
            if (j == 0 || j == i || i == n-1)
                printf("*");
            else
                printf(" ");
        }
        printf("\n");
    }
    return 0;
}
\end{lstlisting}
\clearpage

\subsection{Inverted Right Triangle Star Pattern}
This program prints an inverted right triangle star pattern with 5 rows.
\begin{lstlisting}[caption={Inverted Right Triangle Star Pattern}]
#include <stdio.h>
int main() {
    int n;
    printf("Enter Number of Rows: ");
    scanf("%d",&n);
    for (int i = 0; i < n; i++) {
        for (int j = 0; j < n - i; j++) {
            printf("*");
        }
        printf("\n");
    }
    return 0;
}
\end{lstlisting}
\clearpage

\subsection{Hollow Inverted Right Triangle Star Pattern}
This program prints a hollow inverted right triangle star pattern with 5 rows.
\begin{lstlisting}[caption={Hollow Inverted Right Triangle Star Pattern}]
#include <stdio.h>
int main() {
    int n;
    printf("Enter Number of Rows: ");
    scanf("%d",&n);
    for (int i = 0; i < n; i++) {
        for (int j = 0; j < n - i; j++) {
            if (j == 0 || j == n - i - 1 || i == 0)
                printf("*");
            else
                printf(" ");
        }
        printf("\n");
    }
    return 0;
}
\end{lstlisting}
\clearpage

\subsection{Inverted Mirrored Right Triangle Star Pattern}
This program prints an inverted mirrored right triangle star pattern with 5 rows.
\begin{lstlisting}[caption={Inverted Mirrored Right Triangle Star Pattern}]
#include <stdio.h>
int main() {
    int n;
    printf("Enter Number of Rows: ");
    scanf("%d",&n);
    for (int i = 0; i < n; i++) {
        for (int j = 0; j < i; j++) printf(" ");
        for (int j = 0; j < n - i; j++) printf("*");
        printf("\n");
    }
    return 0;
}
\end{lstlisting}
\clearpage

\subsection{Hollow Inverted Mirrored Right Triangle Star Pattern}
This program prints a hollow inverted mirrored right triangle star pattern with 5 rows.
\begin{lstlisting}[caption={Hollow Inverted Mirrored Right Triangle Star Pattern}]
#include <stdio.h>
int main() {
    int n;
    printf("Enter Number of Rows: ");
    scanf("%d",&n);
    for (int i = 0; i < n; i++) {
        for (int j = 0; j < i; j++) printf(" ");
        for (int j = 0; j < n - i; j++) {
            if (j == 0 || j == n - i - 1 || i == 0)
                printf("*");
            else
                printf(" ");
        }
        printf("\n");
    }
    return 0;
}
\end{lstlisting}
\clearpage

\subsection{Pyramid Star Pattern}
This program prints a pyramid star pattern with 5 rows.
\begin{lstlisting}[caption={Pyramid Star Pattern}]
#include <stdio.h>
int main() {
    int n;
    printf("Enter Number of Rows: ");
    scanf("%d",&n);
    for (int i = 0; i < n; i++) {
        for (int j = 0; j < n - i - 1; j++) printf(" ");
        for (int j = 0; j < 2 * i + 1; j++) printf("*");
        printf("\n");
    }
    return 0;
}
\end{lstlisting}
\clearpage

\subsection{Hollow Pyramid Star Pattern}
This program prints a hollow pyramid star pattern with 5 rows.
\begin{lstlisting}[caption={Hollow Pyramid Star Pattern}]
#include <stdio.h>
int main() {
    int n;
    printf("Enter Number of Rows: ");
    scanf("%d",&n);
    for (int i = 0; i < n; i++) {
        for (int j = 0; j < n - i - 1; j++) printf(" ");
        for (int j = 0; j < 2 * i + 1; j++) {
            if (j == 0 || j == 2 * i || i == n - 1)
                printf("*");
            else
                printf(" ");
        }
        printf("\n");
    }
    return 0;
}
\end{lstlisting}
\clearpage

\subsection{Inverted Pyramid Star Pattern}
This program prints an inverted pyramid star pattern with 5 rows.
\begin{lstlisting}[caption={Inverted Pyramid Star Pattern}]
#include <stdio.h>
int main() {
    int n;
    printf("Enter Number of Rows: ");
    scanf("%d",&n);
    for (int i = 0; i < n; i++) {
        for (int j = 0; j < i; j++) printf(" ");
        for (int j = 0; j < 2 * (n - i) - 1; j++) printf("*");
        printf("\n");
    }
    return 0;
}
\end{lstlisting}
\clearpage

\subsection{Hollow Inverted Pyramid Star Pattern}
This program prints a hollow inverted pyramid star pattern with 5 rows.
\begin{lstlisting}[caption={Hollow Inverted Pyramid Star Pattern}]
#include <stdio.h>
int main() {
    int n;
    printf("Enter Number of Rows: ");
    scanf("%d",&n);
    for (int i = 0; i < n; i++) {
        for (int j = 0; j < i; j++) printf(" ");
        for (int j = 0; j < 2 * (n - i) - 1; j++) {
            if (j == 0 || j == 2 * (n - i) - 2 || i == 0)
                printf("*");
            else
                printf(" ");
        }
        printf("\n");
    }
    return 0;
}
\end{lstlisting}
\clearpage

\subsection{Half Diamond Star Pattern}
This program prints a half diamond star pattern with 5 rows increasing and 4 rows decreasing.
\begin{lstlisting}[caption={Half Diamond Star Pattern}]
#include <stdio.h>
int main() {
    int n;
    printf("Enter Number of Rows: ");
    scanf("%d",&n);
    for (int i = 0; i < n; i++) {
        for (int j = 0; j <= i; j++) printf("*");
        printf("\n");
    }
    for (int i = 0; i < n - 1; i++) {
        for (int j = 0; j < n - i - 1; j++) printf("*");
        printf("\n");
    }
    return 0;
}
\end{lstlisting}
\clearpage

\subsection{Mirrored Half Diamond Star Pattern}
This program prints a mirrored half diamond star pattern with 5 rows increasing and 4 rows decreasing.
\begin{lstlisting}[caption={Mirrored Half Diamond Star Pattern}]
#include <stdio.h>
int main() {
    int n;
    printf("Enter Number of Rows: ");
    scanf("%d",&n);
    for (int i = 0; i < n; i++) {
        for (int j = 0; j < n - i - 1; j++) printf(" ");
        for (int j = 0; j <= i; j++) printf("*");
        printf("\n");
    }
    for (int i = 0; i < n - 1; i++) {
        for (int j = 0; j <= i; j++) printf(" ");
        for (int j = 0; j < n - i - 1; j++) printf("*");
        printf("\n");
    }
    return 0;
}
\end{lstlisting}
\clearpage

\subsection{Diamond Star Pattern}
This program prints a diamond star pattern with 5 rows on each half.
\begin{lstlisting}[caption={Diamond Star Pattern}]
#include <stdio.h>
int main() {
    int n;
    printf("Enter Number of Rows: ");
    scanf("%d",&n);
    for (int i = 0; i < n; i++) {
        for (int j = 0; j < n - i - 1; j++) printf(" ");
        for (int j = 0; j < 2 * i + 1; j++) printf("*");
        printf("\n");
    }
    for (int i = 0; i < n - 1; i++) {
        for (int j = 0; j <= i; j++) printf(" ");
        for (int j = 0; j < 2 * (n - i - 1) - 1; j++) printf("*");
        printf("\n");
    }
    return 0;
}
\end{lstlisting}
\clearpage

\subsection{Hollow Diamond Star Pattern}
This program prints a hollow diamond star pattern with 5 rows on each half.
\begin{lstlisting}[caption={Hollow Diamond Star Pattern}]
#include <stdio.h>
int main() {
    int n;
    printf("Enter Number of Rows: ");
    scanf("%d",&n);
    for (int i = 0; i < n; i++) {
        for (int j = 0; j < n - i - 1; j++) printf(" ");
        for (int j = 0; j < 2 * i + 1; j++) {
            if (j == 0 || j == 2 * i)
                printf("*");
            else
                printf(" ");
        }
        printf("\n");
    }
    for (int i = 0; i < n - 1; i++) {
        for (int j = 0; j <= i; j++) printf(" ");
        for (int j = 0; j < 2 * (n - i - 1) - 1; j++) {
            if (j == 0 || j == 2 * (n - i - 1) - 2)
                printf("*");
            else
                printf(" ");
        }
        printf("\n");
    }
    return 0;
}
\end{lstlisting}
\clearpage

\subsection{Right Arrow Star Pattern}
This program prints a right arrow star pattern with 5 rows on each half.
\begin{lstlisting}[caption={Right Arrow Star Pattern}]
#include <stdio.h>
int main() {
    int n;
    printf("Enter Number of Rows: ");
    scanf("%d",&n);
    for (int i = 0; i < n; i++) {
        for (int j = 0; j < i; j++) printf(" ");
        for (int j = 0; j < n - i; j++) printf("*");
        printf("\n");
    }
    for (int i = 1; i < n; i++) {
        for (int j = 0; j < n - i - 1; j++) printf(" ");
        for (int j = 0; j <= i; j++) printf("*");
        printf("\n");
    }
    return 0;
}
\end{lstlisting}
\clearpage

\subsection{Left Arrow Star Pattern}
This program prints a left arrow star pattern with 5 rows on each half.
\begin{lstlisting}[caption={Left Arrow Star Pattern}]
#include <stdio.h>
int main() {
    int n;
    printf("Enter Number of Rows: ");
    scanf("%d",&n);
    for (int i = 0; i < n; i++) {
        for (int j = 0; j < n - i - 1; j++) printf(" ");
        for (int j = 0; j <= i; j++) printf("*");
        printf("\n");
    }
    for (int i = 1; i < n; i++) {
        for (int j = 0; j < i; j++) printf(" ");
        for (int j = 0; j < n - i; j++) printf("*");
        printf("\n");
    }
    return 0;
}
\end{lstlisting}
\clearpage

\subsection{Plus Star Pattern}
This program prints a plus star pattern with 5 rows on each arm.
\begin{lstlisting}[caption={Plus Star Pattern}]
#include <stdio.h>
int main() {
    int n;
    printf("Enter Number of Rows: ");
    scanf("%d",&n);
    for (int i = 0; i < 2 * n - 1; i++) {
        for (int j = 0; j < 2 * n - 1; j++) {
            if (i == n - 1 || j == n - 1)
                printf("+");
            else
                printf(" ");
        }
        printf("\n");
    }
    return 0;
}
\end{lstlisting}
\clearpage

\subsection{X Star Pattern}
This program prints an X star pattern with 5 rows on each half.
\begin{lstlisting}[caption={X Star Pattern}]
#include <stdio.h>
int main() {
    int n;
    printf("Enter Number of Rows: ");
    scanf("%d",&n);
    for (int i = 0; i < n; i++) {
        for (int j = 0; j < n; j++) {
            if (i == j || i + j == n - 1)
                printf("*");
            else
                printf(" ");
        }
        printf("\n");
    }
    return 0;
}
\end{lstlisting}
\clearpage

\subsection{Eight Star Pattern}
This program prints an eight star pattern with 9 rows.
\begin{lstlisting}[caption={Eight Star Pattern}]
#include <stdio.h>
int main() {
    int n;
    printf("Enter Number of Rows: ");
    scanf("%d",&n);
    for (int i = 0; i < 2 * n - 1; i++) {
        for (int j = 0; j < n; j++) {
            if (i == 0 || i == n - 1 || i == 2 * n - 2 || j == 0 || j == n - 1)
                printf("*");
            else
                printf(" ");
        }
        printf("\n");
    }
    return 0;
}
\end{lstlisting}
\clearpage

\subsection{Heart Star Pattern}
This program prints a heart star pattern with 13 rows.
\begin{lstlisting}[caption={Heart Star Pattern}]
#include <stdio.h>
int main() {
    int n;
    printf("Enter Number of Rows: ");
    scanf("%d",&n);
    for (int i = n / 2; i <= n; i += 2) {
        for (int j = 1; j < n - i; j += 2) printf(" ");
        for (int j = 1; j <= i; j++) printf("*");
        for (int j = 1; j <= n - i; j++) printf(" ");
        for (int j = 1; j <= i; j++) printf("*");
        printf("\n");
    }
    for (int i = n; i >= 1; i--) {
        for (int j = 1; j <= n - i; j++) printf(" ");
        for (int j = 1; j <= 2 * i - 1; j++) printf("*");
        printf("\n");
    }
    return 0;
}
\end{lstlisting}
\clearpage

\subsection{Heart Star Pattern with Name}
This program prints a heart star pattern with the name "Codeforwin" in the center.
\begin{lstlisting}[caption={Heart Star Pattern with Name}]
#include <stdio.h>
#include <string.h>
int main() {
    int n;
    printf("Enter Number of Rows: ");
    scanf("%d",&n);
    char name[] = "Codeforwin";
    int nameLen = strlen(name);
    int heartRows = n + n / 2;
    int nameRow = n / 2 + 1;
    for (int i = n / 2; i <= n; i += 2) {
        for (int j = 1; j < n - i; j += 2) printf(" ");
        for (int j = 1; j <= i; j++) printf("*");
        for (int j = 1; j <= n - i; j++) printf(" ");
        for (int j = 1; j <= i; j++) printf("*");
        printf("\n");
    }
    for (int i = n; i >= 1; i--) {
        for (int j = 1; j <= n - i; j++) printf(" ");
        if (i == nameRow) {
            printf("%*s", 2 * i - 1, name);
        } else {
            for (int j = 1; j <= 2 * i - 1; j++) printf("*");
        }
        printf("\n");
    }
    return 0;
}
\end{lstlisting}
\clearpage

% Starting programming exercises section
\section{Number Pattern Programming Exercises}

% Subsection for each program
\subsection{Number Pattern 1}
This program prints a square pattern of 1s with user-specified size.
\begin{lstlisting}[caption={Number Pattern 1}]
#include <stdio.h>
int main() {
    int n;
    printf("Enter size: ");
    scanf("%d", &n);
    for (int i = 0; i < n; i++) {
        for (int j = 0; j < n; j++) {
            printf("1");
        }
        printf("\n");
    }
    return 0;
}
\end{lstlisting}
\textbf{Sample Output (for size=5):}
\begin{verbatim}
11111
11111
11111
11111
11111
\end{verbatim}
\clearpage

\subsection{Number Pattern 2}
This program prints a pattern alternating 1s and 0s with user-specified size.
\begin{lstlisting}[caption={Number Pattern 2}]
#include <stdio.h>
int main() {
    int n;
    printf("Enter size: ");
    scanf("%d", &n);
    for (int i = 0; i < n; i++) {
        for (int j = 0; j < n; j++) {
            printf("%d", i % 2);
        }
        printf("\n");
    }
    return 0;
}
\end{lstlisting}
\textbf{Sample Output (for size=5):}
\begin{verbatim}
11111
00000
11111
00000
11111
\end{verbatim}
\clearpage

\subsection{Number Pattern 3}
This program prints a pattern of 0s and 1s in a checkerboard style with user-specified size.
\begin{lstlisting}[caption={Number Pattern 3}]
#include <stdio.h>
int main() {
    int n;
    printf("Enter size: ");
    scanf("%d", &n);
    for (int i = 0; i < n; i++) {
        for (int j = 0; j < n; j++) {
            printf("%d", (i + j) % 2);
        }
        printf("\n");
    }
    return 0;
}
\end{lstlisting}
\textbf{Sample Output (for size=5):}
\begin{verbatim}
01010
10101
01010
10101
01010
\end{verbatim}
\clearpage

\subsection{Number Pattern 4}
This program prints a hollow square pattern of 1s with user-specified size.
\begin{lstlisting}[caption={Number Pattern 4}]
#include <stdio.h>
int main() {
    int n;
    printf("Enter size: ");
    scanf("%d", &n);
    for (int i = 0; i < n; i++) {
        for (int j = 0; j < n; j++) {
            if (i == 0 || i == n-1 || j == 0 || j == n-1)
                printf("1");
            else
                printf("0");
        }
        printf("\n");
    }
    return 0;
}
\end{lstlisting}
\textbf{Sample Output (for size=5):}
\begin{verbatim}
11111
10001
10001
10001
11111
\end{verbatim}
\clearpage

\subsection{Number Pattern 5}
This program prints a square pattern with a 0 cross in the middle with user-specified size.
\begin{lstlisting}[caption={Number Pattern 5}]
#include <stdio.h>
int main() {
    int n;
    printf("Enter size: ");
    scanf("%d", &n);
    for (int i = 0; i < n; i++) {
        for (int j = 0; j < n; j++) {
            if (i == n/2 || j == n/2)
                printf("0");
            else
                printf("1");
        }
        printf("\n");
    }
    return 0;
}
\end{lstlisting}
\textbf{Sample Output (for size=5):}
\begin{verbatim}
11111
11111
00000
11111
11111
\end{verbatim}
\clearpage

\subsection{Number Pattern 6}
This program prints a checkerboard pattern of 1s and 0s with user-specified size.
\begin{lstlisting}[caption={Number Pattern 6}]
#include <stdio.h>
int main() {
    int n;
    printf("Enter size: ");
    scanf("%d", &n);
    for (int i = 0; i < n; i++) {
        for (int j = 0; j < n; j++) {
            printf("%d", (i + j + 1) % 2);
        }
        printf("\n");
    }
    return 0;
}
\end{lstlisting}
\textbf{Sample Output (for size=5):}
\begin{verbatim}
10101
01010
10101
01010
10101
\end{verbatim}
\clearpage

\subsection{Number Pattern 7}
This program prints a pattern with 0s in the middle and 1s on the edges with user-specified size.
\begin{lstlisting}[caption={Number Pattern 7}]
#include <stdio.h>
int main() {
    int n;
    printf("Enter size: ");
    scanf("%d", &n);
    for (int i = 0; i < n; i++) {
        for (int j = 0; j < n; j++) {
            if (i == n/2 || j == n/2)
                printf("0");
            else
                printf("1");
        }
        printf("\n");
    }
    return 0;
}
\end{lstlisting}
\textbf{Sample Output (for size=5):}
\begin{verbatim}
11011
11011
00000
11011
11011
\end{verbatim}
\clearpage

\subsection{Number Pattern 8}
This program prints a pattern with 0s forming a cross and 1s elsewhere with user-specified size.
\begin{lstlisting}[caption={Number Pattern 8}]
#include <stdio.h>
int main() {
    int n;
    printf("Enter size: ");
    scanf("%d", &n);
    for (int i = 0; i < n; i++) {
        for (int j = 0; j < n; j++) {
            if (i == j || i + j == n-1)
                printf("0");
            else
                printf("1");
        }
        printf("\n");
    }
    return 0;
}
\end{lstlisting}
\textbf{Sample Output (for size=5):}
\begin{verbatim}
10001
01010
00100
01010
10001
\end{verbatim}
\clearpage

\subsection{Number Pattern 9}
This program prints a pattern with 0s and 1s forming a specific shape with user-specified size.
\begin{lstlisting}[caption={Number Pattern 9}]
#include <stdio.h>
int main() {
    int n;
    printf("Enter size: ");
    scanf("%d", &n);
    for (int i = 0; i < n; i++) {
        for (int j = 0; j < n; j++) {
            if (i == 0 || i == n-1 || j == 0 || j == n-1)
                printf("0");
            else
                printf("1");
        }
        printf("\n");
    }
    return 0;
}
\end{lstlisting}
\textbf{Sample Output (for size=5):}
\begin{verbatim}
00000
01110
01110
01110
00000
\end{verbatim}
\clearpage

\subsection{Number Pattern 10}
This program prints a pattern with increasing row numbers with user-specified size.
\begin{lstlisting}[caption={Number Pattern 10}]
#include <stdio.h>
int main() {
    int n;
    printf("Enter size: ");
    scanf("%d", &n);
    for (int i = 0; i < n; i++) {
        for (int j = 0; j < n; j++) {
            printf("%d", i + 1);
        }
        printf("\n");
    }
    return 0;
}
\end{lstlisting}
\textbf{Sample Output (for size=5):}
\begin{verbatim}
11111
22222
33333
44444
55555
\end{verbatim}
\clearpage

\subsection{Number Pattern 11}
This program prints a pattern repeating 12345 with user-specified size.
\begin{lstlisting}[caption={Number Pattern 11}]
#include <stdio.h>
int main() {
    int n;
    printf("Enter size: ");
    scanf("%d", &n);
    for (int i = 0; i < n; i++) {
        for (int j = 0; j < n; j++) {
            printf("%d", j + 1);
        }
        printf("\n");
    }
    return 0;
}
\end{lstlisting}
\textbf{Sample Output (for size=5):}
\begin{verbatim}
12345
12345
12345
12345
12345
\end{verbatim}
\clearpage

\subsection{Number Pattern 12}
This program prints a pattern with increasing numbers per row with user-specified size.
\begin{lstlisting}[caption={Number Pattern 12}]
#include <stdio.h>
int main() {
    int n;
    printf("Enter size: ");
    scanf("%d", &n);
    for (int i = 0; i < n; i++) {
        for (int j = 0; j < n; j++) {
            printf("%d", i + j + 1);
        }
        printf("\n");
    }
    return 0;
}
\end{lstlisting}
\textbf{Sample Output (for size=5):}
\begin{verbatim}
12345
23456
34567
45678
56789
\end{verbatim}
\clearpage

\subsection{Number Pattern 13}
This program prints a pattern with sequential numbers with user-specified size.
\begin{lstlisting}[caption={Number Pattern 13}]
#include <stdio.h>
int main() {
    int n;
    printf("Enter size: ");
    scanf("%d", &n);
    int count = 1;
    for (int i = 0; i < n; i++) {
        for (int j = 0; j < n; j++) {
            printf("%2d ", count++);
        }
        printf("\n");
    }
    return 0;
}
\end{lstlisting}
\textbf{Sample Output (for size=5):}
\begin{verbatim}
 1  2  3  4  5 
 6  7  8  9 10 
11 12 13 14 15 
16 17 18 19 20 
21 22 23 24 25 
\end{verbatim}
\clearpage

\subsection{Number Pattern 14}
This program prints a pattern with decreasing numbers per row with user-specified size.
\begin{lstlisting}[caption={Number Pattern 14}]
#include <stdio.h>
int main() {
    int n;
    printf("Enter size: ");
    scanf("%d", &n);
    for (int i = 0; i < n; i++) {
        for (int j = 0; j < n; j++) {
            printf("%d", n - i);
        }
        printf("\n");
    }
    return 0;
}
\end{lstlisting}
\textbf{Sample Output (for size=5):}
\begin{verbatim}
55555
44444
33333
22222
11111
\end{verbatim}
\clearpage

\subsection{Number Pattern 15}
This program prints a pattern with increasing end digits with user-specified size.
\begin{lstlisting}[caption={Number Pattern 15}]
#include <stdio.h>
int main() {
    int n;
    printf("Enter size: ");
    scanf("%d", &n);
    for (int i = 0; i < n; i++) {
        for (int j = 0; j < n; j++) {
            printf("%d", j + i + 1 > n ? n : j + i + 1);
        }
        printf("\n");
    }
    return 0;
}
\end{lstlisting}
\textbf{Sample Output (for size=5):}
\begin{verbatim}
12345
23455
34555
45555
55555
\end{verbatim}
\clearpage

\subsection{Number Pattern 16}
This program prints a pattern with decreasing numbers per row with user-specified size.
\begin{lstlisting}[caption={Number Pattern 16}]
#include <stdio.h>
int main() {
    int n;
    printf("Enter size: ");
    scanf("%d", &n);
    for (int i = 0; i < n; i++) {
        for (int j = 0; j < n; j++) {
            printf("%d", j + i + 1 <= n ? j + i + 1 : n - (j + i - n));
        }
        printf("\n");
    }
    return 0;
}
\end{lstlisting}
\textbf{Sample Output (for size=5):}
\begin{verbatim}
12345
23451
34521
45321
54321
\end{verbatim}
\clearpage

\subsection{Number Pattern 17}
This program prints a pattern with mirrored numbers with user-specified size.
\begin{lstlisting}[caption={Number Pattern 17}]
#include <stdio.h>
int main() {
    int n;
    printf("Enter size: ");
    scanf("%d", &n);
    for (int i = 0; i < n; i++) {
        for (int j = 0; j < n; j++) {
            printf("%d", i + 1 <= j + 1 ? i + 1 : j + 1);
        }
        printf("\n");
    }
    return 0;
}
\end{lstlisting}
\textbf{Sample Output (for size=5):}
\begin{verbatim}
12345
21234
32123
43212
54321
\end{verbatim}
\clearpage

\subsection{Number Pattern 18}
This program prints a concentric square pattern with user-specified size.
\begin{lstlisting}[caption={Number Pattern 18}]
#include <stdio.h>
int main() {
    int n;
    printf("Enter size (odd number): ");
    scanf("%d", &n);
    int matrix[n][n];
    int num = 1, top = 0, bottom = n-1, left = 0, right = n-1;
    while (num <= n * n) {
        for (int i = left; i <= right; i++) matrix[top][i] = num++;
        top++;
        for (int i = top; i <= bottom; i++) matrix[i][right] = num++;
        right--;
        for (int i = right; i >= left; i--) matrix[bottom][i] = num++;
        bottom--;
        for (int i = bottom; i >= top; i--) matrix[i][left] = num++;
        left++;
    }
    for (int i = 0; i < n; i++) {
        for (int j = 0; j < n; j++) {
            printf("%2d ", matrix[i][j]);
        }
        printf("\n");
    }
    return 0;
}
\end{lstlisting}
\textbf{Sample Output (for size=5):}
\begin{verbatim}
 1  2  3  4  5 
16 17 18 19  6 
15 24 25 20  7 
14 23 22 21  8 
13 12 11 10  9 
\end{verbatim}
\clearpage

\subsection{Number Pattern 19}
This program prints a spiral number pattern with user-specified size.
\begin{lstlisting}[caption={Number Pattern 19}]
#include <stdio.h>
int main() {
    int n;
    printf("Enter size: ");
    scanf("%d", &n);
    int matrix[n][n];
    int num = 1, top = 0, bottom = n-1, left = 0, right = n-1;
    while (num <= n * n) {
        for (int i = left; i <= right; i++) matrix[top][i] = num++;
        top++;
        for (int i = top; i <= bottom; i++) matrix[i][right] = num++;
        right--;
        for (int i = right; i >= left; i--) matrix[bottom][i] = num++;
        bottom--;
        for (int i = bottom; i >= top; i--) matrix[i][left] = num++;
        left++;
    }
    for (int i = 0; i < n; i++) {
        for (int j = 0; j < n; j++) {
            printf("%2d ", matrix[i][j]);
        }
        printf("\n");
    }
    return 0;
}
\end{lstlisting}
\textbf{Sample Output (for size=5):}
\begin{verbatim}
 1  2  3  4  5 
16 17 18 19  6 
15 24 25 20  7 
14 23 22 21  8 
13 12 11 10  9 
\end{verbatim}
\clearpage

\subsection{Number Pattern 20}
This program prints a right triangle pattern with increasing numbers per row.
\begin{lstlisting}[caption={Number Pattern 20}]
#include <stdio.h>
int main() {
    int n;
    printf("Enter size: ");
    scanf("%d", &n);
    for (int i = 0; i < n; i++) {
        for (int j = 0; j < n - i - 1; j++) printf(" ");
        for (int j = 0; j <= i; j++) printf("%d", i + 1);
        printf("\n");
    }
    return 0;
}
\end{lstlisting}
\textbf{Sample Output (for size=5):}
\begin{verbatim}
    1
   22
  333
 4444
55555
\end{verbatim}
\clearpage

\subsection{Number Pattern 21}
This program prints an inverted right triangle pattern with decreasing numbers.
\begin{lstlisting}[caption={Number Pattern 21}]
#include <stdio.h>
int main() {
    int n;
    printf("Enter size: ");
    scanf("%d", &n);
    for (int i = 0; i < n; i++) {
        for (int j = 0; j < i; j++) printf(" ");
        for (int j = 0; j < n - i; j++) printf("%d", n - i);
        printf("\n");
    }
    return 0;
}
\end{lstlisting}
\textbf{Sample Output (for size=5):}
\begin{verbatim}
55555
 4444
  333
   22
    1
\end{verbatim}
\clearpage

\subsection{Number Pattern 22}
This program prints an inverted right triangle pattern with increasing numbers.
\begin{lstlisting}[caption={Number Pattern 22}]
#include <stdio.h>
int main() {
    int n;
    printf("Enter size: ");
    scanf("%d", &n);
    for (int i = 0; i < n; i++) {
        for (int j = 0; j < i; j++) printf(" ");
        for (int j = 0; j < n - i; j++) printf("%d", i + 1);
        printf("\n");
    }
    return 0;
}
\end{lstlisting}
\textbf{Sample Output (for size=5):}
\begin{verbatim}
11111
 2222
  333
   44
    5
\end{verbatim}
\clearpage

\subsection{Number Pattern 232}
This program prints a right triangle pattern with decreasing numbers.
\begin{lstlisting}[caption={Number Pattern 23}]
#include <stdio.h>
int main() {
    int n;
    printf("Enter size: ");
    scanf("%d", &n);
    for (int i = 0; i < n; i++) {
        for (int j = 0; j < n - i - 1; j++) printf(" ");
        for (int j = 0; j <= i; j++) printf("%d", n - i);
        printf("\n");
    }
    return 0;
}
\end{lstlisting}
\textbf{Sample Output (for size=5):}
\begin{verbatim}
    5
   44
  333
 2222
11111
\end{verbatim}
\clearpage

\subsection{Number Pattern 24}
This program prints a right triangle pattern with sequential numbers.
\begin{lstlisting}[caption={Number Pattern 24}]
#include <stdio.h>
int main() {
    int n;
    printf("Enter size: ");
    scanf("%d", &n);
    for (int i = 0; i < n; i++) {
        for (int j = 0; j < n - i - 1; j++) printf(" ");
        for (int j = 1; j <= i + 1; j++) printf("%d", j);
        printf("\n");
    }
    return 0;
}
\end{lstlisting}
\textbf{Sample Output (for size=5):}
\begin{verbatim}
    1
   12
  123
 1234
12345
\end{verbatim}
\clearpage

\subsection{Number Pattern 25}
This program prints an inverted right triangle pattern with sequential numbers.
\begin{lstlisting}[caption={Number Pattern 25}]
#include <stdio.h>
int main() {
    int n;
    printf("Enter size: ");
    scanf("%d", &n);
    for (int i = 0; i < n; i++) {
        for (int j = 0; j < i; j++) printf(" ");
        for (int j = 1; j <= n - i; j++) printf("%d", j);
        printf("\n");
    }
    return 0;
}
\end{lstlisting}
\textbf{Sample Output (for size=5):}
\begin{verbatim}
12345
 1234
  123
   12
    1
\end{verbatim}
\clearpage

\subsection{Number Pattern 26}
This program prints a right triangle pattern with reverse sequential numbers.
\begin{lstlisting}[caption={Number Pattern 26}]
#include <stdio.h>
int main() {
    int n;
    printf("Enter size: ");
    scanf("%d", &n);
    for (int i = 0; i < n; i++) {
        for (int j = 0; j < n - i - 1; j++) printf(" ");
        for (int j = i + 1; j >= 1; j--) printf("%d", j);
        printf("\n");
    }
    return 0;
}
\end{lstlisting}
\textbf{Sample Output (for size=5):}
\begin{verbatim}
    1
   21
  321
 4321
54321
\end{verbatim}
\clearpage

\subsection{Number Pattern 27}
This program prints an inverted right triangle pattern with reverse sequential numbers.
\begin{lstlisting}[caption={Number Pattern 27}]
#include <stdio.h>
int main() {
    int n;
    printf("Enter size: ");
    scanf("%d", &n);
    for (int i = 0; i < n; i++) {
        for (int j = 0; j < i; j++) printf(" ");
        for (int j = n - i; j >= 1; j--) printf("%d", j);
        printf("\n");
    }
    return 0;
}
\end{lstlisting}
\textbf{Sample Output (for size=5):}
\begin{verbatim}
54321
 4321
  321
   21
    1
\end{verbatim}
\clearpage

\subsection{Number Pattern 28}
This program prints a right triangle pattern with reverse numbers starting from 5.
\begin{lstlisting}[caption={Number Pattern 28}]
#include <stdio.h>
int main() {
    int n;
    printf("Enter size: ");
    scanf("%d", &n);
    for (int i = 0; i < n; i++) {
        for (int j = 0; j < n - i - 1; j++) printf(" ");
        for (int j = n; j >= n - i; j--) printf("%d", j);
        printf("\n");
    }
    return 0;
}
\end{lstlisting}
\textbf{Sample Output (for size=5):}
\begin{verbatim}
    5
   54
  543
 5432
54321
\end{verbatim}
\clearpage

\subsection{Number Pattern 29}
This program prints an inverted right triangle pattern with reverse numbers starting from 5.
\begin{lstlisting}[caption={Number Pattern 29}]
#include <stdio.h>
int main() {
    int n;
    printf("Enter size: ");
    scanf("%d", &n);
    for (int i = 0; i < n; i++) {
        for (int j = 0; j < i; j++) printf(" ");
        for (int j = n; j >= i + 1; j--) printf("%d", j);
        printf("\n");
    }
    return 0;
}
\end{lstlisting}
\textbf{Sample Output (for size=5):}
\begin{verbatim}
54321
 5432
  543
   54
    5
\end{verbatim}
\clearpage

\subsection{Number Pattern 30}
This program prints a right triangle pattern with sequential numbers starting from 1.
\begin{lstlisting}[caption={Number Pattern 30}]
#include <stdio.h>
int main() {
    int n;
    printf("Enter size: ");
    scanf("%d", &n);
    for (int i = 0; i < n; i++) {
        for (int j = 0; j < n - i - 1; j++) printf(" ");
        for (int j = i + 1; j <= n; j++) printf("%d", j);
        printf("\n");
    }
    return 0;
}
\end{lstlisting}
\textbf{Sample Output (for size=5):}
\begin{verbatim}
    5
   45
  345
 2345
12345
\end{verbatim}
\clearpage

\subsection{Number Pattern 31}
This program prints an inverted right triangle pattern with sequential numbers starting from 1.
\begin{lstlisting}[caption={Number Pattern 31}]
#include <stdio.h>
int main() {
    int n;
    printf("Enter size: ");
    scanf("%d", &n);
    for (int i = 0; i < n; i++) {
        for (int j = 0; j < i; j++) printf(" ");
        for (int j = i + 1; j <= n; j++) printf("%d", j);
        printf("\n");
    }
    return 0;
}
\end{lstlisting}
\textbf{Sample Output (for size=5):}
\begin{verbatim}
12345
 2345
  345
   45
    5
\end{verbatim}
\clearpage

\subsection{Number Pattern 32}
This program prints a right triangle pattern with odd sequential numbers.
\begin{lstlisting}[caption={Number Pattern 32}]
#include <stdio.h>
int main() {
    int n;
    printf("Enter size: ");
    scanf("%d", &n);
    for (int i = 0; i < n; i++) {
        for (int j = 0; j < n - i - 1; j++) printf(" ");
        for (int j = 0; j <= i; j++) printf("%d", 2 * (i + j) + 1);
        printf("\n");
    }
    return 0;
}
\end{lstlisting}
\textbf{Sample Output (for size=5):}
\begin{verbatim}
    1
   13
  135
 1357
13579
\end{verbatim}
\clearpage

\subsection{Number Pattern 33}
This program prints an inverted right triangle pattern with odd sequential numbers.
\begin{lstlisting}[caption={Number Pattern 33}]
#include <stdio.h>
int main() {
    int n;
    printf("Enter size: ");
    scanf("%d", &n);
    for (int i = 0; i < n; i++) {
        for (int j = 0; j < i; j++) printf(" ");
        for (int j = 0; j < n - i; j++) printf("%d", 2 * (n - i - j) - 1);
        printf("\n");
    }
    return 0;
}
\end{lstlisting}
\textbf{Sample Output (for size=5):}
\begin{verbatim}
98765
 8765
  765
   65
    5
\end{verbatim}
\clearpage

\subsection{Number Pattern 34}
This program prints a right triangle pattern alternating 1s and 0s.
\begin{lstlisting}[caption={Number Pattern 34}]
#include <stdio.h>
int main() {
    int n;
    printf("Enter size: ");
    scanf("%d", &n);
    for (int i = 0; i < n; i++) {
        for (int j = 0; j < n - i - 1; j++) printf(" ");
        for (int j = 0; j <= i; j++) printf("%d", (i + j + 1) % 2);
        printf("\n");
    }
    return 0;
}
\end{lstlisting}
\textbf{Sample Output (for size=5):}
\begin{verbatim}
    1
   10
  101
 1010
10101
\end{verbatim}
\clearpage

\subsection{Number Pattern 35}
This program prints a right triangle pattern alternating 1s and 0s per row.
\begin{lstlisting}[caption={Number Pattern 35}]
#include <stdio.h>
int main() {
    int n;
    printf("Enter size: ");
    scanf("%d", &n);
    for (int i = 0; i < n; i++) {
        for (int j = 0; j < n - i - 1; j++) printf(" ");
        for (int j = 0; j <= i; j++) printf("%d", i % 2);
        printf("\n");
    }
    return 0;
}
\end{lstlisting}
\textbf{Sample Output (for size=5):}
\begin{verbatim}
    1
   00
  111
 0000
11111
\end{verbatim}
\clearpage

\subsection{Number Pattern 36}
This program prints a right triangle pattern alternating 1s and 0s in a checkerboard style.
\begin{lstlisting}[caption={Number Pattern 36}]
#include <stdio.h>
int main() {
    int n;
    printf("Enter size: ");
    scanf("%d", &n);
    for (int i = 0; i < n; i++) {
        for (int j = 0; j < n - i - 1; j++) printf(" ");
        for (int j = 0; j <= i; j++) printf("%d", (i + j) % 2);
        printf("\n");
    }
    return 0;
}
\end{lstlisting}
\textbf{Sample Output (for size=5):}
\begin{verbatim}
    0
   10
  010
 1010
01010
\end{verbatim}
\clearpage

\subsection{Number Pattern 37}
This program prints a right triangle pattern with specific 1s and 0s.
\begin{lstlisting}[caption={Number Pattern 37}]
#include <stdio.h>
int main() {
    int n;
    printf("Enter size: ");
    scanf("%d", &n);
    for (int i = 0; i < n; i++) {
        for (int j = 0; j < n - i - 1; j++) printf(" ");
        for (int j = 0; j <= i; j++) {
            if (j == 0 || j == i || i == n-1)
                printf("1");
            else
                printf("0");
        }
        printf("\n");
    }
    return 0;
}
\end{lstlisting}
\textbf{Sample Output (for size=5):}
\begin{verbatim}
    1
   11
  101
 1001
11111
\end{verbatim}
\clearpage

\subsection{Number Pattern 38}
This program prints a right triangle pattern with sequential odd numbers.
\begin{lstlisting}[caption={Number Pattern 38}]
#include <stdio.h>
int main() {
    int n;
    printf("Enter size: ");
    scanf("%d", &n);
    for (int i = 0; i < n; i++) {
        for (int j = 0; j < n - i - 1; j++) printf(" ");
        for (int j = 0; j < 2 * i + 1; j++) printf("%d", 2 * i + 1);
        printf("\n");
    }
    return 0;
}
\end{lstlisting}
\textbf{Sample Output (for size=5):}
\begin{verbatim}
    111
   33333
  5555555
 777777777
99999999999
\end{verbatim}
\clearpage

\subsection{Number Pattern 39}
This program prints a right triangle pattern with sequential odd numbers.
\begin{lstlisting}[caption={Number Pattern 39}]
#include <stdio.h>
int main() {
    int n;
    printf("Enter size: ");
    scanf("%d", &n);
    for (int i = 0; i < n; i++) {
        for (int j = 0; j < n - i - 1; j++) printf(" ");
        for (int j = 0; j < 2 * i + 1; j++) printf("%d", j + 1);
        printf("\n");
    }
    return 0;
}
\end{lstlisting}
\textbf{Sample Output (for size=5):}
\begin{verbatim}
    1
   123
  12345
 1234567
123456789
\end{verbatim}
\clearpage

\subsection{Number Pattern 40}
This program prints a right triangle pattern with alternating even and odd numbers.
\begin{lstlisting}[caption={Number Pattern 40}]
#include <stdio.h>
int main() {
    int n;
    printf("Enter size: ");
    scanf("%d", &n);
    for (int i = 0; i < n; i++) {
        for (int j = 0; j < n - i - 1; j++) printf(" ");
        for (int j = 0; j <= i; j++) printf("%d", i % 2 == 0 ? 2 * j + 1 : 2 * (j + 1));
        printf("\n");
    }
    return 0;
}
\end{lstlisting}
\textbf{Sample Output (for size=5):}
\begin{verbatim}
    1
   24
  135
 2468
13579
\end{verbatim}
\clearpage

\subsection{Number Pattern 41}
This program prints a right triangle pattern with palindromic odd numbers.
\begin{lstlisting}[caption={Number Pattern 41}]
#include <stdio.h>
int main() {
    int n;
    printf("Enter size: ");
    scanf("%d", &n);
    for (int i = 0; i < n; i++) {
        for (int j = 0; j < n - i - 1; j++) printf(" ");
        for (int j = 1; j <= i + 1; j++) printf("%d", j);
        for (int j = i; j >= 1; j--) printf("%d", j);
        printf("\n");
    }
    return 0;
}
\end{lstlisting}
\textbf{Sample Output (for size=5):}
\begin{verbatim}
    1
   121
  12321
 1234321
123454321
\end{verbatim}
\clearpage

\subsection{Number Pattern 42}
This program prints a right triangle pattern with palindromic even numbers.
\begin{lstlisting}[caption={Number Pattern 42}]
#include <stdio.h>
int main() {
    int n;
    printf("Enter size: ");
    scanf("%d", &n);
    for (int i = 0; i < n; i++) {
        for (int j = 0; j < n - i - 1; j++) printf(" ");
        for (int j = 1; j <= i + 1; j++) printf("%d", 2 * j);
        for (int j = i; j >= 1; j--) printf("%d", 2 * j);
        printf("\n");
    }
    return 0;
}
\end{lstlisting}
\textbf{Sample Output (for size=5):}
\begin{verbatim}
    2
   242
  24642
 2468642
2468108642
\end{verbatim}
\clearpage

\subsection{Number Pattern 43}
This program prints a right triangle pattern with palindromic numbers.
\begin{lstlisting}[caption={Number Pattern 43}]
#include <stdio.h>
int main() {
    int n;
    printf("Enter size: ");
    scanf("%d", &n);
    for (int i = 0; i < n; i++) {
        for (int j = 0; j < n - i - 1; j++) printf(" ");
        for (int j = 1; j <= i + 1; j++) printf("%d", j);
        for (int j = i; j >= 1; j--) printf("%d", j);
        printf("\n");
    }
    return 0;
}
\end{lstlisting}
\textbf{Sample Output (for size=5):}
\begin{verbatim}
    1
   121
  12321
 1234321
123454321
\end{verbatim}
\clearpage

\subsection{Number Pattern 44}
This program prints a right triangle pattern with increasing sequential numbers.
\begin{lstlisting}[caption={Number Pattern 44}]
#include <stdio.h>
int main() {
    int n;
    printf("Enter size: ");
    scanf("%d", &n);
    for (int i = 0; i < n; i++) {
        for (int j = 0; j < n - i - 1; j++) printf(" ");
        for (int j = i + 1; j <= 2 * i + 1; j++) printf("%d", j);
        for (int j = 2 * i; j >= i + 1; j--) printf("%d", j);
        printf("\n");
    }
    return 0;
}
\end{lstlisting}
\textbf{Sample Output (for size=5):}
\begin{verbatim}
    1
   232
  34543
 4567654
567898765
\end{verbatim}
\clearpage

\subsection{Number Pattern 45}
This program prints a right triangle pattern with sequential numbers.
\begin{lstlisting}[caption={Number Pattern 45}]
#include <stdio.h>
int main() {
    int n;
    printf("Enter size: ");
    scanf("%d", &n);
    int count = 1;
    for (int i = 0; i < n; i++) {
        for (int j = 0; j < n - i - 1; j++) printf(" ");
        for (int j = 0; j <= i; j++) printf("%d ", count++);
        printf("\n");
    }
    return 0;
}
\end{lstlisting}
\textbf{Sample Output (for size=5):}
\begin{verbatim}
    1 
   2 3 
  4 5 6 
 7 8 9 10 
11 12 13 14 15 
\end{verbatim}
\clearpage

\subsection{Number Pattern 46}
This program prints a right triangle pattern with mixed sequential numbers.
\begin{lstlisting}[caption={Number Pattern 46}]
#include <stdio.h>
int main() {
    int n;
    printf("Enter size: ");
    scanf("%d", &n);
    for (int i = 0; i < n; i++) {
        for (int j = 0; j < n - i - 1; j++) printf(" ");
        if (i == 0) printf("1");
        else if (i == 1) printf("21");
        else {
            for (int j = i; j >= 1; j--) printf("%d", j);
            for (int j = 2; j <= i + 1; j++) printf("%d", j);
        }
        printf("\n");
    }
    return 0;
}
\end{lstlisting}
\textbf{Sample Output (for size=5):}
\begin{verbatim}
    1
   21
  32123
 4321234
543212345
\end{verbatim}
\clearpage

\subsection{Number Pattern 47}
This program prints a right triangle pattern with increasing sequential numbers.
\begin{lstlisting}[caption={Number Pattern 47}]
#include <stdio.h>
int main() {
    int n;
    printf("Enter size: ");
    scanf("%d", &n);
    for (int i = 0; i < n; i++) {
        for (int j = 0; j < n - i - 1; j++) printf(" ");
        for (int j = 1; j <= i + 1; j++) printf("%d", j);
        for (int j = i; j >= 1; j--) printf("%d", j);
        printf("\n");
    }
    return 0;
}
\end{lstlisting}
\textbf{Sample Output (for size=5):}
\begin{verbatim}
    1
   121
  12321
 1234321
123454321
\end{verbatim}
\clearpage

\subsection{Number Pattern 48}
This program prints a diamond pattern with mirrored numbers.
\begin{lstlisting}[caption={Number Pattern 48}]
#include <stdio.h>
int main() {
    int n;
    printf("Enter size: ");
    scanf("%d", &n);
    for (int i = 0; i < n; i++) {
        for (int j = 0; j < i; j++) printf(" ");
        printf("%d", i + 1);
        for (int j = 0; j < 2 * (n - i - 1); j++) printf(" ");
        if (i != n - 1) printf("%d", i + 1);
        printf("\n");
    }
    for (int i = n - 2; i >= 0; i--) {
        for (int j = 0; j < i; j++) printf(" ");
        printf("%d", i + 1);
        for (int j = 0; j < 2 * (n - i - 1); j++) printf(" ");
        if (i != n - 1) printf("%d", i + 1);
        printf("\n");
    }
    return 0;
}
\end{lstlisting}
\textbf{Sample Output (for size=5):}
\begin{verbatim}
1        1
 2      2
  3    3
   4  4
    5
   4  4
  3    3
 2      2
1        1
\end{verbatim}
\clearpage

\subsection{Number Pattern 49}
This program prints a right triangle pattern with specific number sequences.
\begin{lstlisting}[caption={Number Pattern 49}]
#include <stdio.h>
int main() {
    int n;
    printf("Enter size: ");
    scanf("%d", &n);
    int count = 1;
    for (int i = 0; i < n; i++) {
        for (int j = 0; j < n - i - 1; j++) printf(" ");
        printf("%d", i + 1);
        for (int j = 0; j < i; j++) {
            printf(" %d", count);
            count++;
        }
        printf("\n");
    }
    return 0;
}
\end{lstlisting}
\textbf{Sample Output (for size=5):}
\begin{verbatim}
    1
   2 1
  3 2 3
 4 4 5 6
5 7 8 9 10
\end{verbatim}
\clearpage

\subsection{Number Pattern 50}
This program prints a right triangle pattern with arithmetic sequence numbers.
\begin{lstlisting}[caption={Number Pattern 50}]
#include <stdio.h>
int main() {
    int n;
    printf("Enter size: ");
    scanf("%d", &n);
    int count = 1, diff = 1;
    for (int i = 0; i < n; i++) {
        for (int j = 0; j < n - i - 1; j++) printf(" ");
        for (int j = 0; j <= i; j++) {
            printf("%d ", count);
            count += diff;
            diff++;
        }
        printf("\n");
    }
    return 0;
}
\end{lstlisting}
\textbf{Sample Output (for size=5):}
\begin{verbatim}
    1 
   2 4 
  7 11 16 
 22 29 37 46 
56 67 79 92 106 
\end{verbatim}
\clearpage

\subsection{Number Pattern 51}
This program prints a right triangle pattern with mixed sequential numbers.
\begin{lstlisting}[caption={Number Pattern 51}]
#include <stdio.h>
int main() {
    int n;
    printf("Enter size: ");
    scanf("%d", &n);
    int count = 1;
    for (int i = 0; i < n; i++) {
        for (int j = 0; j < n - i - 1; j++) printf(" ");
        if (i % 2 == 0) {
            for (int j = 0; j <= i; j++) printf("%d ", count++);
        } else {
            int start = count + i;
            for (int j = 0; j <= i; j++) printf("%d ", start--);
            count += i + 1;
        }
        printf("\n");
    }
    return 0;
}
\end{lstlisting}
\textbf{Sample Output (for size=5):}
\begin{verbatim}
    1 
   3 2 
  4 5 6 
 10 9 8 7 
11 12 13 14 15 
\end{verbatim}
\clearpage

\subsection{Number Pattern 52}
This program prints a right triangle pattern with repeating numbers.
\begin{lstlisting}[caption={Number Pattern 52}]
#include <stdio.h>
int main() {
    int n;
    printf("Enter size: ");
    scanf("%d", &n);
    for (int i = 0; i < n; i++) {
        for (int j = 0; j < n - i - 1; j++) printf(" ");
        for (int j = 0; j <= i; j++) {
            if (i == n - 1)
                printf("1");
            else
                printf("%d", n - i);
        }
        printf("\n");
    }
    return 0;
}
\end{lstlisting}
\textbf{Sample Output (for size=5):}
\begin{verbatim}
    5
   44
  333
 2222
11111
\end{verbatim}
\clearpage

\subsection{Number Pattern 53}
This program prints an inverted right triangle pattern based on digits of input number N.
\begin{lstlisting}[caption={Number Pattern 53}]
#include <stdio.h>
#include <string.h>
int main() {
    char n[10];
    printf("Enter number N: ");
    scanf("%s", n);
    int len = strlen(n);
    for (int i = 0; i < len; i++) {
        for (int j = 0; j < len - i; j++) {
            printf("%c", n[j]);
        }
        printf("\n");
    }
    return 0;
}
\end{lstlisting}
\textbf{Sample Output (for N=12345):}
\begin{verbatim}
12345
1234
123
12
1
\end{verbatim}
\clearpage

\subsection{Number Pattern 54}
This program prints an inverted right triangle pattern starting from digits of input number N.
\begin{lstlisting}[caption={Number Pattern 54}]
#include <stdio.h>
#include <string.h>
int main() {
    char n[10];
    printf("Enter number N: ");
    scanf("%s", n);
    int len = strlen(n);
    for (int i = 0; i < len; i++) {
        for (int j = i; j < len; j++) {
            printf("%c", n[j]);
        }
        printf("\n");
    }
    return 0;
}
\end{lstlisting}
\textbf{Sample Output (for N=12345):}
\begin{verbatim}
12345
2345
345
45
5
\end{verbatim}
\clearpage

\subsection{Number Pattern 55}
This program prints a diamond pattern with sequential numbers.
\begin{lstlisting}[caption={Number Pattern 55}]
#include <stdio.h>
int main() {
    int n;
    printf("Enter size: ");
    scanf("%d", &n);
    for (int i = 0; i < n; i++) {
        for (int j = 0; j < n - i - 1; j++) printf(" ");
        for (int j = 1; j <= i + 1; j++) printf("%d", j);
        printf("\n");
    }
    for (int i = n - 2; i >= 0; i--) {
        for (int j = 0; j < n - i - 1; j++) printf(" ");
        for (int j = 1; j <= i + 1; j++) printf("%d", j);
        printf("\n");
    }
    return 0;
}
\end{lstlisting}
\textbf{Sample Output (for size=5):}
\begin{verbatim}
    1
   12
  123
 1234
12345
 1234
  123
   12
    1
\end{verbatim}
\clearpage

\subsection{Number Pattern 56}
This program prints a diamond pattern with odd sequential numbers.
\begin{lstlisting}[caption={Number Pattern 56}]
#include <stdio.h>
int main() {
    int n;
    printf("Enter size: ");
    scanf("%d", &n);
    for (int i = 0; i < n; i++) {
        for (int j = 0; j < n - i - 1; j++) printf(" ");
        for (int j = 0; j < 2 * i + 1; j++) printf("%d", j + 1);
        printf("\n");
    }
    for (int i = n - 2; i >= 0; i--) {
        for (int j = 0; j < n - i - 1; j++) printf(" ");
        for (int j = 0; j < 2 * i + 1; j++) printf("%d", j + 1);
        printf("\n");
    }
    return 0;
}
\end{lstlisting}
\textbf{Sample Output (for size=5):}
\begin{verbatim}
    1
   123
  12345
 1234567
123456789
 1234567
  12345
   123
    1
\end{verbatim}
\clearpage

\subsection{Number Pattern 57}
This program prints a diamond pattern with palindromic numbers.
\begin{lstlisting}[caption={Number Pattern 57}]
#include <stdio.h>
int main() {
    int n;
    printf("Enter size: ");
    scanf("%d", &n);
    for (int i = 0; i < n; i++) {
        for (int j = 0; j < n - i - 1; j++) printf(" ");
        for (int j = 1; j <= i + 1; j++) printf("%d", j);
        for (int j = i; j >= 1; j--) printf("%d", j);
        printf("\n");
    }
    for (int i = n - 2; i >= 0; i--) {
        for (int j = 0; j < n - i - 1; j++) printf(" ");
        for (int j = 1; j <= i + 1; j++) printf("%d", j);
        for (int j = i; j >= 1; j--) printf("%d", j);
        printf("\n");
    }
    return 0;
}
\end{lstlisting}
\textbf{Sample Output (for size=5):}
\begin{verbatim}
    1
   121
  12321
 1234321
123454321
 1234321
  12321
   121
    1
\end{verbatim}
\clearpage

\subsection{Number Pattern 58}
This program prints a diamond pattern with stars and palindromic numbers.
\begin{lstlisting}[caption={Number Pattern 58}]
#include <stdio.h>
int main() {
    int n;
    printf("Enter size: ");
    scanf("%d", &n);
    printf("*\n");
    for (int i = 0; i < n; i++) {
        printf("*");
        for (int j = 1; j <= i + 1; j++) printf("%d", j);
        for (int j = i; j >= 1; j--) printf("%d", j);
        printf("*\n");
    }
    for (int i = n - 2; i >= 0; i--) {
        printf("*");
        for (int j = 1; j <= i + 1; j++) printf("%d", j);
        for (int j = i; j >= 1; j--) printf("%d", j);
        printf("*\n");
    }
    printf("*\n");
    return 0;
}
\end{lstlisting}
\textbf{Sample Output (for size=5):}
\begin{verbatim}
*
*1*
*121*
*12321*
*1234321*
*123454321*
*1234321*
*12321*
*121*
*1*
*
\end{verbatim}
\clearpage

\subsection{Number Pattern 59}
This program prints a diamond pattern with mirrored numbers.
\begin{lstlisting}[caption={Number Pattern 59}]
#include <stdio.h>
int main() {
    int n;
    printf("Enter size: ");
    scanf("%d", &n);
    for (int i = 0; i < n; i++) {
        for (int j = 0; j < i; j++) printf(" ");
        printf("%d", i + 1);
        for (int j = 0; j < 2 * (n - i - 1); j++) printf(" ");
        if (i != n - 1) printf("%d", i + 1);
        printf("\n");
    }
    for (int i = n - 2; i >= 0; i--) {
        for (int j = 0; j < i; j++) printf(" ");
        printf("%d", i + 1);
        for (int j = 0; j < 2 * (n - i - 1); j++) printf(" ");
        if (i != n - 1) printf("%d", i + 1);
        printf("\n");
    }
    return 0;
}
\end{lstlisting}
\textbf{Sample Output (for size=5):}
\begin{verbatim}
1        1
 2      2
  3    3
   4  4
    5
   4  4
  3    3
 2      2
1        1
\end{verbatim}
\clearpage

% Starting programming exercises section
\section{Function and Recursion Programming Exercises}

% Subsection for each program
\subsection{Find Cube of a Number Using Function}
This program calculates the cube of a given number using a function.
\begin{lstlisting}[caption={Find Cube of a Number Using Function}]
#include <stdio.h>
double cube(double num) {
    return num * num * num;
}
int main() {
    double num;
    printf("Enter a number: ");
    scanf("%lf", &num);
    printf("Cube: %.2lf\n", cube(num));
    return 0;
}
\end{lstlisting}
\clearpage

\subsection{Diameter, Circumference, and Area of Circle Using Functions}
This program calculates the diameter, circumference, and area of a circle using functions.
\begin{lstlisting}[caption={Diameter, Circumference, and Area of Circle Using Functions}]
#include <stdio.h>
#define PI 3.14159
double diameter(double r) { return 2 * r; }
double circumference(double r) { return 2 * PI * r; }
double area(double r) { return PI * r * r; }
int main() {
    double radius;
    printf("Enter radius: ");
    scanf("%lf", &radius);
    printf("Diameter: %.2lf\n", diameter(radius));
    printf("Circumference: %.2lf\n", circumference(radius));
    printf("Area: %.2lf\n", area(radius));
    return 0;
}
\end{lstlisting}
\clearpage

\subsection{Maximum and Minimum Between Two Numbers Using Functions}
This program finds the maximum and minimum of two numbers using functions.
\begin{lstlisting}[caption={Maximum and Minimum Between Two Numbers Using Functions}]
#include <stdio.h>
double max(double a, double b) { return a > b ? a : b; }
double min(double a, double b) { return a < b ? a : b; }
int main() {
    double num1, num2;
    printf("Enter two numbers: ");
    scanf("%lf %lf", &num1, &num2);
    printf("Maximum: %.2lf\n", max(num1, num2));
    printf("Minimum: %.2lf\n", min(num1, num2));
    return 0;
}
\end{lstlisting}
\clearpage

\subsection{Check Even or Odd Using Functions}
This program checks if a number is even or odd using a function.
\begin{lstlisting}[caption={Check Even or Odd Using Functions}]
#include <stdio.h>
char* isEvenOdd(int num) { return num % 2 == 0 ? "Even" : "Odd"; }
int main() {
    int num;
    printf("Enter a number: ");
    scanf("%d", &num);
    printf("Number is %s\n", isEvenOdd(num));
    return 0;
}
\end{lstlisting}
\clearpage

\subsection{Check Prime, Armstrong, or Perfect Number Using Functions}
This program checks if a number is prime, Armstrong, or perfect using functions.
\begin{lstlisting}[caption={Check Prime, Armstrong, or Perfect Number Using Functions}]
#include <stdio.h>
#include <math.h>
int isPrime(int n) {
    if (n <= 1) return 0;
    for (int i = 2; i <= sqrt(n); i++)
        if (n % i == 0) return 0;
    return 1;
}
int isArmstrong(int n) {
    int sum = 0, temp = n, digits = 0;
    while (temp) { digits++; temp /= 10; }
    temp = n;
    while (temp) {
        sum += pow(temp % 10, digits);
        temp /= 10;
    }
    return sum == n;
}
int isPerfect(int n) {
    int sum = 1;
    for (int i = 2; i <= n / 2; i++)
        if (n % i == 0) sum += i;
    return sum == n;
}
int main() {
    int num;
    printf("Enter a number: ");
    scanf("%d", &num);
    printf("Prime: %s\n", isPrime(num) ? "Yes" : "No");
    printf("Armstrong: %s\n", isArmstrong(num) ? "Yes" : "No");
    printf("Perfect: %s\n", isPerfect(num) ? "Yes" : "No");
    return 0;
}
\end{lstlisting}
\clearpage

\subsection{Find All Prime Numbers in Interval Using Functions}
This program prints all prime numbers in a given interval using a function.
\begin{lstlisting}[caption={Find All Prime Numbers in Interval Using Functions}]
#include <stdio.h>
#include <math.h>
int isPrime(int n) {
    if (n <= 1) return 0;
    for (int i = 2; i <= sqrt(n); i++)
        if (n % i == 0) return 0;
    return 1;
}
void printPrimes(int start, int end) {
    for (int i = start; i <= end; i++)
        if (isPrime(i)) printf("%d ", i);
    printf("\n");
}
int main() {
    int start, end;
    printf("Enter interval (start end): ");
    scanf("%d %d", &start, &end);
    printf("Primes: ");
    printPrimes(start, end);
    return 0;
}
\end{lstlisting}
\clearpage

\subsection{Print All Strong Numbers in Interval Using Functions}
This program prints all strong numbers in a given interval using a function.
\begin{lstlisting}[caption={Print All Strong Numbers in Interval Using Functions}]
#include <stdio.h>
long factorial(int n) {
    return n == 0 ? 1 : n * factorial(n - 1);
}
int isStrong(int n) {
    int sum = 0, temp = n;
    while (temp) {
        sum += factorial(temp % 10);
        temp /= 10;
    }
    return sum == n;
}
void printStrong(int start, int end) {
    for (int i = start; i <= end; i++)
        if (isStrong(i)) printf("%d ", i);
    printf("\n");
}
int main() {
    int start, end;
    printf("Enter interval (start end): ");
    scanf("%d %d", &start, &end);
    printf("Strong numbers: ");
    printStrong(start, end);
    return 0;
}
\end{lstlisting}
\clearpage

\subsection{Print All Armstrong Numbers in Interval Using Functions}
This program prints all Armstrong numbers in a given interval using a function.
\begin{lstlisting}[caption={Print All Armstrong Numbers in Interval Using Functions}]
#include <stdio.h>
#include <math.h>
int isArmstrong(int n) {
    int sum = 0, temp = n, digits = 0;
    while (temp) { digits++; temp /= 10; }
    temp = n;
    while (temp) {
        sum += pow(temp % 10, digits);
        temp /= 10;
    }
    return sum == n;
}
void printArmstrong(int start, int end) {
    for (int i = start; i <= end; i++)
        if (isArmstrong(i)) printf("%d ", i);
    printf("\n");
}
int main() {
    int start, end;
    printf("Enter interval (start end): ");
    scanf("%d %d", &start, &end);
    printf("Armstrong numbers: ");
    printArmstrong(start, end);
    return 0;
}
\end{lstlisting}
\clearpage

\subsection{Print All Perfect Numbers in Interval Using Functions}
This program prints all perfect numbers in a given interval using a function.
\begin{lstlisting}[caption={Print All Perfect Numbers in Interval Using Functions}]
#include <stdio.h>
int isPerfect(int n) {
    int sum = 1;
    for (int i = 2; i <= n / 2; i++)
        if (n % i == 0) sum += i;
    return sum == n;
}
void printPerfect(int start, int end) {
    for (int i = start; i <= end; i++)
        if (isPerfect(i)) printf("%d ", i);
    printf("\n");
}
int main() {
    int start, end;
    printf("Enter interval (start end): ");
    scanf("%d %d", &start, &end);
    printf("Perfect numbers: ");
    printPerfect(start, end);
    return 0;
}
\end{lstlisting}
\clearpage

\subsection{Power of a Number Using Recursion}
This program calculates the power of a number using recursion.
\begin{lstlisting}[caption={Power of a Number Using Recursion}]
#include <stdio.h>
double power(double x, int y) {
    if (y == 0) return 1;
    if (y < 0) return 1 / power(x, -y);
    return x * power(x, y - 1);
}
int main() {
    double x;
    int y;
    printf("Enter base and exponent: ");
    scanf("%lf %d", &x, &y);
    printf("%.2lf^%d = %.2lf\n", x, y, power(x, y));
    return 0;
}
\end{lstlisting}
\clearpage

\subsection{Print Natural Numbers from 1 to n Using Recursion}
This program prints all natural numbers from 1 to n using recursion.
\begin{lstlisting}[caption={Print Natural Numbers from 1 to n Using Recursion}]
#include <stdio.h>
void printNatural(int n, int i) {
    if (i > n) return;
    printf("%d ", i);
    printNatural(n, i + 1);
}
int main() {
    int n;
    printf("Enter n: ");
    scanf("%d", &n);
    printf("Natural numbers: ");
    printNatural(n, 1);
    printf("\n");
    return 0;
}
\end{lstlisting}
\clearpage

\subsection{Print Even or Odd Numbers in Range Using Recursion}
This program prints all even or odd numbers in a given range using recursion.
\begin{lstlisting}[caption={Print Even or Odd Numbers in Range Using Recursion}]
#include <stdio.h>
void printEvenOdd(int start, int end, int isEven) {
    if (start > end) return;
    if (start % 2 == isEven) printf("%d", start);
    printEvenOdd(start + 1, end, isEven);
}
int main() {
    int start, end, choice;
    printf("Enter range (start end): ");
    scanf("%d %d", &start, &end);
    printf("Enter 0 for even, 1 for odd: ");
    scanf("%d", &choice);
    printf("%s numbers: ", choice == 0 ? "Even" : "Odd");
    printEvenOdd(start, end, choice);
    printf("\n");
    return 0;
}
\end{lstlisting}
\clearpage

\subsection{Sum of Natural Numbers from 1 to n Using Recursion}
This program calculates the sum of natural numbers from 1 to n using recursion.
\begin{lstlisting}[caption={Sum of Natural Numbers from 1 to n Using Recursion}]
#include <stdio.h>
int sumNatural(int n) {
    if (n <= 0) return 0;
    return n + sumNatural(n - 1);
}
int main() {
    int n;
    printf("Enter n: ");
    scanf("%d", &n);
    printf("Sum: %d\n", sumNatural(n));
    return 0;
}
\end{lstlisting}
\clearpage

\subsection{Sum of Even or Odd Numbers in Range Using Recursion}
This program calculates the sum of even or odd numbers in a given range using recursion.
\begin{lstlisting}[caption={Sum of Even or Odd Numbers in Range Using Recursion}]
#include <stdio.h>
int sumEvenOdd(int start, int end, int isEven) {
    if (start > end) return 0;
    if (start % 2 == isEven) return start + sumEvenOdd(start + 1, end, isEven);
    return sumEvenOdd(start + 1, end, isEven);
}
int main() {
    int start, end, choice;
    printf("Enter range (start end): ");
    scanf("%d %d", &start, &end);
    printf("Enter 0 for even, 1 for odd: ");
    scanf("%d", &choice);
    printf("Sum of %s numbers: %d\n", choice == 0 ? "even" : "odd", sumEvenOdd(start, end, choice));
    return 0;
}
\end{lstlisting}
\clearpage

\subsection{Reverse of a Number Using Recursion}
This program finds the reverse of a number using recursion.
\begin{lstlisting}[caption={Reverse of a Number Using Recursion}]
#include <stdio.h>
int reverse(int n, int rev) {
    if (n == 0) return rev;
    return reverse(n / 10, rev * 10 + n % 10);
}
int main() {
    int num;
    printf("Enter a number: ");
    scanf("%d", &num);
    printf("Reverse: %d\n", reverse(num, 0));
    return 0;
}
\end{lstlisting}
\clearpage

\subsection{Check Palindrome Using Recursion}
This program checks if a number is a palindrome using recursion.
\begin{lstlisting}[caption={Check Palindrome Using Recursion}]
#include <stdio.h>
int reverse(int n, int rev) {
    if (n == 0) return rev;
    return reverse(n / 10, rev * 10 + n % 10);
}
int isPalindrome(int n) {
    return n == reverse(n, 0);
}
int main() {
    int num;
    printf("Enter a number: ");
    scanf("%d", &num);
    printf("Palindrome: %s\n", isPalindrome(num) ? "Yes" : "No");
    return 0;
}
\end{lstlisting}
\clearpage

\subsection{Sum of Digits Using Recursion}
This program calculates the sum of digits of a number using recursion.
\begin{lstlisting}[caption={Sum of Digits Using Recursion}]
#include <stdio.h>
int sumDigits(int n) {
    if (n == 0) return 0;
    return (n % 10) + sumDigits(n / 10);
}
int main() {
    int num;
    printf("Enter a number: ");
    scanf("%d", &num);
    printf("Sum of digits: %d\n", sumDigits(num));
    return 0;
}
\end{lstlisting}
\clearpage

\subsection{Factorial of a Number Using Recursion}
This program calculates the factorial of a number using recursion.
\begin{lstlisting}[caption={Factorial of a Number Using Recursion}]
#include <stdio.h>
unsigned long long factorial(int n) {
    if (n <= 1) return 1;
    return n * factorial(n - 1);
}
int main() {
    int num;
    printf("Enter a number: ");
    scanf("%d", &num);
    if (num < 0) printf("Factorial not defined for negative numbers\n");
    else printf("Factorial: %llu\n", factorial(num));
    return 0;
}
\end{lstlisting}
\clearpage

\subsection{Generate nth Fibonacci Term Using Recursion}
This program generates the nth Fibonacci term using recursion.
\begin{lstlisting}[caption={Generate nth Fibonacci Term Using Recursion}]
#include <stdio.h>
unsigned long long fibonacci(int n) {
    if (n <= 1) return n;
    return fibonacci(n - 1) + fibonacci(n - 2);
}
int main() {
    int n;
    printf("Enter n: ");
    scanf("%d", &n);
    if (n < 0) printf("Invalid input\n");
    else printf("Fibonacci term: %llu\n", fibonacci(n));
    return 0;
}
\end{lstlisting}
\clearpage

\subsection{GCD of Two Numbers Using Recursion}
This program calculates the GCD of two numbers using recursion.
\begin{lstlisting}[caption={GCD of Two Numbers Using Recursion}]
#include <stdio.h>
int gcd(int a, int b) {
    if (b == 0) return a;
    return gcd(b, a % b);
}
int main() {
    int a, b;
    printf("Enter two numbers: ");
    scanf("%d %d", &a, &b);
    printf("GCD: %d\n", gcd(a, b));
    return 0;
}
\end{lstlisting}
\clearpage

\subsection{LCM of Two Numbers Using Recursion}
This program calculates the LCM of two numbers using recursion.
\begin{lstlisting}[caption={LCM of Two Numbers Using Recursion}]
#include <stdio.h>
int gcd(int a, int b) {
    if (b == 0) return a;
    return gcd(b, a % b);
}
int lcm(int a, int b) {
    return (a * b) / gcd(a, b);
}
int main() {
    int a, b;
    printf("Enter two numbers: ");
    scanf("%d %d", &a, &b);
    printf("LCM: %d\n", lcm(a, b));
    return 0;
}
\end{lstlisting}
\clearpage

\subsection{Display Array Elements Using Recursion}
This program displays all elements of an array using recursion.
\begin{lstlisting}[caption={Display Array Elements Using Recursion}]
#include <stdio.h>
void displayArray(int arr[], int size, int i) {
    if (i >= size) return;
    printf("%d ", arr[i]);
    displayArray(arr, size, i + 1);
}
int main() {
    int size;
    printf("Enter array size: ");
    scanf("%d", &size);
    int arr[size];
    printf("Enter %d elements: ", size);
    for (int i = 0; i < size; i++) scanf("%d", &arr[i]);
    printf("Array elements: ");
    displayArray(arr, size, 0);
    printf("\n");
    return 0;
}
\end{lstlisting}
\clearpage

\subsection{Sum of Array Elements Using Recursion}
This program calculates the sum of array elements using recursion.
\begin{lstlisting}[caption={Sum of Array Elements Using Recursion}]
#include <stdio.h>
int sumArray(int arr[], int size, int i) {
    if (i >= size) return 0;
    return arr[i] + sumArray(arr, size, i + 1);
}
int main() {
    int size;
    printf("Enter array size: ");
    scanf("%d", &size);
    int arr[size];
    printf("Enter %d elements: ", size);
    for (int i = 0; i < size; i++) scanf("%d", &arr[i]);
    printf("Sum of array: %d\n", sumArray(arr, size, 0));
    return 0;
}
\end{lstlisting}
\clearpage

\subsection{Maximum and Minimum Elements in Array Using Recursion}
This program finds the maximum and minimum elements in an array using recursion.
\begin{lstlisting}[caption={Maximum and Minimum Elements in Array Using Recursion}]
#include <stdio.h>
int maxArray(int arr[], int size, int i) {
    if (i == size - 1) return arr[i];
    int maxRest = maxArray(arr, size, i + 1);
    return arr[i] > maxRest ? arr[i] : maxRest;
}
int minArray(int arr[], int size, int i) {
    if (i == size - 1) return arr[i];
    int minRest = minArray(arr, size, i + 1);
    return arr[i] < minRest ? arr[i] : minRest;
}
int main() {
    int size;
    printf("Enter array size: ");
    scanf("%d", &size);
    int arr[size];
    printf("Enter %d elements: ", size);
    for (int i = 0; i < size; i++) scanf("%d", &arr[i]);
    printf("Maximum: %d\n", maxArray(arr, size, 0));
    printf("Minimum: %d\n", minArray(arr, size, 0));
    return 0;
}
\end{lstlisting}
\clearpage

% Starting programming exercises section
\section{Matrix Programming Exercises}

% Subsection for each program
\subsection{Add Two Matrices}
This program adds two matrices of the same size.
\begin{lstlisting}[caption={Add Two Matrices}]
#include <stdio.h>
int main() {
    int r, c;
    printf("Enter rows and columns: ");
    scanf("%d %d", &r, &c);
    int a[r][c], b[r][c], sum[r][c];
    printf("Enter first matrix elements:\n");
    for (int i = 0; i < r; i++)
        for (int j = 0; j < c; j++) scanf("%d", &a[i][j]);
    printf("Enter second matrix elements:\n");
    for (int i = 0; i < r; i++)
        for (int j = 0; j < c; j++) scanf("%d", &b[i][j]);
    for (int i = 0; i < r; i++)
        for (int j = 0; j < c; j++) sum[i][j] = a[i][j] + b[i][j];
    printf("Sum of matrices:\n");
    for (int i = 0; i < r; i++) {
        for (int j = 0; j < c; j++) printf("%d ", sum[i][j]);
        printf("\n");
    }
    return 0;
}
\end{lstlisting}
\clearpage

\subsection{Subtract Two Matrices}
This program subtracts one matrix from another of the same size.
\begin{lstlisting}[caption={Subtract Two Matrices}]
#include <stdio.h>
int main() {
    int r, c;
    printf("Enter rows and columns: ");
    scanf("%d %d", &r, &c);
    int a[r][c], b[r][c], diff[r][c];
    printf("Enter first matrix elements:\n");
    for (int i = 0; i < r; i++)
        for (int j = 0; j < c; j++) scanf("%d", &a[i][j]);
    printf("Enter second matrix elements:\n");
    for (int i = 0; i < r; i++)
        for (int j = 0; j < c; j++) scanf("%d", &b[i][j]);
    for (int i = 0; i < r; i++)
        for (int j = 0; j < c; j++) diff[i][j] = a[i][j] - b[i][j];
    printf("Difference of matrices:\n");
    for (int i = 0; i < r; i++) {
        for (int j = 0; j < c; j++) printf("%d ", diff[i][j]);
        printf("\n");
    }
    return 0;
}
\end{lstlisting}
\clearpage

\subsection{Scalar Matrix Multiplication}
This program multiplies a matrix by a scalar value.
\begin{lstlisting}[caption={Scalar Matrix Multiplication}]
#include <stdio.h>
int main() {
    int r, c, scalar;
    printf("Enter rows and columns: ");
    scanf("%d %d", &r, &c);
    int a[r][c];
    printf("Enter matrix elements:\n");
    for (int i = 0; i < r; i++)
        for (int j = 0; j < c; j++) scanf("%d", &a[i][j]);
    printf("Enter scalar value: ");
    scanf("%d", &scalar);
    printf("Matrix after scalar multiplication:\n");
    for (int i = 0; i < r; i++) {
        for (int j = 0; j < c; j++) printf("%d ", a[i][j] * scalar);
        printf("\n");
    }
    return 0;
}
\end{lstlisting}
\clearpage

\subsection{Multiply Two Matrices}
This program multiplies two matrices if compatible.
\begin{lstlisting}[caption={Multiply Two Matrices}]
#include <stdio.h>
int main() {
    int r1, c1, r2, c2;
    printf("Enter rows and columns of first matrix: ");
    scanf("%d %d", &r1, &c1);
    printf("Enter rows and columns of second matrix: ");
    scanf("%d %d", &r2, &c2);
    if (c1 != r2) {
        printf("Matrix multiplication not possible\n");
        return 1;
    }
    int a[r1][c1], b[r2][c2], product[r1][c2];
    printf("Enter first matrix elements:\n");
    for (int i = 0; i < r1; i++)
        for (int j = 0; j < c1; j++) scanf("%d", &a[i][j]);
    printf("Enter second matrix elements:\n");
    for (int i = 0; i < r2; i++)
        for (int j = 0; j < c2; j++) scanf("%d", &b[i][j]);
    for (int i = 0; i < r1; i++)
        for (int j = 0; j < c2; j++) {
            product[i][j] = 0;
            for (int k = 0; k < c1; k++)
                product[i][j] += a[i][k] * b[k][j];
        }
    printf("Product of matrices:\n");
    for (int i = 0; i < r1; i++) {
        for (int j = 0; j < c2; j++) printf("%d ", product[i][j]);
        printf("\n");
    }
    return 0;
}
\end{lstlisting}
\clearpage

\subsection{Check Whether Two Matrices Are Equal}
This program checks if two matrices are equal.
\begin{lstlisting}[caption={Check Whether Two Matrices Are Equal}]
#include <stdio.h>
int main() {
    int r1, c1, r2, c2;
    printf("Enter rows and columns of first matrix: ");
    scanf("%d %d", &r1, &c1);
    printf("Enter rows and columns of second matrix: ");
    scanf("%d %d", &r2, &c2);
    if (r1 != r2 || c1 != c2) {
        printf("Matrices are not equal (different dimensions)\n");
        return 1;
    }
    int a[r1][c1], b[r2][c2];
    printf("Enter first matrix elements:\n");
    for (int i = 0; i < r1; i++)
        for (int j = 0; j < c1; j++) scanf("%d", &a[i][j]);
    printf("Enter second matrix elements:\n");
    for (int i = 0; i < r2; i++)
        for (int j = 0; j < c2; j++) scanf("%d", &b[i][j]);
    int equal = 1;
    for (int i = 0; i < r1; i++)
        for (int j = 0; j < c1; j++)
            if (a[i][j] != b[i][j]) {
                equal = 0;
                break;
            }
    printf("Matrices are %s\n", equal ? "equal" : "not equal");
    return 0;
}
\end{lstlisting}
\clearpage

\subsection{Sum of Main Diagonal Elements}
This program finds the sum of the main diagonal elements of a matrix.
\begin{lstlisting}[caption={Sum of Main Diagonal Elements}]
#include <stdio.h>
int main() {
    int n;
    printf("Enter size of square matrix: ");
    scanf("%d", &n);
    int a[n][n], sum = 0;
    printf("Enter matrix elements:\n");
    for (int i = 0; i < n; i++)
        for (int j = 0; j < n; j++) {
            scanf("%d", &a[i][j]);
            if (i == j) sum += a[i][j];
        }
    printf("Sum of main diagonal: %d\n", sum);
    return 0;
}
\end{lstlisting}
\clearpage

\subsection{Sum of Minor Diagonal Elements}
This program finds the sum of the minor diagonal elements of a matrix.
\begin{lstlisting}[caption={Sum of Minor Diagonal Elements}]
#include <stdio.h>
int main() {
    int n;
    printf("Enter size of square matrix: ");
    scanf("%d", &n);
    int a[n][n], sum = 0;
    printf("Enter matrix elements:\n");
    for (int i = 0; i < n; i++)
        for (int j = 0; j < n; j++) {
            scanf("%d", &a[i][j]);
            if (i + j == n - 1) sum += a[i][j];
        }
    printf("Sum of minor diagonal: %d\n", sum);
    return 0;
}
\end{lstlisting}
\clearpage

\subsection{Sum of Each Row and Column}
This program finds the sum of each row and column of a matrix.
\begin{lstlisting}[caption={Sum of Each Row and Column}]
#include <stdio.h>
int main() {
    int r, c;
    printf("Enter rows and columns: ");
    scanf("%d %d", &r, &c);
    int a[r][c];
    printf("Enter matrix elements:\n");
    for (int i = 0; i < r; i++)
        for (int j = 0; j < c; j++) scanf("%d", &a[i][j]);
    printf("Row sums:\n");
    for (int i = 0; i < r; i++) {
        int rowSum = 0;
        for (int j = 0; j < c; j++) rowSum += a[i][j];
        printf("Row %d: %d\n", i + 1, rowSum);
    }
    printf("Column sums:\n");
    for (int j = 0; j < c; j++) {
        int colSum = 0;
        for (int i = 0; i < r; i++) colSum += a[i][j];
        printf("Column %d: %d\n", j + 1, colSum);
    }
    return 0;
}
\end{lstlisting}
\clearpage

\subsection{Interchange Diagonals of a Matrix}
This program interchanges the main and minor diagonals of a square matrix.
\begin{lstlisting}[caption={Interchange Diagonals of a Matrix}]
#include <stdio.h>
int main() {
    int n;
    printf("Enter size of square matrix: ");
    scanf("%d", &n);
    int a[n][n];
    printf("Enter matrix elements:\n");
    for (int i = 0; i < n; i++)
        for (int j = 0; j < n; j++) scanf("%d", &a[i][j]);
    for (int i = 0; i < n; i++) {
        int temp = a[i][i];
        a[i][i] = a[i][n - 1 - i];
        a[i][n - 1 - i] = temp;
    }
    printf("Matrix after interchanging diagonals:\n");
    for (int i = 0; i < n; i++) {
        for (int j = 0; j < n; j++) printf("%d ", a[i][j]);
        printf("\n");
    }
    return 0;
}
\end{lstlisting}
\clearpage

\subsection{Find Upper Triangular Matrix}
This program displays the upper triangular part of a matrix.
\begin{lstlisting}[caption={Find Upper Triangular Matrix}]
#include <stdio.h>
int main() {
    int r, c;
    printf("Enter rows and columns: ");
    scanf("%d %d", &r, &c);
    int a[r][c];
    printf("Enter matrix elements:\n");
    for (int i = 0; i < r; i++)
        for (int j = 0; j < c; j++) scanf("%d", &a[i][j]);
    printf("Upper triangular matrix:\n");
    for (int i = 0; i < r; i++) {
        for (int j = 0; j < c; j++)
            if (j >= i) printf("%d ", a[i][j]);
            else printf("0 ");
        printf("\n");
    }
    return 0;
}
\end{lstlisting}
\clearpage

\subsection{Find Lower Triangular Matrix}
This program displays the lower triangular part of a matrix.
\begin{lstlisting}[caption={Find Lower Triangular Matrix}]
#include <stdio.h>
int main() {
    int r, c;
    printf("Enter rows and columns: ");
    scanf("%d %d", &r, &c);
    int a[r][c];
    printf("Enter matrix elements:\n");
    for (int i = 0; i < r; i++)
        for (int j = 0; j < c; j++) scanf("%d", &a[i][j]);
    printf("Lower triangular matrix:\n");
    for (int i = 0; i < r; i++) {
        for (int j = 0; j < c; j++)
            if (j <= i) printf("%d ", a[i][j]);
            else printf("0 ");
        printf("\n");
    }
    return 0;
}
\end{lstlisting}
\clearpage

\subsection{Sum of Upper Triangular Matrix}
This program finds the sum of elements in the upper triangular part of a matrix.
\begin{lstlisting}[caption={Sum of Upper Triangular Matrix}]
#include <stdio.h>
int main() {
    int n;
    printf("Enter size of square matrix: ");
    scanf("%d", &n);
    int a[n][n], sum = 0;
    printf("Enter matrix elements:\n");
    for (int i = 0; i < n; i++)
        for (int j = 0; j < n; j++) {
            scanf("%d", &a[i][j]);
            if (j >= i) sum += a[i][j];
        }
    printf("Sum of upper triangular elements: %d\n", sum);
    return 0;
}
\end{lstlisting}
\clearpage

\subsection{Sum of Lower Triangular Matrix}
This program finds the sum of elements in the lower triangular part of a matrix.
\begin{lstlisting}[caption={Sum of Lower Triangular Matrix}]
#include <stdio.h>
int main() {
    int n;
    printf("Enter size of square matrix: ");
    scanf("%d", &n);
    int a[n][n], sum = 0;
    printf("Enter matrix elements:\n");
    for (int i = 0; i < n; i++)
        for (int j = 0; j < n; j++) {
            scanf("%d", &a[i][j]);
            if (j <= i) sum += a[i][j];
        }
    printf("Sum of lower triangular elements: %d\n", sum);
    return 0;
}
\end{lstlisting}
\clearpage

\subsection{Find Transpose of a Matrix}
This program finds the transpose of a matrix.
\begin{lstlisting}[caption={Find Transpose of a Matrix}]
#include <stdio.h>
int main() {
    int r, c;
    printf("Enter rows and columns: ");
    scanf("%d %d", &r, &c);
    int a[r][c], transpose[c][r];
    printf("Enter matrix elements:\n");
    for (int i = 0; i < r; i++)
        for (int j = 0; j < c; j++) scanf("%d", &a[i][j]);
    for (int i = 0; i < r; i++)
        for (int j = 0; j < c; j++) transpose[j][i] = a[i][j];
    printf("Transpose of matrix:\n");
    for (int i = 0; i < c; i++) {
        for (int j = 0; j < r; j++) printf("%d ", transpose[i][j]);
        printf("\n");
    }
    return 0;
}
\end{lstlisting}
\clearpage

\subsection{Find Determinant of a Matrix}
This program calculates the determinant of a 3x3 matrix.
\begin{lstlisting}[caption={Find Determinant of a Matrix}]
#include <stdio.h>
int main() {
    int a[3][3];
    printf("Enter elements of 3x3 matrix:\n");
    for (int i = 0; i < 3; i++)
        for (int j = 0; j < 3; j++) scanf("%d", &a[i][j]);
    int det = a[0][0] * (a[1][1] * a[2][2] - a[1][2] * a[2][1]) -
              a[0][1] * (a[1][0] * a[2][2] - a[1][2] * a[2][0]) +
              a[0][2] * (a[1][0] * a[2][1] - a[1][1] * a[2][0]);
    printf("Determinant: %d\n", det);
    return 0;
}
\end{lstlisting}
\clearpage

\subsection{Check Identity Matrix}
This program checks if a matrix is an identity matrix.
\begin{lstlisting}[caption={Check Identity Matrix}]
#include <stdio.h>
int main() {
    int n;
    printf("Enter size of square matrix: ");
    scanf("%d", &n);
    int a[n][n], isIdentity = 1;
    printf("Enter matrix elements:\n");
    for (int i = 0; i < n; i++)
        for (int j = 0; j < n; j++) {
            scanf("%d", &a[i][j]);
            if (i == j && a[i][j] != 1) isIdentity = 0;
            if (i != j && a[i][j] != 0) isIdentity = 0;
        }
    printf("Matrix is %s\n", isIdentity ? "an identity matrix" : "not an identity matrix");
    return 0;
}
\end{lstlisting}
\clearpage

\subsection{Check Sparse Matrix}
This program checks if a matrix is sparse (most elements are zero).
\begin{lstlisting}[caption={Check Sparse Matrix}]
#include <stdio.h>
int main() {
    int r, c;
    printf("Enter rows and columns: ");
    scanf("%d %d", &r, &c);
    int a[r][c], zeroCount = 0;
    printf("Enter matrix elements:\n");
    for (int i = 0; i < r; i++)
        for (int j = 0; j < c; j++) {
            scanf("%d", &a[i][j]);
            if (a[i][j] == 0) zeroCount++;
        }
    int total = r * c;
    printf("Matrix is %s\n", zeroCount > total / 2 ? "sparse" : "not sparse");
    return 0;
}
\end{lstlisting}
\clearpage

\subsection{Check Symmetric Matrix}
This program checks if a matrix is symmetric (equal to its transpose).
\begin{lstlisting}[caption={Check Symmetric Matrix}]
#include <stdio.h>
int main() {
    int n;
    printf("Enter size of square matrix: ");
    scanf("%d", &n);
    int a[n][n], isSymmetric = 1;
    printf("Enter matrix elements:\n");
    for (int i = 0; i < n; i++)
        for (int j = 0; j < n; j++) scanf("%d", &a[i][j]);
    for (int i = 0; i < n; i++)
        for (int j = 0; j < n; j++)
            if (a[i][j] != a[j][i]) {
                isSymmetric = 0;
                break;
            }
    printf("Matrix is %s\n", isSymmetric ? "symmetric" : "not symmetric");
    return 0;
}
\end{lstlisting}
\clearpage

% Starting programming exercises section
\section{String Programming Exercises}

% Subsection for each program
\subsection{Find Length of a String}
This program finds the length of a string without using library functions.
\begin{lstlisting}[caption={Find Length of a String}]
#include <stdio.h>
int main() {
    char str[100];
    printf("Enter a string: ");
    scanf("%[^\n]", str);
    int len = 0;
    while (str[len] != '\0') len++;
    printf("Length of string: %d\n", len);
    return 0;
}
\end{lstlisting}
\clearpage

\subsection{Copy One String to Another String}
This program copies one string to another without using library functions.
\begin{lstlisting}[caption={Copy One String to Another String}]
#include <stdio.h>
int main() {
    char src[100], dest[100];
    printf("Enter a string: ");
    scanf("%[^\n]", src);
    int i = 0;
    while (src[i] != '\0') {
        dest[i] = src[i];
        i++;
    }
    dest[i] = '\0';
    printf("Copied string: %s\n", dest);
    return 0;
}
\end{lstlisting}
\clearpage

\subsection{Concatenate Two Strings}
This program concatenates two strings without using library functions.
\begin{lstlisting}[caption={Concatenate Two Strings}]
#include <stdio.h>
int main() {
    char str1[200], str2[100];
    printf("Enter first string: ");
    scanf("%[^\n]", str1);
    getchar();
    printf("Enter second string: ");
    scanf("%[^\n]", str2);
    int i = 0, j = 0;
    while (str1[i] != '\0') i++;
    while (str2[j] != '\0') {
        str1[i] = str2[j];
        i++;
        j++;
    }
    str1[i] = '\0';
    printf("Concatenated string: %s\n", str1);
    return 0;
}
\end{lstlisting}
\clearpage

\subsection{Compare Two Strings}
This program compares two strings without using library functions.
\begin{lstlisting}[caption={Compare Two Strings}]
#include <stdio.h>
int main() {
    char str1[100], str2[100];
    printf("Enter first string: ");
    scanf("%[^\n]", str1);
    getchar();
    printf("Enter second string: ");
    scanf("%[^\n]", str2);
    int i = 0, equal = 1;
    while (str1[i] != '\0' || str2[i] != '\0') {
        if (str1[i] != str2[i]) {
            equal = 0;
            break;
        }
        i++;
    }
    printf("Strings are %s\n", equal ? "equal" : "not equal");
    return 0;
}
\end{lstlisting}
\clearpage

\subsection{Convert Lowercase String to Uppercase}
This program converts a lowercase string to uppercase.
\begin{lstlisting}[caption={Convert Lowercase String to Uppercase}]
#include <stdio.h>
int main() {
    char str[100];
    printf("Enter a string: ");
    scanf("%[^\n]", str);
    for (int i = 0; str[i] != '\0'; i++)
        if (str[i] >= 'a' && str[i] <= 'z')
            str[i] = str[i] - 32;
    printf("Uppercase string: %s\n", str);
    return 0;
}
\end{lstlisting}
\clearpage

\subsection{Convert Uppercase String to Lowercase}
This program converts an uppercase string to lowercase.
\begin{lstlisting}[caption={Convert Uppercase String to Lower episode::Lowercase}]
#include <stdio.h>
int main() {
    char str[100];
    printf("Enter a string: ");
    scanf("%[^\n]", str);
    for (int i = 0; str[i] != '\0'; i++)
        if (str[i] >= 'A' && str[i] <= 'Z')
            str[i] = str[i] + 32;
    printf("Lowercase string: %s\n", str);
    return 0;
}
\end{lstlisting}
\clearpage

\subsection{Toggle Case of Each Character}
This program toggles the case of each character in a string.
\begin{lstlisting}[caption={Toggle Case of Each Character}]
#include <stdio.h>
int main() {
    char str[100];
    printf("Enter a string: ");
    scanf("%[^\n]", str);
    for (int i = 0; str[i] != '\0'; i++) {
        if (str[i] >= 'a' && str[i] <= 'z')
            str[i] = str[i] - 32;
        else if (str[i] >= 'A' && str[i] <= 'Z')
            str[i] = str[i] + 32;
    }
    printf("Toggled case string: %s\n", str);
    return 0;
}
\end{lstlisting}
\clearpage

\subsection{Count Alphabets, Digits, and Special Characters}
This program counts the total number of alphabets, digits, and special characters in a string.
\begin{lstlisting}[caption={Count Alphabets, Digits, and Special Characters}]
#include <stdio.h>
int main() {
    char str[100];
    int alphabets = 0, digits = 0, special = 0;
    printf("Enter a string: ");
    scanf("%[^\n]", str);
    for (int i = 0; str[i] != '\0'; i++) {
        if ((str[i] >= 'a' && str[i] <= 'z') || (str[i] >= 'A' && str[i] <= 'Z'))
            alphabets++;
        else if (str[i] >= '0' && str[i] <= '9')
            digits++;
        else
            special++;
    }
    printf("Alphabets: %d\nDigits: %d\nSpecial characters: %d\n", alphabets, digits, special);
    return 0;
}
\end{lstlisting}
\clearpage

\subsection{Count Vowels and Consonants}
This program counts the total number of vowels and consonants in a string.
\begin{lstlisting}[caption={Count Vowels and Consonants}]
#include <stdio.h>
int main() {
    char str[100];
    int vowels = 0, consonants = 0;
    printf("Enter a string: ");
    scanf("%[^\n]", str);
    for (int i = 0; str[i] != '\0'; i++) {
        char ch = str[i] >= 'A' && str[i] <= 'Z' ? str[i] + 32 : str[i];
        if (ch == 'a' || ch == 'e' || ch == 'i' || ch == 'o' || ch == 'u')
            vowels++;
        else if ((ch >= 'a' && ch <= 'z'))
            consonants++;
    }
    printf("Vowels: %d\nConsonants: %d\n", vowels, consonants);
    return 0;
}
\end{lstlisting}
\clearpage

\subsection{Count Total Number of Words}
This program counts the total number of words in a string.
\begin{lstlisting}[caption={Count Total Number of Words}]
#include <stdio.h>
int main() {
    char str[100];
    int words = 0, inWord = 0;
    printf("Enter a string: ");
    scanf("%[^\n]", str);
    for (int i = 0; str[i] != '\0'; i++) {
        if (str[i] != ' ' && !inWord) {
            inWord = 1;
            words++;
        } else if (str[i] == ' ')
            inWord = 0;
    }
    printf("Total words: %d\n", words);
    return 0;
}
\end{lstlisting}
\clearpage

\subsection{Find Reverse of a String}
This program reverses a string.
\begin{lstlisting}[caption={Find Reverse of a String}]
#include <stdio.h>
int main() {
    char str[100], temp;
    printf("Enter a string: ");
    scanf("%[^\n]", str);
    int len = 0;
    while (str[len] != '\0') len++;
    for (int i = 0, j = len - 1; i < j; i++, j--) {
        temp = str[i];
        str[i] = str[j];
        str[j] = temp;
    }
    printf("Reversed string: %s\n", str);
    return 0;
}
\end{lstlisting}
\clearpage

\subsection{Check Whether a String is Palindrome}
This program checks if a string is a palindrome.
\begin{lstlisting}[caption={Check Whether a String is Palindrome}]
#include <stdio.h>
int main() {
    char str[100];
    printf("Enter a string: ");
    scanf("%[^\n]", str);
    int len = 0, isPalindrome = 1;
    while (str[len] != '\0') len++;
    for (int i = 0, j = len - 1; i < j; i++, j--)
        if (str[i] != str[j]) {
            isPalindrome = 0;
            break;
        }
    printf("String is %s\n", isPalindrome ? "a palindrome" : "not a palindrome");
    return 0;
}
\end{lstlisting}
\clearpage

\subsection{Reverse Order of Words}
This program reverses the order of words in a string.
\begin{lstlisting}[caption={Reverse Order of Words}]
#include <stdio.h>
#include <string.h>
void reverse(char str[], int start, int end) {
    char temp;
    while (start < end) {
        temp = str[start];
        str[start++] = str[end];
        str[end--] = temp;
    }
}
int main() {
    char str[100];
    printf("Enter a string: ");
    scanf("%[^\n]", str);
    int len = strlen(str);
    reverse(str, 0, len - 1);
    int start = 0;
    for (int i = 0; i <= len; i++) {
        if (str[i] == ' ' || str[i] == '\0') {
            reverse(str, start, i - 1);
            start = i + 1;
        }
    }
    printf("Reversed words: %s\n", str);
    return 0;
}
\end{lstlisting}
\clearpage

\subsection{Find First Occurrence of a Character}
This program finds the first occurrence of a character in a string.
\begin{lstlisting}[caption={Find First Occurrence of a Character}]
#include <stdio.h>
int main() {
    char str[100], ch;
    printf("Enter a string: ");
    scanf("%[^\n]", str);
    getchar();
    printf("Enter a character: ");
    scanf("%c", &ch);
    int pos = -1;
    for (int i = 0; str[i] != '\0'; i++)
        if (str[i] == ch) {
            pos = i + 1;
            break;
        }
    if (pos == -1) printf("Character not found\n");
    else printf("First occurrence at position: %d\n", pos);
    return 0;
}
\end{lstlisting}
\clearpage

\subsection{Find Last Occurrence of a Character}
This program finds the length of a string without using library functions.
\begin{lstlisting}[caption={Find Length of a String}]
#include <stdio.h>
int main() {
    char str[100];
    printf("Enter a string: ");
    scanf("%[^\n]", str);
    int len = 0;
    while (str[len] != '\0') len++;
    printf("Length of string: %d\n", len);
    return 0;
}
\end{lstlisting}
\clearpage

\subsection{Copy One String to Another String}
This program copies one string to another without using library functions.
\begin{lstlisting}[caption={Copy One String to Another String}]
#include <stdio.h>
int main() {
    char src[100], dest[100];
    printf("Enter a string: ");
    scanf("%[^\n]", src);
    int i = 0;
    while (src[i] != '\0') {
        dest[i] = src[i];
        i++;
    }
    dest[i] = '\0';
    printf("Copied string: %s\n", dest);
    return 0;
}
\end{lstlisting}
\clearpage

\subsection{Concatenate Two Strings}
This program concatenates two strings without using library functions.
\begin{lstlisting}[caption={Concatenate Two Strings}]
#include <stdio.h>
int main() {
    char str1[200], str2[100];
    printf("Enter first string: ");
    scanf("%[^\n]", str1);
    getchar();
    printf("Enter second string: ");
    scanf("%[^\n]", str2);
    int i = 0, j = 0;
    while (str1[i] != '\0') i++;
    while (str2[j] != '\0') {
        str1[i] = str2[j];
        i++;
        j++;
    }
    str1[i] = '\0';
    printf("Concatenated string: %s\n", str1);
    return 0;
}
\end{lstlisting}
\clearpage

\subsection{Compare Two Strings}
This program compares two strings without using library functions.
\begin{lstlisting}[caption={Compare Two Strings}]
#include <stdio.h>
int main() {
    char str1[100], str2[100];
    printf("Enter first string: ");
    scanf("%[^\n]", str1);
    getchar();
    printf("Enter second string: ");
    scanf("%[^\n]", str2);
    int i = 0, equal = 1;
    while (str1[i] != '\0' || str2[i] != '\0') {
        if (str1[i] != str2[i]) {
            equal = 0;
            break;
        }
        i++;
    }
    printf("Strings are %s\n", equal ? "equal" : "not equal");
    return 0;
}
\end{lstlisting}
\clearpage

\subsection{Convert Lowercase String to Uppercase}
This program converts a lowercase string to uppercase.
\begin{lstlisting}[caption={Convert Lowercase String to Uppercase}]
#include <stdio.h>
int main() {
    char str[100];
    printf("Enter a string: ");
    scanf("%[^\n]", str);
    for (int i = 0; str[i] != '\0'; i++)
        if (str[i] >= 'a' && str[i] <= 'z')
            str[i] = str[i] - 32;
    printf("Uppercase string: %s\n", str);
    return 0;
}
\end{lstlisting}
\clearpage

\subsection{Convert Uppercase String to Lowercase}
This program converts an uppercase string to lowercase.
\begin{lstlisting}[caption={Convert Uppercase String to Lowercase}]
#include <stdio.h>
int main() {
    char str[100];
    printf("Enter a string: ");
    scanf("%[^\n]", str);
    for (int i = 0; str[i] != '\0'; i++)
        if (str[i] >= 'A' && str[i] <= 'Z')
            str[i] = str[i] + 32;
    printf("Lowercase string: %s\n", str);
    return 0;
}
\end{lstlisting}
\clearpage

\subsection{Toggle Case of Each Character}
This program toggles the case of each character in a string.
\begin{lstlisting}[caption={Toggle Case of Each Character}]
#include <stdio.h>
int main() {
    char str[100];
    printf("Enter a string: ");
    scanf("%[^\n]", str);
    for (int i = 0; str[i] != '\0'; i++) {
        if (str[i] >= 'a' && str[i] <= 'z')
            str[i] = str[i] - 32;
        else if (str[i] >= 'A' && str[i] <= 'Z')
            str[i] = str[i] + 32;
    }
    printf("Toggled case string: %s\n", str);
    return 0;
}
\end{lstlisting}
\clearpage

\subsection{Count Alphabets, Digits, and Special Characters}
This program counts the total number of alphabets, digits, and special characters in a string.
\begin{lstlisting}[caption={Count Alphabets, Digits, and Special Characters}]
#include <stdio.h>
int main() {
    char str[100];
    int alphabets = 0, digits = 0, special = 0;
    printf("Enter a string: ");
    scanf("%[^\n]", str);
    for (int i = 0; str[i] != '\0'; i++) {
        if ((str[i] >= 'a' && str[i] <= 'z') || (str[i] >= 'A' && str[i] <= 'Z'))
            alphabets++;
        else if (str[i] >= '0' && str[i] <= '9')
            digits++;
        else
            special++;
    }
    printf("Alphabets: %d\nDigits: %d\nSpecial characters: %d\n", alphabets, digits, special);
    return 0;
}
\end{lstlisting}
\clearpage

\subsection{Count Vowels and Consonants}
This program counts the total number of vowels and consonants in a string.
\begin{lstlisting}[caption={Count Vowels and Consonants}]
#include <stdio.h>
int main() {
    char str[100];
    int vowels = 0, consonants = 0;
    printf("Enter a string: ");
    scanf("%[^\n]", str);
    for (int i = 0; str[i] != '\0'; i++) {
        char ch = str[i] >= 'A' && str[i] <= 'Z' ? str[i] + 32 : str[i];
        if (ch == 'a' || ch == 'e' || ch == 'i' || ch == 'o' || ch == 'u')
            vowels++;
        else if ((ch >= 'a' && ch <= 'z'))
            consonants++;
    }
    printf("Vowels: %d\nConsonants: %d\n", vowels, consonants);
    return 0;
}
\end{lstlisting}
\clearpage

\subsection{Count Total Number of Words}
This program counts the total number of words in a string.
\begin{lstlisting}[caption={Count Total Number of Words}]
#include <stdio.h>
int main() {
    char str[100];
    int words = 0, inWord = 0;
    printf("Enter a string: ");
    scanf("%[^\n]", str);
    for (int i = 0; str[i] != '\0'; i++) {
        if (str[i] != ' ' && !inWord) {
            inWord = 1;
            words++;
        } else if (str[i] == ' ')
            inWord = 0;
    }
    printf("Total words: %d\n", words);
    return 0;
}
\end{lstlisting}
\clearpage

\subsection{Find Reverse of a String}
This program reverses a string.
\begin{lstlisting}[ Caption={Find Reverse of a String}]
#include <stdio.h>
int main() {
    char str[100], temp;
    printf("Enter a string: ");
    scanf("%[^\n]", str);
    int len = 0;
    while (str[len] != '\0') len++;
    for (int i = 0, j = len - 1; i < j; i++, j--) {
        temp = str[i];
        str[i] = str[j];
        str[j] = temp;
    }
    printf("Reversed string: %s\n", str);
    return 0;
}
\end{lstlisting}
\clearpage

\subsection{Check Whether a String is Palindrome}
This program checks if a string is a palindrome.
\begin{lstlisting}[caption={Check Whether a String is Palindrome}]
#include <stdio.h>
int main() {
    char str[100];
    printf("Enter a string: ");
    scanf("%[^\n]", str);
    int len = 0, isPalindrome = 1;
    while (str[len] != '\0') len++;
    for (int i = 0, j = len - 1; i < j; i++, j--)
        if (str[i] != str[j]) {
            isPalindrome = 0;
            break;
        }
    printf("String is %s\n", isPalindrome ? "a palindrome" : "not a palindrome");
    return 0;
}
\end{lstlisting}
\clearpage

\subsection{Reverse Order of Words}
This program reverses the order of words in a string.
\begin{lstlisting}[caption={Reverse Order of Words}]
#include <stdio.h>
void reverse(char str[], int start, int end) {
    char temp;
    while (start < end) {
        temp = str[start];
        str[start++] = str[end];
        str[end--] = temp;
    }
}
int main() {
    char str[100];
    printf("Enter a string: ");
    scanf("%[^\n]", str);
    int len = 0;
    while (str[len] != '\0') len++;
    reverse(str, 0, len - 1);
    int start = 0;
    for (int i = 0; i <= len; i++) {
        if (str[i] == ' ' || str[i] == '\0') {
            reverse(str, start, i - 1);
            start =kowo i + 1;
        }
    }
    printf("Reversed words: %s\n", str);
    return 0;
}
\end{lstlisting}
\clearpage

\subsection{Find First Occurrence of a Character}
This program finds the first occurrence of a character in a string.
\begin{lstlisting}[caption={Find First Occurrence of a Character}]
#include <stdio.h>
int main() {
    char str[100], ch;
    printf("Enter a string: ");
    scanf("%[^\n]", str);
    getchar();
    printf("Enter a character: ");
    scanf("%c", &ch);
    int pos = -1;
    for (int i = 0; str[i] != '\0'; i++)
        if (str[i] == ch) {
            pos = i + 1;
            break;
        }
    if (pos == -1) printf("Character not found\n");
    else printf("First occurrence at position: %d\n", pos);
    return 0;
}
\end{lstlisting}
\clearpage

\subsection{Find Last Occurrence of a Character}
This program finds the last occurrence of a character in a string.
\begin{lstlisting}[caption={Find Last Occurrence of a Character}]
#include <stdio.h>
int main() {
    char str[100], ch;
    printf("Enter a string: ");
    scanf("%[^\n]", str);
    getchar();
    printf("Enter a character: ");
    scanf("%c", &ch);
    int pos = -1;
    for (int i = 0; str[i] != '\0'; i++)
        if (str[i] == ch) pos = i + 1;
    if (pos == -1) printf("Character not found\n");
    else printf("Last occurrence at position: %d\n", pos);
    return 0;
}
\end{lstlisting}
\clearpage

\subsection{Search All Occurrences of a Character}
This program searches for all occurrences of a character in a string.
\begin{lstlisting}[caption={Search All Occurrences of a Character}]
#include <stdio.h>
int main() {
    char str[100], ch;
    printf("Enter a string: ");
    scanf("%[^\n]", str);
    getchar();
    printf("Enter a character: ");
    scanf("%c", &ch);
    printf("Occurrences of '%c' at positions: ", ch);
    int found = 0;
    for (int i = 0; str[i] != '\0'; i++)
        if (str[i] == ch) {
            printf("%d ", i + 1);
            found = 1;
        }
    printf("\n");
    if (!found) printf("Character not found\n");
    return 0;
}
\end{lstlisting}
\clearpage

\subsection{Count Occurrences of a Character}
This program counts the occurrences of a character in a string.
\begin{lstlisting}[caption={Count Occurrences of a Character}]
#include <stdio.h>
int main() {
    char str[100], ch;
    printf("Enter a string: ");
    scanf("%[^\n]", str);
    getchar();
    printf("Enter a character: ");
    scanf("%c", &ch);
    int count = 0;
    for (int i = 0; str[i] != '\0'; i++)
        if (str[i] == ch) count++;
    printf("Character '%c' occurs %d times\n", ch, count);
    return 0;
}
\end{lstlisting}
\clearpage

\subsection{Find Highest Frequency Character}
This program finds the character with the highest frequency in a string.
\begin{lstlisting}[caption={Find Highest Frequency Character}]
#include <stdio.h>
int main() {
    char str[100];
    int freq[256] = {0};
    printf("Enter a string: ");
    scanf("%[^\n]", str);
    for (int i = 0; str[i] != '\0'; i++)
        freq[(unsigned char)str[i]]++;
    char maxChar = str[0];
    int maxFreq = freq[(unsigned char)str[0]];
    for (int i = 1; str[i] != '\0'; i++)
        if (freq[(unsigned char)str[i]] > maxFreq) {
            maxFreq = freq[(unsigned char)str[i]];
            maxChar = str[i];
        }
    printf("Highest frequency character: '%c' (%d times)\n", maxChar, maxFreq);
    return 0;
}
\end{lstlisting}
\clearpage

\subsection{Find Lowest Frequency Character}
This program finds the character with the lowest frequency in a string.
\begin{lstlisting}[caption={Find Lowest Frequency Character}]
#include <stdio.h>
int main() {
    char str[100];
    int freq[256] = {0};
    printf("Enter a string: ");
    scanf("%[^\n]", str);
    for (int i = 0; str[i] != '\0'; i++)
        freq[(unsigned char)str[i]]++;
    char minChar = str[0];
    int minFreq = freq[(unsigned char)str[0]];
    for (int i = 1; str[i] != '\0'; i++)
        if (freq[(unsigned char)str[i]] > 0 && freq[(unsigned char)str[i]] < minFreq) {
            minFreq = freq[(unsigned char)str[i]];
            minChar = str[i];
        }
    printf("Lowest frequency character: '%c' (%d times)\n", minChar, minFreq);
    return 0;
}
\end{lstlisting}
\clearpage

\subsection{Count Frequency of Each Character}
This program counts the frequency of each character in a string.
\begin{lstlisting}[caption={Count Frequency of Each Character}]
#include <stdio.h>
int main() {
    char str[100];
    int freq[256] = {0};
    printf("Enter a string: ");
    scanf("%[^\n]", str);
    for (int i = 0; str[i] != '\0'; i++)
        freq[(unsigned char)str[i]]++;
    printf("Character frequencies:\n");
    for (int i = 0; i < 256; i++)
        if (freq[i] > 0)
            printf("'%c': %d times\n", (char)i, freq[i]);
    return 0;
}
\end{lstlisting}
\clearpage

\subsection{Remove First Occurrence of a Character}
This program removes the first occurrence of a character from a string.
\begin{lstlisting}[caption={Remove First Occurrence of a Character}]
#include <stdio.h>
int main() {
    char str[100], ch;
    printf("Enter a string: ");
    scanf("%[^\n]", str);
    getchar();
    printf("Enter a character to remove: ");
    scanf("%c", &ch);
    int found = 0;
    for (int i = 0; str[i] != '\0'; i++)
        if (str[i] == ch && !found) {
            found = 1;
            for (int j = i; str[j] != '\0'; j++)
                str[j] = str[j + 1];
            break;
        }
    printf("String after removal: %s\n", str);
    return 0;
}
\end{lstlisting}
\clearpage

\subsection{Remove Last Occurrence of a Character}
This program removes the last occurrence of a character from a string.
\begin{lstlisting}[caption={Remove Last Occurrence of a Character}]
#include <stdio.h>
int main() {
    char str[100], ch;
    printf("Enter a string: ");
    scanf("%[^\n]", str);
    getchar();
    printf("Enter a character to remove: ");
    scanf("%c", &ch);
    int last = -1;
    for (int i = 0; str[i] != '\0'; i++)
        if (str[i] == ch) last = i;
    if (last != -1) {
        for (int i = last; str[i] != '\0'; i++)
            str[i] = str[i + 1];
    }
    printf("String after removal: %s\n", str);
    return 0;
}
\end{lstlisting}
\clearpage

\subsection{Remove All Occurrences of a Character}
This program removes all occurrences of a character from a string.
\begin{lstlisting}[caption={Remove All Occurrences of a Character}]
#include <stdio.h>
int main() {
    char str[100], ch;
    printf("Enter a string: ");
    scanf("%[^\n]", str);
    getchar();
    printf("Enter a character to remove: ");
    scanf("%c", &ch);
    int j = 0;
    for (int i = 0; str[i] != '\0'; i++)
        if (str[i] != ch)
            str[j++] = str[i];
    str[j] = '\0';
    printf("String after removal: %s\n", str);
    return 0;
}
\end{lstlisting}
\clearpage

\subsection{Remove All Repeated Characters}
This program removes all repeated characters from a string, keeping only the first occurrence.
\begin{lstlisting}[caption={Remove All Repeated Characters}]
#include <stdio.h>
int main() {
    char str[100];
    int seen[256] = {0};
    printf("Enter a string: ");
    scanf("%[^\n]", str);
    int j = 0;
    for (int i = 0; str[i] != '\0'; i++) {
        if (!seen[(unsigned char)str[i]]) {
            seen[(unsigned char)str[i]] = 1;
            str[j++] = str[i];
        }
    }
    str[j] = '\0';
    printf("String after removing duplicates: %s\n", str);
    return 0;
}
\end{lstlisting}
\clearpage

\subsection{Replace First Occurrence of a Character}
This program replaces the first occurrence of a character with another in a string.
\begin{lstlisting}[caption={Replace First Occurrence of a Character}]
#include <stdio.h>
int main() {
    char str[100], oldCh, newCh;
    printf("Enter a string: ");
    scanf("%[^\n]", str);
    getchar();
    printf("Enter character to replace: ");
    scanf("%c", &oldCh);
    getchar();
    printf("Enter new character: ");
    scanf("%c", &newCh);
    for (int i = 0; str[i] != '\0'; i++)
        if (str[i] == oldCh) {
            str[i] = newCh;
            break;
        }
    printf("String after replacement: %s\n", str);
    return 0;
}
\end{lstlisting}
\clearpage

\subsection{Replace Last Occurrence of a Character}
This program replaces the last occurrence of a character with another in a string.
\begin{lstlisting}[caption={Replace Last Occurrence of a Character}]
#include <stdio.h>
int main() {
    char str[100], oldCh, newCh;
    printf("Enter a string: ");
    scanf("%[^\n]", str);
    getchar();
    printf("Enter character to replace: ");
    scanf("%c", &oldCh);
    getchar();
    printf("Enter new character: ");
    scanf("%c", &newCh);
    int last = -1;
    for (int i = 0; str[i] != '\0'; i++)
        if (str[i] == oldCh) last = i;
    if (last != -1) str[last] = newCh;
    printf("String after replacement: %s\n", str);
    return 0;
}
\end{lstlisting}
\clearpage

\subsection{Replace All Occurrences of a Character}
This program replaces all occurrences of a character with another in a string.
\begin{lstlisting}[caption={Replace All Occurrences of a Character}]
#include <stdio.h>
int main() {
    char str[100], oldCh, newCh;
    printf("Enter a string: ");
    scanf("%[^\n]", str);
    getchar();
    printf("Enter character to replace: ");
    scanf("%c", &oldCh);
    getchar();
    printf("Enter new character: ");
    scanf("%c", &newCh);
    for (int i = 0; str[i] != '\0'; i++)
        if (str[i] == oldCh) str[i] = newCh;
    printf("String after replacement: %s\n", str);
    return 0;
}
\end{lstlisting}
\clearpage

\subsection{Find First Occurrence of a Word}
This program finds the first occurrence of a word in a string.
\begin{lstlisting}[caption={Find First Occurrence of a Word}]
#include <stdio.h>
#include <string.h>
int main() {
    char str[100], word[50];
    printf("Enter a string: ");
    scanf("%[^\n]", str);
    getchar();
    printf("Enter a word: ");
    scanf("%[^\n]", word);
    int len = strlen(word), pos = -1;
    for (int i = 0; str[i] != '\0'; i++) {
        int j = 0;
        while (j < len && str[i + j] == word[j]) j++;
        if (j == len && (i == 0 || str[i - 1] == ' ') && 
            (str[i + len] == ' ' || str[i + len] == '\0')) {
            pos = i + 1;
            break;
        }
    }
    if (pos == -1) printf("Word not found\n");
    else printf("First occurrence at position: %d\n", pos);
    return 0;
}
\end{lstlisting}
\clearpage

\subsection{Find Last Occurrence of a Word}
This program finds the last occurrence of a word in a string.
\begin{lstlisting}[caption={Find Last Occurrence of a Word}]
#include <stdio.h>
#include <string.h>
int main() {
    char str[100], word[50];
    printf("Enter a string: ");
    scanf("%[^\n]", str);
    getchar();
    printf("Enter a word: ");
    scanf("%[^\n]", word);
    int len = strlen(word), last = -1;
    for (int i = 0; str[i] != '\0'; i++) {
        int j = 0;
        while (j < len && str[i + j] == word[j]) j++;
        if (j == len && (i == 0 || str[i - 1] == ' ') && 
            (str[i + len] == ' ' || str[i + len] == '\0'))
            last = i + 1;
    }
    if (last == -1) printf("Word not found\n");
    else printf("Last occurrence at position: %d\n", last);
    return 0;
}
\end{lstlisting}
\clearpage

\subsection{Search All Occurrences of a Word}
This program searches for all occurrences of a word in a string.
\begin{lstlisting}[caption={Search All Occurrences of a Word}]
#include <stdio.h>
#include <string.h>
int main() {
    char str[100], word[50];
    printf("Enter a string: ");
    scanf("%[^\n]", str);
    getchar();
    printf("Enter a word: ");
    scanf("%[^\n]", word);
    int len = strlen(word), found = 0;
    printf("Occurrences of '%s' at positions: ", word);
    for (int i = 0; str[i] != '\0'; i++) {
        int j = 0;
        while (j < len && str[i + j] == word[j]) j++;
        if (j == len && (i == 0 || str[i - 1] == ' ') && 
            (str[i + len] == ' ' || str[i + len] == '\0')) {
            printf("%d ", i + 1);
            found = 1;
        }
    }
    printf("\n");
    if (!found) printf("Word not found\n");
    return 0;
}
\end{lstlisting}
\clearpage

\subsection{Count Occurrences of a Word}
This program counts the occurrences of a word in a string.
\begin{lstlisting}[caption={Count Occurrences of a Word}]
#include <stdio.h>
#include <string.h>
int main() {
    char str[100], word[50];
    printf("Enter a string: ");
    scanf("%[^\n]", str);
    getchar();
    printf("Enter a word: ");
    scanf("%[^\n]", word);
    int len = strlen(word), count = 0;
    for (int i = 0; str[i] != '\0'; i++) {
        int j = 0;
        while (j < len && str[i + j] == word[j]) j++;
        if (j == len && (i == 0 || str[i - 1] == ' ') && 
            (str[i + len] == ' ' || str[i + len] == '\0'))
            count++;
    }
    printf("Word '%s' occurs %d times\n", word, count);
    return 0;
}
\end{lstlisting}
\clearpage

\subsection{Remove First Occurrence of a Word}
This program removes the first occurrence of a word from a string.
\begin{lstlisting}[caption={Remove First Occurrence of a Word}]
#include <stdio.h>
#include <string.h>
int main() {
    char str[100], word[50];
    printf("Enter a string: ");
    scanf("%[^\n]", str);
    getchar();
    printf("Enter a word to remove: ");
    scanf("%[^\n]", word);
    int len = strlen(word), pos = -1;
    for (int i = 0; str[i] != '\0'; i++) {
        int j = 0;
        while (j < len && str[i + j] == word[j]) j++;
        if (j == len && (i == 0 || str[i - 1] == ' ') && 
            (str[i + len] == ' ' || str[i + len] == '\0')) {
            pos = i;
            break;
        }
    }
    if (pos != -1) {
        for (int i = pos; str[i + len] != '\0'; i++)
            str[i] = str[i + len];
        str[pos + len] = '\0';
    }
    printf("String after removal: %s\n", str);
    return 0;
}
\end{lstlisting}
\clearpage

\subsection{Remove Last Occurrence of a Word}
This program removes the last occurrence of a word from a string.
\begin{lstlisting}[caption={Remove Last Occurrence of a Word}]
#include <stdio.h>
#include <string.h>
int main() {
    char str[100], word[50];
    printf("Enter a string: ");
    scanf("%[^\n]", str);
    getchar();
    printf("Enter a word to remove: ");
    scanf("%[^\n]", word);
    int len = strlen(word), last = -1;
    for (int i = 0; str[i] != '\0'; i++) {
        int j = 0;
        while (j < len && str[i + j] == word[j]) j++;
        if (j == len && (i == 0 || str[i - 1] == ' ') && 
            (str[i + len] == ' ' || str[i + len] == '\0'))
            last = i;
    }
    if (last != -1) {
        for (int i = last; str[i + len] != '\0'; i++)
            str[i] = str[i + len];
        str[last + len] = '\0';
    }
    printf("String after removal: %s\n", str);
    return 0;
}
\end{lstlisting}
\clearpage

\subsection{Remove All Occurrences of a Word}
This program removes all occurrences of a word from a string.
\begin{lstlisting}[caption={Remove All Occurrences of a Word}]
#include <stdio.h>
#include <string.h>
int main() {
    char str[100], word[50], temp[100];
    printf("Enter a string: ");
    scanf("%[^\n]", str);
    getchar();
    printf("Enter a word to remove: ");
    scanf("%[^\n]", word);
    int len = strlen(word), j = 0;
    for (int i = 0; str[i] != '\0';) {
        int k = 0;
        while (k < len && str[i + k] == word[k]) k++;
        if (k == len && (i == 0 || str[i - 1] == ' ') && 
            (str[i + len] == ' ' || str[i + len] == '\0')) {
            i += len;
        } else {
            temp[j++] = str[i++];
        }
    }
    temp[j] = '\0';
    strcpy(str, temp);
    printf("String after removal: %s\n", str);
    return 0;
}
\end{lstlisting}
\clearpage

\subsection{Trim Leading White Space Characters}
This program trims leading whitespace characters from a string.
\begin{lstlisting}[caption={Trim Leading White Space Characters}]
#include <stdio.h>
int main() {
    char str[100];
    printf("Enter a string: ");
    scanf("%[^\n]", str);
    int i = 0;
    while (str[i] == ' ') i++;
    int j = 0;
    while (str[i] != '\0') str[j++] = str[i++];
    str[j] = '\0';
    printf("String after trimming leading whitespace: %s\n", str);
    return 0;
}
\end{lstlisting}
\clearpage

\subsection{Trim Trailing White Space Characters}
This program trims trailing whitespace characters from a string.
\begin{lstlisting}[caption={Trim Trailing White Space Characters}]
#include <stdio.h>
int main() {
    char str[100];
    printf("Enter a string: ");
    scanf("%[^\n]", str);
    int len = 0;
    while (str[len] != '\0') len++;
    while (len > 0 && str[len - 1] == ' ') len--;
    str[len] = '\0';
    printf("String after trimming trailing whitespace: %s\n", str);
    return 0;
}
\end{lstlisting}
\clearpage

\subsection{Trim Both Leading and Trailing White Space}
This program trims both leading and trailing whitespace characters from a string.
\begin{lstlisting}[caption={Trim Both Leading and Trailing White Space}]
#include <stdio.h>
int main() {
    char str[100];
    printf("Enter a string: ");
    scanf("%[^\n]", str);
    int i = 0;
    while (str[i] == ' ') i++;
    int len = 0;
    while (str[len] != '\0') len++;
    while (len > i && str[len - 1] == ' ') len--;
    int j = 0;
    for (int k = i; k < len; k++) str[j++] = str[k];
    str[j] = '\0';
    printf("String after trimming whitespace: %s\n", str);
    return 0;
}
\end{lstlisting}
\clearpage

\subsection{Remove All Extra Blank Spaces}
This program removes all extra blank spaces from a string, leaving single spaces between words.
\begin{lstlisting}[caption={Remove All Extra Blank Spaces}]
#include <stdio.h>
int main() {
    char str[100], temp[100];
    printf("Enter a string: ");
    scanf("%[^\n]", str);
    int j = 0, inWord = 0;
    for (int i = 0; str[i] != '\0'; i++) {
        if (str[i] != ' ') {
            temp[j++] = str[i];
            inWord = 1;
        } else if (inWord && str[i] == ' ') {
            temp[j++] = ' ';
            inWord = 0;
        }
    }
    if (j > 0 && temp[j - 1] == ' ') j--;
    temp[j] = '\0';
    for (int i = 0; temp[i] != '\0'; i++) str[i] = temp[i];
    str[j] = '\0';
    printf("String after removing extra spaces: %s\n", str);
    return 0;
}
\end{lstlisting}
\clearpage

% Starting programming exercises section
\section{Pointer Programming Exercises}

% Subsection for each program
\subsection{Create, Initialize, and Use Pointers}
This program demonstrates creating, initializing, and using pointers.
\begin{lstlisting}[caption={Create, Initialize, and Use Pointers}]
#include <stdio.h>
int main() {
    int num = 10;
    int *ptr;
    ptr = &num;
    printf("Value of num: %d\n", num);
    printf("Address of num: %p\n", (void*)&num);
    printf("Value at ptr: %d\n", *ptr);
    printf("Address stored in ptr: %p\n", (void*)ptr);
    *ptr = 20;
    printf("Modified value of num: %d\n", num);
    return 0;
}
\end{lstlisting}
\clearpage

\subsection{Add Two Numbers Using Pointers}
This program adds two numbers using pointers.
\begin{lstlisting}[caption={Add Two Numbers Using Pointers}]
#include <stdio.h>
int main() {
    int num1, num2;
    int *ptr1 = &num1, *ptr2 = &num2;
    printf("Enter two numbers: ");
    scanf("%d %d", ptr1, ptr2);
    int sum = *ptr1 + *ptr2;
    printf("Sum: %d\n", sum);
    return 0;
}
\end{lstlisting}
\clearpage

\subsection{Swap Two Numbers Using Pointers}
This program swaps two numbers using pointers.
\begin{lstlisting}[caption={Swap Two Numbers Using Pointers}]
#include <stdio.h>
void swap(int *a, int *b) {
    int temp = *a;
    *a = *b;
    *b = temp;
}
int main() {
    int num1, num2;
    printf("Enter two numbers: ");
    scanf("%d %d", &num1, &num2);
    printf("Before swap: num1 = %d, num2 = %d\n", num1, num2);
    swap(&num1, &num2);
    printf("After swap: num1 = %d, num2 = %d\n", num1, num2);
    return 0;
}
\end{lstlisting}
\clearpage

\subsection{Input and Print Array Elements Using Pointers}
This program inputs and prints array elements using pointers.
\begin{lstlisting}[caption={Input and Print Array Elements Using Pointers}]
#include <stdio.h>
int main() {
    int size;
    printf("Enter array size: ");
    scanf("%d", &size);
    int arr[size];
    int *ptr = arr;
    printf("Enter %d elements: ", size);
    for (int i = 0; i < size; i++) scanf("%d", ptr + i);
    printf("Array elements: ");
    for (int i = 0; i < size; i++) printf("%d ", *(ptr + i));
    printf("\n");
    return 0;
}
\end{lstlisting}
\clearpage

\subsection{Copy One Array to Another Using Pointers}
This program copies one array to another using pointers.
\begin{lstlisting}[caption={Copy One Array to Another Using Pointers}]
#include <stdio.h>
int main() {
    int size;
    printf("Enter array size: ");
    scanf("%d", &size);
    int arr1[size], arr2[size];
    int *ptr1 = arr1, *ptr2 = arr2;
    printf("Enter %d elements: ", size);
    for (int i = 0; i < size; i++) scanf("%d", ptr1 + i);
    for (int i = 0; i < size; i++) *(ptr2 + i) = *(ptr1 + i);
    printf("Copied array: ");
    for (int i = 0; i < size; i++) printf("%d ", *(ptr2 + i));
    printf("\n");
    return 0;
}
\end{lstlisting}
\clearpage

\subsection{Swap Two Arrays Using Pointers}
This program swaps two arrays using pointers.
\begin{lstlisting}[caption={Swap Two Arrays Using Pointers}]
#include <stdio.h>
int main() {
    int size;
    printf("Enter array size: ");
    scanf("%d", &size);
    int arr1[size], arr2[size];
    int *ptr1 = arr1, *ptr2 = arr2;
    printf("Enter first array elements: ");
    for (int i = 0; i < size; i++) scanf("%d", ptr1 + i);
    printf("Enter second array elements: ");
    for (int i = 0; i < size; i++) scanf("%d", ptr2 + i);
    for (int i = 0; i < size; i++) {
        int temp = *(ptr1 + i);
        *(ptr1 + i) = *(ptr2 + i);
        *(ptr2 + i) = temp;
    }
    printf("First array after swap: ");
    for (int i = 0; i < size; i++) printf("%d ", *(ptr1 + i));
    printf("\nSecond array after swap: ");
    for (int i = 0; i < size; i++) printf("%d ", *(ptr2 + i));
    printf("\n");
    return 0;
}
\end{lstlisting}
\clearpage

\subsection{Reverse an Array Using Pointers}
This program reverses an array using pointers.
\begin{lstlisting}[caption={Reverse an Array Using Pointers}]
#include <stdio.h>
int main() {
    int size;
    printf("Enter array size: ");
    scanf("%d", &size);
    int arr[size];
    int *ptr = arr;
    printf("Enter %d elements: ", size);
    for (int i = 0; i < size; i++) scanf("%d", ptr + i);
    for (int i = 0, j = size - 1; i < j; i++, j--) {
        int temp = *(ptr + i);
        *(ptr + i) = *(ptr + j);
        *(ptr + j) = temp;
    }
    printf("Reversed array: ");
    for (int i = 0; i < size; i++) printf("%d ", *(ptr + i));
    printf("\n");
    return 0;
}
\end{lstlisting}
\clearpage

\subsection{Search an Element in Array Using Pointers}
This program searches for an element in an array using pointers.
\begin{lstlisting}[caption={Search an Element in Array Using Pointers}]
#include <stdio.h>
int main() {
    int size, element;
    printf("Enter array size: ");
    scanf("%d", &size);
    int arr[size];
    int *ptr = arr;
    printf("Enter %d elements: ", size);
    for (int i = 0; i < size; i++) scanf("%d", ptr + i);
    printf("Enter element to search: ");
    scanf("%d", &element);
    int found = 0;
    for (int i = 0; i < size; i++)
        if (*(ptr + i) == element) {
            printf("Element found at position %d\n", i + 1);
            found = 1;
            break;
        }
    if (!found) printf("Element not found\n");
    return 0;
}
\end{lstlisting}
\clearpage

\subsection{Access Two Dimensional Array Using Pointers}
This program accesses a two-dimensional array using pointers.
\begin{lstlisting}[caption={Access Two Dimensional Array Using Pointers}]
#include <stdio.h>
int main() {
    int r, c;
    printf("Enter rows and columns: ");
    scanf("%d %d", &r, &c);
    int arr[r][c];
    int (*ptr)[c] = arr;
    printf("Enter %d elements:\n", r * c);
    for (int i = 0; i < r; i++)
        for (int j = 0; j < c; j++) scanf("%d", &ptr[i][j]);
    printf("Matrix elements:\n");
    for (int i = 0; i < r; i++) {
        for (int j = 0; j < c; j++) printf("%d ", ptr[i][j]);
        printf("\n");
    }
    return 0;
}
\end{lstlisting}
\clearpage

\subsection{Add Two Matrices Using Pointers}
This program adds two matrices using pointers.
\begin{lstlisting}[caption={Add Two Matrices Using Pointers}]
#include <stdio.h>
int main() {
    int r, c;
    printf("Enter rows and columns: ");
    scanf("%d %d", &r, &c);
    int a[r][c], b[r][c], sum[r][c];
    int (*ptr1)[c] = a, (*ptr2)[c] = b, (*ptr3)[c] = sum;
    printf("Enter first matrix elements:\n");
    for (int i = 0; i < r; i++)
        for (int j = 0; j < c; j++) scanf("%d", &ptr1[i][j]);
    printf("Enter second matrix elements:\n");
    for (int i = 0; i < r; i++)
        for (int j = 0; j < c; j++) scanf("%d", &ptr2[i][j]);
    for (int i = 0; i < r; i++)
        for (int j = 0; j < c; j++) ptr3[i][j] = ptr1[i][j] + ptr2[i][j];
    printf("Sum of matrices:\n");
    for (int i = 0; i < r; i++) {
        for (int j = 0; j < c; j++) printf("%d ", ptr3[i][j]);
        printf("\n");
    }
    return 0;
}
\end{lstlisting}
\clearpage

\subsection{Multiply Two Matrices Using Pointers}
This program multiplies two matrices using pointers.
\begin{lstlisting}[caption={Multiply Two Matrices Using Pointers}]
#include <stdio.h>
int main() {
    int r1, c1, r2, c2;
    printf("Enter rows and columns of first matrix: ");
    scanf("%d %d", &r1, &c1);
    printf("Enter rows and columns of second matrix: ");
    scanf("%d %d", &r2, &c2);
    if (c1 != r2) {
        printf("Matrix multiplication not possible\n");
        return 1;
    }
    int a[r1][c1], b[r2][c2], product[r1][c2];
    int (*ptr1)[c1] = a, (*ptr2)[c2] = b, (*ptr3)[c2] = product;
    printf("Enter first matrix elements:\n");
    for (int i = 0; i < r1; i++)
        for (int j = 0; j < c1; j++) scanf("%d", &ptr1[i][j]);
    printf("Enter second matrix elements:\n");
    for (int i = 0; i < r2; i++)
        for (int j = 0; j < c2; j++) scanf("%d", &ptr2[i][j]);
    for (int i = 0; i < r1; i++)
        for (int j = 0; j < c2; j++) {
            ptr3[i][j] = 0;
            for (int k = 0; k < c1; k++)
                ptr3[i][j] += ptr1[i][k] * ptr2[k][j];
        }
    printf("Product of matrices:\n");
    for (int i = 0; i < r1; i++) {
        for (int j = 0; j < c2; j++) printf("%d ", ptr3[i][j]);
        printf("\n");
    }
    return 0;
}
\end{lstlisting}
\clearpage

\subsection{Find Length of String Using Pointers}
This program finds the length of a string using pointers.
\begin{lstlisting}[caption={Find Length of String Using Pointers}]
#include <stdio.h>
int main() {
    char str[100];
    char *ptr = str;
    printf("Enter a string: ");
    scanf("%[^\n]", str);
    int len = 0;
    while (*(ptr + len) != '\0') len++;
    printf("Length of string: %d\n", len);
    return 0;
}
\end{lstlisting}
\clearpage

\subsection{Copy One String to Another Using Pointers}
This program copies one string to another using pointers.
\begin{lstlisting}[caption={Copy One String to Another Using Pointers}]
#include <stdio.h>
int main() {
    char src[100], dest[100];
    char *ptr1 = src, *ptr2 = dest;
    printf("Enter a string: ");
    scanf("%[^\n]", src);
    while (*ptr1 != '\0') {
        *ptr2 = *ptr1;
        ptr1++;
        ptr2++;
    }
    *ptr2 = '\0';
    printf("Copied string: %s\n", dest);
    return 0;
}
\end{lstlisting}
\clearpage

\subsection{Concatenate Two Strings Using Pointers}
This program concatenates two strings using pointers.
\begin{lstlisting}[caption={Concatenate Two Strings Using Pointers}]
#include <stdio.h>
int main() {
    char str1[200], str2[100];
    char *ptr1 = str1, *ptr2 = str2;
    printf("Enter first string: ");
    scanf("%[^\n]", str1);
    getchar();
    printf("Enter second string: ");
    scanf("%[^\n]", str2);
    while (*ptr1 != '\0') ptr1++;
    while (*ptr2 != '\0') {
        *ptr1 = *ptr2;
        ptr1++;
        ptr2++;
    }
    *ptr1 = '\0';
    printf("Concatenated string: %s\n", str1);
    return 0;
}
\end{lstlisting}
\clearpage

\subsection{Compare Two Strings Using Pointers}
This program compares two strings using pointers.
\begin{lstlisting}[caption={Compare Two Strings Using Pointers}]
#include <stdio.h>
int main() {
    char str1[100], str2[100];
    char *ptr1 = str1, *ptr2 = str2;
    printf("Enter first string: ");
    scanf("%[^\n]", str1);
    getchar();
    printf("Enter second string: ");
    scanf("%[^\n]", str2);
    int equal = 1;
    while (*ptr1 != '\0' || *ptr2 != '\0') {
        if (*ptr1 != *ptr2) {
            equal = 0;
            break;
        }
        ptr1++;
        ptr2++;
    }
    printf("Strings are %s\n", equal ? "equal" : "not equal");
    return 0;
}
\end{lstlisting}
\clearpage

\subsection{Find Reverse of a String Using Pointers}
This program reverses a string using pointers.
\begin{lstlisting}[caption={Find Reverse of a String Using Pointers}]
#include <stdio.h>
int main() {
    char str[100];
    char *start = str, *end;
    printf("Enter a string: ");
    scanf("%[^\n]", str);
    end = str;
    while (*end != '\0') end++;
    end--;
    char temp;
    while (start < end) {
        temp = *start;
        *start = *end;
        *end = temp;
        start++;
        end--;
    }
    printf("Reversed string: %s\n", str);
    return 0;
}
\end{lstlisting}
\clearpage

\subsection{Sort Array Using Pointers}
This program sorts an array in ascending order using pointers.
\begin{lstlisting}[caption={Sort Array Using Pointers}]
#include <stdio.h>
int main() {
    int size;
    printf("Enter array size: ");
    scanf("%d", &size);
    int arr[size];
    int *ptr = arr;
    printf("Enter %d elements: ", size);
    for (int i = 0; i < size; i++) scanf("%d", ptr + i);
    for (int i = 0; i < size - 1; i++)
        for (int j = 0; j < size - i - 1; j++)
            if (*(ptr + j) > *(ptr + j + 1)) {
                int temp = *(ptr + j);
                *(ptr + j) = *(ptr + j + 1);
                *(ptr + j + 1) = temp;
            }
    printf("Sorted array: ");
    for (int i = 0; i < size; i++) printf("%d ", *(ptr + i));
    printf("\n");
    return 0;
}
\end{lstlisting}
\clearpage

\subsection{Return Multiple Values from Function Using Pointers}
This program returns multiple values (sum and product) from a function using pointers.
\begin{lstlisting}[caption={Return Multiple Values from Function Using Pointers}]
#include <stdio.h>
void compute(int a, int b, int *sum, int *product) {
    *sum = a + b;
    *product = a * b;
}
int main() {
    int num1, num2, sum, product;
    printf("Enter two numbers: ");
    scanf("%d %d", &num1, &num2);
    compute(num1, num2, &sum, &product);
    printf("Sum: %d\nProduct: %d\n", sum, product);
    return 0;
}
\end{lstlisting}
\clearpage

% Starting programming exercises section
\section{File Handling Programming Exercises}

% Subsection for each program
\subsection{Create a File and Write Contents}
This program creates a file, writes content, saves, and closes it.
\begin{lstlisting}[caption={Create a File and Write Contents}]
#include <stdio.h>
int main() {
    FILE *fp;
    char content[100];
    fp = fopen("sample.txt", "w");
    if (fp == NULL) {
        printf("Cannot open file\n");
        return 1;
    }
    printf("Enter content to write: ");
    scanf("%[^\n]", content);
    fprintf(fp, "%s", content);
    fclose(fp);
    printf("Content written to file\n");
    return 0;
}
\end{lstlisting}
\clearpage

\subsection{Read File Contents and Display}
This program reads file contents and displays them on the console.
\begin{lstlisting}[caption={Read File Contents and Display}]
#include <stdio.h>
int main() {
    FILE *fp;
    char ch;
    fp = fopen("sample.txt", "r");
    if (fp == NULL) {
        printf("Cannot open file\n");
        return 1;
    }
    printf("File contents:\n");
    while ((ch = fgetc(fp)) != EOF) {
        putchar(ch);
    }
    fclose(fp);
    return 0;
}
\end{lstlisting}
\clearpage

\subsection{Write Even, Odd, and Prime Numbers to Separate Files}
This program reads numbers from a file and writes even, odd, and prime numbers to separate files.
\begin{lstlisting}[caption={Write Even, Odd, and Prime Numbers to Separate Files}]
#include <stdio.h>
int isPrime(int n) {
    if (n <= 1) return 0;
    for (int i = 2; i * i <= n; i++)
        if (n % i == 0) return 0;
    return 1;
}
int main() {
    FILE *fin, *feven, *fodd, *fprime;
    int num;
    fin = fopen("numbers.txt", "r");
    feven = fopen("even.txt", "w");
    fodd = fopen("odd.txt", "w");
    fprime = fopen("prime.txt", "w");
    if (fin == NULL || feven == NULL || fodd == NULL || fprime == NULL) {
        printf("File operation failed\n");
        return 1;
    }
    while (fscanf(fin, "%d", &num) != EOF) {
        if (num % 2 == 0) fprintf(feven, "%d\n", num);
        else fprintf(fodd, "%d\n", num);
        if (isPrime(num)) fprintf(fprime, "%d\n", num);
    }
    fclose(fin);
    fclose(feven);
    fclose(fodd);
    fclose(fprime);
    printf("Numbers separated into files\n");
    return 0;
}
\end{lstlisting}
\clearpage

\subsection{Append Content to a File}
This program appends content to an existing file.
\begin{lstlisting}[caption={Append Content to a File}]
#include <stdio.h>
int main() {
    FILE *fp;
    char content[100];
    fp = fopen("sample.txt", "a");
    if (fp == NULL) {
        printf("Cannot open file\n");
        return 1;
    }
    printf("Enter content to append: ");
    scanf("%[^\n]", content);
    fprintf(fp, "\n%s", content);
    fclose(fp);
    printf("Content appended to file\n");
    return 0;
}
\end{lstlisting}
\clearpage

\subsection{Compare Two Files}
This program compares the contents of two files.
\begin{lstlisting}[caption={Compare Two Files}]
#include <stdio.h>
int main() {
    FILE *fp1, *fp2;
    char ch1, ch2;
    fp1 = fopen("file1.txt", "r");
    fp2 = fopen("file2.txt", "r");
    if (fp1 == NULL || fp2 == NULL) {
        printf("Cannot open file(s)\n");
        return 1;
    }
    int identical = 1;
    while ((ch1 = fgetc(fp1)) != EOF && (ch2 = fgetc(fp2)) != EOF) {
        if (ch1 != ch2) {
            identical = 0;
            break;
        }
    }
    if (identical && fgetc(fp1) == EOF && fgetc(fp2) == EOF)
        printf("Files are identical\n");
    else
        printf("Files are different\n");
    fclose(fp1);
    fclose(fp2);
    return 0;
}
\end{lstlisting}
\clearpage

\subsection{Copy Contents from One File to Another}
This program copies contents from one file to another.
\begin{lstlisting}[caption={Copy Contents from One File to Another}]
#include <stdio.h>
int main() {
    FILE *fsrc, *fdest;
    char ch;
    fsrc = fopen("source.txt", "r");
    fdest = fopen("destination.txt", "w");
    if (fsrc == NULL || fdest == NULL) {
        printf("File operation failed\n");
        return 1;
    }
    while ((ch = fgetc(fsrc)) != EOF) {
        fputc(ch, fdest);
    }
    fclose(fsrc);
    fclose(fdest);
    printf("File copied successfully\n");
    return 0;
}
\end{lstlisting}
\clearpage

\subsection{Merge Two Files to a Third File}
This program merges contents of two files into a third file.
\begin{lstlisting}[caption={Merge Two Files to a Third File}]
#include <stdio.h>
int main() {
    FILE *fp1, *fp2, *fmerged;
    char ch;
    fp1 = fopen("file1.txt", "r");
    fp2 = fopen("file2.txt", "r");
    fmerged = fopen("merged.txt", "w");
    if (fp1 == NULL || fp2 == NULL || fmerged == NULL) {
        printf("File operation failed\n");
        return 1;
    }
    while ((ch = fgetc(fp1)) != EOF) fputc(ch, fmerged);
    fputc('\n', fmerged);
    while ((ch = fgetc(fp2)) != EOF) fputc(ch, fmerged);
    fclose(fp1);
    fclose(fp2);
    fclose(fmerged);
    printf("Files merged successfully\n");
    return 0;
}
\end{lstlisting}
\clearpage

\subsection{Count Characters, Words, and Lines}
This program counts characters, words, and lines in a text file.
\begin{lstlisting}[caption={Count Characters, Words, and Lines}]
#include <stdio.h>
int main() {
    FILE *fp;
    char ch;
    int chars = 0, words = 0, lines = 0, inWord = 0;
    fp = fopen("sample.txt", "r");
    if (fp == NULL) {
        printf("Cannot open file\n");
        return 1;
    }
    while ((ch = fgetc(fp)) != EOF) {
        chars++;
        if (ch == '\n') lines++;
        if (ch != ' ' && ch != '\n' && !inWord) {
            inWord = 1;
            words++;
        } else if (ch == ' ' || ch == '\n') {
            inWord = 0;
        }
    }
    if (chars > 0 && ch != '\n') lines++;
    fclose(fp);
    printf("Characters: %d\nWords: %d\nLines: %d\n", chars, words, lines);
    return 0;
}
\end{lstlisting}
\clearpage

\subsection{Remove a Word from Text File}
This program removes all occurrences of a specific word from a text file.
\begin{lstlisting}[caption={Remove a Word from Text File}]
#include <stdio.h>
#include <string.h>
int main() {
    FILE *fin, *fout;
    char word[50], temp[100], buffer[100];
    printf("Enter word to remove: ");
    scanf("%s", word);
    fin = fopen("input.txt", "r");
    fout = fopen("temp.txt", "w");
    if (fin == NULL || fout == NULL) {
        printf("File operation failed\n");
        return 1;
    }
    while (fscanf(fin, "%s", buffer) != EOF) {
        if (strcmp(buffer, word) != 0) {
            fprintf(fout, "%s ", buffer);
        }
    }
    fclose(fin);
    fclose(fout);
    remove("input.txt");
    rename("temp.txt", "input.txt");
    printf("Word removed from file\n");
    return 0;
}
\end{lstlisting}
\clearpage

\subsection{Remove Specific Line from a Text File}
This program removes a specific line from a text file based on line number.
\begin{lstlisting}[caption={Remove Specific Line from a Text File}]
#include <stdio.h>
int main() {
    FILE *fin, *fout;
    char line[100];
    int lineNum, currentLine = 0;
    printf("Enter line number to remove: ");
    scanf("%d", &lineNum);
    fin = fopen("input.txt", "r");
    fout = fopen("temp.txt", "w");
    if (fin == NULL || fout == NULL) {
        printf("File operation failed\n");
        return 1;
    }
    while (fgets(line, sizeof(line), fin)) {
        currentLine++;
        if (currentLine != lineNum) {
            fputs(line, fout);
        }
    }
    fclose(fin);
    fclose(fout);
    remove("input.txt");
    rename("temp.txt", "input.txt");
    printf("Line removed from file\n");
    return 0;
}
\end{lstlisting}
\clearpage

\subsection{Remove Empty Lines from a Text File}
This program removes empty lines from a text file.
\begin{lstlisting}[caption={Remove Empty Lines from a Text File}]
#include <stdio.h>
#include <string.h>
int main() {
    FILE *fin, *fout;
    char line[100];
    fin = fopen("input.txt", "r");
    fout = fopen("temp.txt", "w");
    if (fin == NULL || fout == NULL) {
        printf("File operation failed\n");
        return 1;
    }
    while (fgets(line, sizeof(line), fin)) {
        int len = strlen(line);
        if (len > 1 || (len == 1 && line[0] != '\n')) {
            fputs(line, fout);
        }
    }
    fclose(fin);
    fclose(fout);
    remove("input.txt");
    rename("temp.txt", "input.txt");
    printf("Empty lines removed from file\n");
    return 0;
}
\end{lstlisting}
\clearpage

\subsection{Find Occurrence of a Word in a Text File}
This program finds the first occurrence of a word in a text file.
\begin{lstlisting}[caption={Find Occurrence of a Word in a Text File}]
#include <stdio.h>
#include <string.h>
int main() {
    FILE *fp;
    char word[50], buffer[50];
    int line = 0, found = 0;
    printf("Enter word to find: ");
    scanf("%s", word);
    fp = fopen("sample.txt", "r");
    if (fp == NULL) {
        printf("Cannot open file\n");
        return 1;
    }
    while (fscanf(fp, "%s", buffer) != EOF) {
        if (fgetc(fp) == '\n') line++;
        if (strcmp(buffer, word) == 0) {
            printf("Word '%s' found at line %d\n", word, line + 1);
            found = 1;
            break;
        }
    }
    if (!found) printf("Word not found\n");
    fclose(fp);
    return 0;
}
\end{lstlisting}
\clearpage

\subsection{Count Occurrences of a Word in a Text File}
This program counts the occurrences of a specific word in a text file.
\begin{lstlisting}[caption={Count Occurrences of a Word in a Text File}]
#include <stdio.h>
#include <string.h>
int main() {
    FILE *fp;
    char word[50], buffer[50];
    int count = 0;
    printf("Enter word to count: ");
    scanf("%s", word);
    fp = fopen("sample.txt", "r");
    if (fp == NULL) {
        printf("Cannot open file\n");
        return 1;
    }
    while (fscanf(fp, "%s", buffer) != EOF) {
        if (strcmp(buffer, word) == 0) {
            count++;
        }
    }
    printf("Word '%s' occurs %d times\n", word, count);
    fclose(fp);
    return 0;
}
\end{lstlisting}
\clearpage

\subsection{Count Occurrences of All Words in a Text File}
This program counts the occurrences of all unique words in a text file.
\begin{lstlisting}[caption={Count Occurrences of All Words in a Text File}]
#include <stdio.h>
#include <string.h>
#define MAX_WORDS 100
#define MAX_LEN 50
int main() {
    FILE *fp;
    char words[MAX_WORDS][MAX_LEN];
    int count[MAX_WORDS] = {0}, numWords = 0;
    char buffer[50];
    fp = fopen("sample.txt", "r");
    if (fp == NULL) {
        printf("Cannot open file\n");
        return 1;
    }
    while (fscanf(fp, "%s", buffer) != EOF) {
        int found = 0;
        for (int i = 0; i < numWords; i++) {
            if (strcmp(buffer, words[i]) == 0) {
                count[i]++;
                found = 1;
                break;
            }
        }
        if (!found) {
            strcpy(words[numWords], buffer);
            count[numWords]++;
            numWords++;
        }
    }
    printf("Word frequencies:\n");
    for (int i = 0; i < numWords; i++) {
        printf("%s: %d\n", words[i], count[i]);
    }
    fclose(fp);
    return 0;
}
\end{lstlisting}
\clearpage

\subsection{Find and Replace a Word in a Text File}
This program finds and replaces all occurrences of a word in a text file.
\begin{lstlisting}[caption={Find and Replace a Word in a Text File}]
#include <stdio.h>
#include <string.h>
int main() {
    FILE *fin, *fout;
    char oldWord[50], newWord[50], buffer[50];
    printf("Enter word to replace: ");
    scanf("%s", oldWord);
    printf("Enter new word: ");
    scanf("%s", newWord);
    fin = fopen("input.txt", "r");
    fout = fopen("temp.txt", "w");
    if (fin == NULL || fout == NULL) {
        printf("File operation failed\n");
        return 1;
    }
    while (fscanf(fin, "%s", buffer) != EOF) {
        if (strcmp(buffer, oldWord) == 0) {
            fprintf(fout, "%s ", newWord);
        } else {
            fprintf(fout, "%s ", buffer);
        }
    }
    fclose(fin);
    fclose(fout);
    remove("input.txt");
    rename("temp.txt", "input.txt");
    printf("Word replaced in file\n");
    return 0;
}
\end{lstlisting}
\clearpage

\subsection{Replace Specific Line in a Text File}
This program replaces a specific line in a text file with new content.
\begin{lstlisting}[caption={Replace Specific Line in a Text File}]
#include <stdio.h>
int main() {
    FILE *fin, *fout;
    char line[100], newLine[100];
    int lineNum, currentLine = 0;
    printf("Enter line number to replace: ");
    scanf("%d", &lineNum);
    getchar();
    printf("Enter new line content: ");
    scanf("%[^\n]", newLine);
    fin = fopen("input.txt", "r");
    fout = fopen("temp.txt", "w");
    if (fin == NULL || fout == NULL) {
        printf("File operation failed\n");
        return 1;
    }
    while (fgets(line, sizeof(line), fin)) {
        currentLine++;
        if (currentLine == lineNum) {
            fprintf(fout, "%s\n", newLine);
        } else {
            fputs(line, fout);
        }
    }
    fclose(fin);
    fclose(fout);
    remove("input.txt");
    rename("temp.txt", "input.txt");
    printf("Line replaced in file\n");
    return 0;
}
\end{lstlisting}
\clearpage

\subsection{Print Source Code of Same Program}
This program prints its own source code.
\begin{lstlisting}[caption={Print Source Code of Same Program}]
#include <stdio.h>
int main() {
    FILE *fp;
    char ch;
    fp = fopen(__FILE__, "r");
    if (fp == NULL) {
        printf("Cannot open file\n");
        return 1;
    }
    printf("Source code:\n");
    while ((ch = fgetc(fp)) != EOF) {
        putchar(ch);
    }
    fclose(fp);
    return 0;
}
\end{lstlisting}
\clearpage

\subsection{Convert Uppercase to Lowercase and Vice Versa}
This program converts uppercase to lowercase and vice versa in a text file.
\begin{lstlisting}[caption={Convert Uppercase to Lowercase and Vice Versa}]
#include <stdio.h>
int main() {
    FILE *fin, *fout;
    char ch;
    fin = fopen("input.txt", "r");
    fout = fopen("temp.txt", "w");
    if (fin == NULL || fout == NULL) {
        printf("File operation failed\n");
        return 1;
    }
    while ((ch = fgetc(fin)) != EOF) {
        if (ch >= 'A' && ch <= 'Z') {
            fputc(ch + 32, fout);
        } else if (ch >= 'a' && ch <= 'z') {
            fputc(ch - 32, fout);
        } else {
            fputc(ch, fout);
        }
    }
    fclose(fin);
    fclose(fout);
    remove("input.txt");
    rename("temp.txt", "input.txt");
    printf("Case conversion completed\n");
    return 0;
}
\end{lstlisting}
\clearpage

\subsection{Find Properties of a File Using stat()}
This program displays properties of a file using the stat() function.
\begin{lstlisting}[caption={Find Properties of a File Using stat()}]
#include <stdio.h>
#include <sys/stat.h>
#include <time.h>
int main() {
    struct stat fileStat;
    char filename[100];
    printf("Enter filename: ");
    scanf("%s", filename);
    if (stat(filename, &fileStat) == -1) {
        printf("Cannot access file\n");
        return 1;
    }
    printf("File size: %lld bytes\n", (long long)fileStat.st_size);
    printf("Last modified: %s", ctime(&fileStat.st_mtime));
    printf("Permissions: %o\n", fileStat.st_mode & 0777);
    printf("File type: %s\n", S_ISDIR(fileStat.st_mode) ? "Directory" : "Regular file");
    return 0;
}
\end{lstlisting}
\clearpage

\subsection{Check if a File or Directory Exists}
This program checks if a file or directory exists.
\begin{lstlisting}[caption={Check if a File or Directory Exists}]
#include <stdio.h>
#include <sys/stat.h>
int main() {
    struct stat fileStat;
    char path[100];
    printf("Enter file or directory path: ");
    scanf("%s", path);
    if (stat(path, &fileStat) == 0) {
        if (S_ISDIR(fileStat.st_mode)) {
            printf("Directory exists\n");
        } else {
            printf("File exists\n");
        }
    } else {
        printf("File or directory does not exist\n");
    }
    return 0;
}
\end{lstlisting}
\clearpage

\subsection{Rename a File Using rename()}
This program renames a file using the rename() function.
\begin{lstlisting}[caption={Rename a File Using rename()}]
#include <stdio.h>
int main() {
    char oldName[100], newName[100];
    printf("Enter current filename: ");
    scanf("%s", oldName);
    printf("Enter new filename: ");
    scanf("%s", newName);
    if (rename(oldName, newName) == 0) {
        printf("File renamed successfully\n");
    } else {
        printf("Error renaming file\n");
    }
    return 0;
}
\end{lstlisting}
\clearpage

\subsection{List All Files and Sub-Directories Recursively}
This program lists all files and sub-directories recursively.
\begin{lstlisting}[caption={List All Files and Sub-Directories Recursively}]
#include <stdio.h>
#include <dirent.h>
#include <string.h>
void listDir(const char *path, int level) {
    DIR *dir;
    struct dirent *entry;
    if (!(dir = opendir(path))) return;
    while ((entry = readdir(dir)) != NULL) {
        if (strcmp(entry->d_name, ".") == 0 || strcmp(entry->d_name, "..") == 0) continue;
        for (int i = 0; i < level; i++) printf("  ");
        printf("%s\n", entry->d_name);
        if (entry->d_type == DT_DIR) {
            char newPath[1024];
            snprintf(newPath, sizeof(newPath), "%s/%s", path, entry->d_name);
            listDir(newPath, level + 1);
        }
    }
    closedir(dir);
}
int main() {
    char path[100];
    printf("Enter directory path: ");
    scanf("%s", path);
    printf("Directory contents:\n");
    listDir(path, 0);
    return 0;
}
\end{lstlisting}
\clearpage

% Starting programming exercises section
\section{Macro Programming Exercises}

% Subsection for each program
\subsection{Create Custom Header File}
This program demonstrates the use of a custom header file \texttt{mymacros.h} with a simple macro.
\begin{lstlisting}[caption={Create Custom Header File}]
#include <stdio.h>
#include "mymacros.h"
int main() {
    printf("Value of PI from custom header: %.5f\n", PI);
    return 0;
}
\end{lstlisting}
\clearpage

\subsection{Define, Undefine, and Redefine a Macro}
This program demonstrates defining, undefining, and redefining a macro.
\begin{lstlisting}[caption={Define, Undefine, and Redefine a Macro}]
#include <stdio.h>
#define SIZE 10
int main() {
    printf("Initial SIZE: %d\n", SIZE);
    #undef SIZE
    #define SIZE 20
    printf("Redefined SIZE: %d\n", SIZE);
    #undef SIZE
    #ifdef SIZE
        printf("SIZE is defined\n");
    #else
        printf("SIZE is undefined\n");
    #endif
    return 0;
}
\end{lstlisting}
\clearpage

\subsection{Find Sum Using Macro}
This program uses a macro to find the sum of two numbers.
\begin{lstlisting}[caption={Find Sum Using Macro}]
#include <stdio.h>
#define SUM(a, b) ((a) + (b))
int main() {
    int x, y;
    printf("Enter two numbers: ");
    scanf("%d %d", &x, &y);
    printf("Sum of %d and %d is: %d\n", x, y, SUM(x, y));
    return 0;
}
\end{lstlisting}
\clearpage

\subsection{Find Square and Cube Using Macro}
This program uses macros to find the square and cube of a number, defined in \texttt{mymacros.h}.
\begin{lstlisting}[caption={Find Square and Cube Using Macro}]
#include <stdio.h>
#include "mymacros.h"
int main() {
    int num;
    printf("Enter a number: ");
    scanf("%d", &num);
    printf("Square of %d is: %d\n", num, SQUARE(num));
    printf("Cube of %d is: %d\n", num, CUBE(num));
    return 0;
}
\end{lstlisting}
\clearpage

\subsection{Check Even/Odd Using Macro}
This program uses a macro to check if a number is even or odd, defined in \texttt{mymacros.h}.
\begin{lstlisting}[caption={Check Even/Odd Using Macro}]
#include <stdio.h>
#include "mymacros.h"
int main() {
    int num;
    printf("Enter a number: ");
    scanf("%d", &num);
    if (IS_EVEN(num))
        printf("%d is even\n", num);
    else
        printf("%d is odd\n", num);
    return 0;
}
\end{lstlisting}
\clearpage

\subsection{Find Maximum or Minimum Using Macro}
This program uses macros to find the maximum and minimum of two numbers, defined in \texttt{mymacros.h}.
\begin{lstlisting}[caption={Find Maximum or Minimum Using Macro}]
#include <stdio.h>
#include "mymacros.h"
int main() {
    int a, b;
    printf("Enter two numbers: ");
    scanf("%d %d", &a, &b);
    printf("Maximum of %d and %d is: %d\n", a, b, MAX(a, b));
    printf("Minimum of %d and %d is: %d\n", a, b, MIN(a, b));
    return 0;
}
\end{lstlisting}
\clearpage

\subsection{Check Lowercase/Uppercase Using Macro}
This program uses macros to check if a character is lowercase or uppercase, defined in \texttt{mymacros.h}.
\begin{lstlisting}[caption={Check Lowercase/Uppercase Using Macro}]
#include <stdio.h>
#include "mymacros.h"
int main() {
    char ch;
    printf("Enter a character: ");
    scanf(" %c", &ch);
    if (IS_LOWER(ch))
        printf("%c is lowercase\n", ch);
    else if (IS_UPPER(ch))
        printf("%c is uppercase\n", ch);
    else
        printf("%c is not a letter\n", ch);
    return 0;
}
\end{lstlisting}
\clearpage

\subsection{Swap Two Numbers Using Macro}
This program uses a macro to swap two numbers, defined in \texttt{mymacros.h}.
\begin{lstlisting}[caption={Swap Two Numbers Using Macro}]
#include <stdio.h>
#include "mymacros.h"
int main() {
    int a, b;
    printf("Enter two numbers: ");
    scanf("%d %d", &a, &b);
    printf("Before swap: a = %d, b = %d\n", a, b);
    SWAP(a, b);
    printf("After swap: a = %d, b = %d\n", a, b);
    return 0;
}
\end{lstlisting}
\clearpage

\subsection{Multiline Macros}
This program demonstrates a multiline macro to calculate the area and perimeter of a rectangle.
\begin{lstlisting}[caption={Multiline Macros}]
#include <stdio.h>
#define RECTANGLE_CALC(length, width, area, perimeter) \
    do { \
        (area) = (length) * (width); \
        (perimeter) = 2 * ((length) + (width)); \
    } while (0)
int main() {
    int length, width, area, perimeter;
    printf("Enter length and width of rectangle: ");
    scanf("%d %d", &length, &width);
    RECTANGLE_CALC(length, width, area, perimeter);
    printf("Area: %d\nPerimeter: %d\n", area, perimeter);
    return 0;
}
\end{lstlisting}
\clearpage

% Conclusion
\section{Conclusion}
This document provides C programs for 18 Basic Programming Exercises, 16 Bitwise Operators Exercises, 5 Conditional Operators Exercises, 21 If-else Exercises, 8 Switch Case Exercises, 50 Loop Exercises, 35 Star Pattern Exercises, 59 Number Pattern Exercises, 24 Function And Recursion Exercises, 18 Matrix Exercises, 38 String Exercises, 18 Pointer Exercises, 22 File Handling Exercises, 9 Macro And Pre-processor Directive Exercises, each presented on a separate page. To see the output of each program, compile and run them using a C compiler. For additional assistance or further exercises, contact the document author.

\end{document}